\chapter{Introduction}
เรียนเลขไปทำไม

หลายคนอาจจะคิดว่า เราลงเรียนแคลคูลัสมาแล้วตั้ง 2--3 ตัว ทำไมยังต้องมาเรียนวิชาวิยุตคณิตนี้อีก จึงขอย้อนไปมองว่า ทำไมวิศวกรจึงต้องเรียนแคลคูลัส

หากเป็นวิศวกรสาขาอื่นๆ เช่น วิศวกรไฟฟ้า ปัญหาที่พบเจอส่วนใหญ่จะเกี่ยวกับกระแสสัญญาณ หรือคลื่นความถี่ ซึ่งเป็นปริมาณที่\emph{ต่อเนื่อง} วิชา\emph{แคลคูลัส} (calculus) พูดถึงสิ่งที่ต่อเนื่องเหล่านี้

แต่หากเป็นวิศวกรคอมพิวเตอร์ ปัญหาที่พบเจอส่วนใหญ่จะเกี่ยวกับบิต (bits) ที่ติดหรือดับ ซึ่งเป็นปริมาณที่\emph{ไม่ต่อเนื่อง} วิชา\emph{วิยุตคณิต} (discrete mathematics) พูดถึงสิ่งที่ไม่ต่อเนื่อง ที่นับแยกเป็นชิ้นเป็นอันได้

(ก่อนที่จะไปต่อ จะเห็นว่าศัพท์บัญญัติภาษาไทยบางตัว อ่านแล้วก็ไม่รู้เรื่องว่าแปลว่าอะไรอยู่ดี ดังนั้น จะใช้ทับศัพท์ภาษาอังกฤษในกรณีเหล่านี้)

แล้ววิชานี้สำคัญจริงหรือ

คราวนี้จะขอมองไปข้างหน้าบ้าง ว่าวิศวกรคอมพิวเตอร์จบแล้วจะต้องทำอะไร

วิศวกรคอมพิวเตอร์ ย่อมต้องแก้ปัญหาเชิงคอมพิวเตอร์ แต่เมื่อแก้แล้ว จะมั่นใจได้อย่างไรว่าวิธีแก้ปัญหานั้นถูกต้อง

วิธีหนึ่งที่ทำได้ก็คือ ใช้คณิตศาสตร์มาอธิบายและให้เหตุผล ตัวอย่างเช่น
\begin{itemize}
\item หากแก้ปัญหาที่เกี่ยวข้องกับการนับ เมื่อนับแล้วต้องการเรียกสิ่งของที่นับมาให้เป็นกลุ่มเป็นก้อน ก็สามารถใช้เซต (set) มาช่วยอธิบายได้

\item หากจะนับจำนวนว่าสิ่งของที่ว่ามีเท่าไร ก็สามารถใช้จำนวนนับ $\mathbb{N}=\set{0, 1, 2, \dots}$

\item เมื่อเราเขียนโปรแกรมคอมพิวเตอร์แล้ว จะมั่นใจได้อย่างไรว่าโปรแกรมทำงานถูกต้องตามที่ต้องการหรือไม่ เราสามารถใช้ตรรกศาสตร์ทำให้เรามั่นใจได้

\item เมื่อเราเขียนโปรแกรมอ่านค่าและแสดงผล ก็เสมือนเราเขียนฟังก์ชันขึ้นมาฟังก์ชันหนึ่ง

\item หากเราต้องการพูดถึงของสองสิ่งที่เกี่ยวข้องกัน ก็สามารถใช้ความสัมพันธ์เข้ามาอธิบาย

\item วิธีหนึ่งในการเขียนแจกแจงความสัมพันธ์ คือการวาดแผนภาพความสัมพันธ์ ซึ่งเป็นกรณีพิเศษของวิชาทฤษฎีกราฟ

\item เวลาเราซื้อสินค้าผ่านอินเทอร์เน็ต เราต้องบอกที่อยู่และเลขบัตรเครดิต ซึ่งถือว่าเป็นข้อมูลส่วนบุคคล ไม่ควรเผยแพร่ให้ผู้ที่ไม่เกี่ยวข้องทราบ แต่ผู้คนในอินเทอร์เน็ตนั้นมีมากมาย จะทำอย่างไรจึงจะสื่อสารความลับในที่สาธารณะได้ ทฤษฎีจำนวนเป็นวิธีหนึ่งที่นำมาใช้แก้ปัญหานี้ได้

\item หากเราต้องการทราบว่าสายอักขระ (string) ที่รับมานั้นมีรูปแบบตรงตามที่ต้องการหรือไม่ เราสามารถใช้นิพจน์ปรกติ (regular expression) ซึ่งเป็นส่วนหนึ่งของทฤษฎีเครื่องสถานะจำกัด (automata theory) ช่วยสร้างตัวตรวจสอบได้

\item หากสร้างอุปกรณ์คอมพิวเตอร์ขึ้นมาชิ้นหนึ่ง เราจะมีวิธีรับประกันลูกค้าอย่างไรว่า อุปกรณ์ชิ้นนี้จะไม่เสื่อมสภาพหรือชำรุดเสียหายก่อนเวลาอันสมควร ก็ต้องใช้ความน่าจะเป็นมาช่วยคำนวณ
\end{itemize}
จะเห็นว่า ปัญหาเชิงคอมพิวเตอร์เกือบทุกปัญหาสามารถนำคณิตศาสตร์เข้ามาอธิบายและแก้ไขได้ โดยหัวข้อต่างๆ ในตัวอย่างข้างต้น คือสิ่งที่เราจะได้ศึกษากันในวิชานี้

\section{Why proofs?}

แล้วเราจะมั่นใจได้อย่างไรว่าวิธีที่เราคิดค้นขึ้นมานั้นถูกต้องแน่ๆ

ย่อมต้องพิสูจน์

แล้วการพิสูจน์คืออะไร

\begin{definition}
\emph{การพิสูจน์} (proof) คือวิธีการตรวจสอบให้มั่นใจว่าสิ่งที่คิดไว้เป็นจริง (หรือไม่จริง อย่างใดอย่างหนึ่ง)
\end{definition}

การพิสูจน์ไม่จำเป็นจะต้องเกิดขึ้นในวิชาคณิตศาสตร์เท่านั้น แต่แท้จริงแล้วเกิดขึ้นในชีวิตประจำวัน ตัวอย่างเช่น
\begin{itemize}
\item การสังเกตและทดลอง ในวิชาวิทยาศาสตร์ เพื่อตรวจสอบว่าสมมุติฐานที่ตั้งมานั้นเป็นจริง (หรือไม่จริง อย่างใดอย่างหนึ่ง)

\item การสุ่มตัวอย่างหรือหาตัวอย่างค้าน หากสุ่มตัวอย่างมาหลายๆ ตัวอย่างแล้วได้ผลลัพธ์เช่นเดิม ก็ค่อนข้างมั่นใจได้ว่าสิ่งที่คิดไว้เป็นจริง แต่ถ้าหาตัวอย่างค้านมาได้ ก็เป็นวิธีแย้งว่าสิ่งที่คิดไว้เป็นจริงไม่ได้

\item กระบวนการพิพากษาคดีความ ที่นำพยานหลักฐานต่างๆ มามัดตัวผู้ต้องสงสัย หรือมาค้านว่าผู้ต้องสงสัยนั้นบริสุทธิ์
\end{itemize}
แต่การพิสูจน์ในวิชาคณิตศาสตร์ต้องเป็บแบบแผนที่ชัดเจนมากกว่าวิธีการพิสูจน์โดยทั่วไป
\begin{definition}
\emph{การพิสูจน์ทางคณิตศาสตร์} (mathematical proof) คือการตรวจสอบความเป็นจริงของ\emph{ประ\-พจน์} (proposition) ตามขั้นตอนวิธี\emph{นิรนัยเชิงตรรกะ} (logical deductions) จากเซตของ\emph{สัจพจน์} (axioms)
\end{definition}
ก่อนจะเข้าใจได้ว่าการพิสูจน์ทางคณิตศาสตร์คืออะไร เราต้องเข้าใจความหมายของศัพท์ต่างๆ ในนิยามข้างต้นเสียก่อน
\begin{definition}
\emph{ประพจน์} (proposition) คือประโยคที่ไม่จริงก็เท็จ
\end{definition}
แต่เราอาจจะไม่สามารถฟันธงไปทางใดทางหนึ่งได้ว่า ประพจน์ที่กำลังพิจารณาอยู่นั้นจริงหรือเท็จ
\begin{example}
$2+3=5$
\end{example}
\begin{example}
$\forall n\in\mathbb{N}, n^2+n+41$ เป็นจำนวนเฉพาะ
\end{example}
หากลองแทนค่าด้วย $n=0,1,\dots,39$ จะพบว่าเป็นจริง แต่เมื่อ $n=40$ จะได้ว่า $n^2+n+41=1681=41^2$ ซึ่งไม่เป็นจำนวนเฉพาะ ดังนั้น ประพจน์นี้เป็นเท็จ

แม้ว่าการทดลองซ้ำๆ แล้วได้ผลลัพธ์เดียวกันจะเป็นวิธีพิสูจน์ที่ยอมรับได้ในเชิงวิทยาศาสตร์ แต่ก็ไม่ถือว่าเป็นการพิสูจน์ทางคณิตศาสตร์ เพราะไม่สามารถรับประกันได้ว่าเราได้พิจารณาครบทั้งหมดทุกกรณีที่เป็นไปได้

\begin{theorem}[4-color theorem]
หากต้องการลงสีแผนที่ภูมิศาสตร์ โดยให้ดินแดนที่ติดกันมีสีต่างกัน จะสามารถใช้สีไม่เกิน 4 สีได้
\end{theorem}

\begin{conjecture}[Goldbach, 1742]
จำนวนคู่ทุกจำนวนที่มากกว่า 2 สามารถเขียนเป็นผลรวมของจำนวนเฉพาะ 2 ตัวได้
\end{conjecture}
แม้ว่าจะผ่านมากว่า 278 ปีแล้ว แต่ก็ยังไม่มีใครบอกได้ว่าข้อความคาดการณ์นี้เป็นจริงหรือไม่ แต่ขณะนี้ทราบแล้วว่า conjecture นี้จริงกับทุกจำนวนที่ไม่เกิน $10^{18}$ แต่จริงจริงๆ หรือไม่ ยังต้องพิสูจน์

\begin{example}
$\forall n\in\mathbb{Z}: n\geq 2\implies n^2\geq 4$
\end{example}

เราได้เห็นตัวอย่างประพจน์มาหลายตัวอย่างแล้ว แต่ประโยคที่ไม่ใช่ประพจน์ก็มี เช่น ประโยคคำถาม และประโยคขอร้อง

หลังจากที่เราได้รู้จักประพจน์แล้ว เราสามารถใช้การอนุมานทางตรรกะ เพื่อพิจารณาความสมเหตุสมผลของข้อมูลที่มีอยู่

\begin{definition}
\emph{การอนุมานทางตรรกะ} (logical inference) เป็นการใช้วิธีการทางตรรกศาสตร์เพื่อพิจารณาว่าสิ่งใดบ้างที่สามารถสรุปได้จากข้อมูลที่มี และสิ่งใดที่สรุปไม่ได้
\end{definition}
%
\begin{example}\label{ex:baseball-reasoning}
หากทราบว่า
\begin{enumerate}[ref=(\arabic*)]
\item\label{ex:read-pass} ถ้าฉันอ่านหนังสือ ฉันจะผ่านวิชา 261216
\item\label{ex:noplay-read} ถ้าฉันไม่เล่นเบสบอล ฉันจะอ่านหนังสือ
\item\label{ex:nopass} ฉันตกวิชา 261216
\end{enumerate}
แล้วถ้าเราจะสรุปว่า ดังนั้น ฉันเล่นเบสบอล จะสมเหตุสมผลหรือไม่
\begin{pf}
ข้อสรุปดังกล่าวสมเหตุสมผล เนื่องจากฉันตกวิชา 261216~\ref{ex:nopass} จึงเป็นไปไม่ได้ที่ฉันจะอ่านหนังสือ เพราะถ้าฉันอ่านฉันจะผ่าน~\ref{ex:read-pass} \enskip เนื่องจากฉันไม่อ่านหนังสือ จึงเป็นไปไม่ได้ที่ฉันจะไม่เล่นเบสบอล เพราะถ้าฉันไม่เล่น ฉันจะอ่าน~\ref{ex:noplay-read} \enskip ดังนั้น สรุปได้ว่า ฉันเล่นเบสบอล
\end{pf}
\end{example}

องค์ประกอบสุดท้ายของการพิสูจน์ทางคณิตศาสตร์ คือสมมุติฐานที่เราต้องใช้ในการพิสูจน์ที่เรียกว่าสัจพจน์

\begin{definition}
\emph{สัจพจน์} (axiom) คือประพจน์ที่สมมุติว่าเป็นจริงเสมอ
\end{definition}
การพิสูจน์ทางคณิตศาสตร์ทุกครั้ง เราต้องตั้งสมมุติฐานบางอย่าง มิฉะนั้นเราจะพิสูจน์อะไรไม่ได้เลย \enskip Axioms เป็นสมมุติฐานที่เราใช้ในการพิสูจน์ ที่เราไม่ได้ตั้งขึ้นเอง แต่ได้รับการยอมรับโดยทั่วไป

\begin{example}
ถ้า $a=b$ และ $b=c$ แล้ว $a=c$ (สมบัติการถ่ายทอด)
\end{example}
สังเกตว่า เราสามารถเปลี่ยนเครื่องหมาย $=$ เป็นเครื่องหมาย $<,\leq,>,\geq$ ได้ ซึ่งก็ยังเป็นสัจพจน์ที่เราสามารถใช้ในการพิสูจน์ต่างๆ ได้อยู่ \enskip อย่างไรก็ดี หากเปลี่ยนเป็นเครื่องหมาย $\neq$ จะเห็นว่าประพจน์  ``ถ้า $a\neq b$ และ $b\neq c$ แล้ว $a\neq c$'' นั้นไม่เป็นจริง เนื่องจากเราสามารถหาตัวอย่างค้านโดยที่ $a=0, b=1, c=0$ ได้ \enskip แต่ถ้าเราไม่ระมัดระวัง แล้วกำหนดให้ประพจน์ดังกล่าวเป็นสัจพจน์เพื่อใช้ในการพิสูจน์ต่างๆ จะเห็นว่า เราสามารถพิสูจน์หลายๆ สิ่งที่ไม่จริงขึ้นมาได้ \enskip ดังนั้น การเพิ่มประพจน์ใดๆ เข้าไปในเซตของสัจพจน์ จะต้องกระทำด้วยความระมัดระวัง เพื่อป้องกันไม่ให้เกิด\emph{ความไม่ต้องกัน} (inconsistency) ขึ้น อันจะส่งผลให้เราเขียนบทพิสูจน์ที่ไม่เป็นเหตุเป็นผลได้ \enskip ในท้ายที่สุด หากเรายอมรับเซตของสัจพจน์ใดๆ ว่าเป็นจริง ก็ต้องยอมรับบทพิสูจน์ที่ได้จากการให้เหตุผลบนพื้นฐานของเซตของสัจพจน์นั้นๆ ด้วย


\section{Terminology}

ในวิชานี้ เราจะได้พบคำศัพท์ต่างๆ ที่มีความหมายหรือลักษณะการใช้งานที่คล้ายคลึงกัน จึงขอแจกแจงความแตกต่าง ณ ที่นี้
\begin{itemize}
    \item Definition (นิยาม) เป็นคำจำกัดความเพื่ออำนวยความสะดวกในการใช้งาน \enskip ตัวอย่างเช่น เราอาจจะนิยามคำว่า ``การหารลงตัว'' ระหว่างจำนวนเต็มบวกสองตัว ว่าตัวหารจะต้องหารตัวตั้งได้พอดี ไม่เหลือเศษเลย \enskip ในกรณีนี้ หากมีการอ้างถึงคำว่า ``การหารลงตัว'' ในภายหลัง ทุกฝ่ายจะเข้าใจความหมายตรงกัน \enskip อย่างไรก็ดี นิยามสำหรับคำใดคำหนึ่งอาจจะแตกต่างกันไปในแต่ละบริบทก็ได้ ดังนั้น ควรตรวจสอบให้มั่นใจว่าเราเข้าใจความหมายของนิยามนั้นๆ อย่างถูกต้องตามบริบทแล้ว \enskip หากไม่แน่ใจ ให้ย้อนกลับไปตรวจสอบคำจำกัดความที่ได้เขียนไว้ หรือสอบถามผู้เขียนเพิ่มเติมหากยังไม่ชัดเจน
    \item Axiom (สัจพจน์) เป็นประพจน์ที่เราสมมุติไว้ว่าจะเป็นจริงเสมอ โดยไม่จำเป็นต้องทำการพิสูจน์ \enskip ทั้งนี้ ในบริบทที่ต่างกัน เซตของสัจพจน์ที่เราสมมุติว่าเป็นจริงอาจจะแตกต่างกันไปด้วย \enskip หากไม่แน่ใจว่าสิ่งใดเป็นสัจพจน์หรือไม่ในบริบทหนึ่งๆ ให้ถามให้มั่นใจเสียก่อน
    \item Conjecture (บทคาดเดา) เป็นประพจน์ที่ได้รับการคาดเดาว่าน่าจะเป็นจริง แต่ยังหาบทพิสูจน์ไม่ได้
    \item Theorem (ทฤษฎีบท) เป็นประพจน์ที่พิสูจน์แล้วว่าเป็นจริง และมีความสำคัญระดับหนึ่ง ซึ่งสามารถนำไปใช้ต่อ\-ยอดในการพิสูจน์ประพจน์อื่นๆ ได้
    \item Lemma (บทตั้ง) เป็นประพจน์ที่สามารถพิสูจน์ได้ว่าเป็นจริง แต่มีความสำคัญไม่มากนัก มักจะใช้เป็นส่วนย่อยๆ ในบทพิสูจน์ของทฤษฎีบทที่มีความสำคัญมากกว่า
    \item Corollary (บทแทรก) เป็นประพจน์ที่พิสูจน์ได้ว่าเป็นจริง แต่บทพิสูจน์นั้นเป็นการประยุกต์ใช้ทฤษฎีบทที่เพิ่งได้พิสูจน์มาโดยง่าย \enskip โดยทั่วไปแล้ว บทพิสูจน์ของบทแทรกมักจะมีความยาวไม่เกิน 2--3 บรรทัดเท่านั้น
\end{itemize}