\section{Logical operators}

\emph{ตัวดำเนินการทางตรรกะ} (logical operators) มีไว้เพื่อใช้สร้างประพจน์ที่มีความซับซ้อนมากขึ้นจากประพจน์ย่อยๆ ตัวดำเนินการเหล่านี้ ส่วนใหญ่แล้วมีใช้ทั่วไปในชีวิตประจำวันนอกเหนือวิชาคณิตศาสตร์ ตัวอย่างเช่น ในภาษาไทย คำสันธานหลายๆ คำใช้สร้างประโยคที่มีความซับซ้อนมากขึ้น เทียบเท่ากับ conjunctions ในภาษาอังกฤษ

เริ่มด้วยตัวดำเนินการ หรือคำสันธานที่นิยมใช้กันอย่างแพร่หลาย
\begin{definition}[and]
    $P\wedge Q$ ``$P$ และ $Q$'' เป็นจริง แปลว่า $P$ เป็นจริง และ $Q$ เป็นจริง กล่าวคือ ประพจน์ผลลัพธ์จะเป็นจริงก็ต่อเมื่อประพจน์ย่อยทั้งคู่เป็นจริง
\end{definition}
\begin{definition}[or]
    $P\vee Q$ ``$P$ หรือ $Q$'' เป็นจริง แปลว่า $P$ เป็นจริง หรือ $Q$ เป็นจริง กล่าวคือ ประพจน์ผลลัพธ์จะเป็นจริงก็ต่อเมื่อประพจน์ย่อยสักตัวเป็นจริง
\end{definition}
จากนิยามของตัวดำเนินการดังกล่าว เราสามารถเขียน\emph{ตารางค่าความจริง} (truth table) อธิบายได้
\[
\begin{array}{c|c|c|c}
P & Q & P\wedge Q & P\vee Q \\\hline
\true & \true & \true & \true \\
\true & \false & \false & \true \\
\false & \true & \false & \true \\
\false & \false & \false & \false
\end{array}
\]

\begin{example}
2 เป็นเลขคู่ หรือ 2 เป็นเลขคี่
\end{example}

\begin{example}
ฉันมันไม่ดี หรือเธอไปมีรักใหม่

หากอ่านเป็นประโยคคำถาม ตัวอย่างนี้ไม่ใช่ประพจน์ แต่ถ้าอ่านเป็นประโยคบอกเล่า ประพจน์นี้เป็นจริงถ้าฉันมันไม่ดีเป็นจริง หรือเธอไปมีรักใหม่เป็นจริง หรือทั้งคู่เป็นจริง
\end{example}

คำสันธานอีกคำหนึ่งที่นิยมใช้กัน คือคำว่า ``แต่''
\begin{example}
เธอลืมได้ แต่ฉันยัง

ผู้พูดกำลังสื่อถึงความจริงสองประการ กล่าวคือ เธอลืมได้ และ ฉันยังลืมไม่ได้ \enskip เนื่องจากประพจน์ย่อยทั้งสองตัวนี้เป็นจริงทั้งคู่ จะได้ว่า คำว่า ``แต่'' มีความหมายในเชิงคณิตศาสตร์เช่นเดียวกับคำว่า ``และ''
\end{example}

ตัวดำเนินการถัดไป มักจะใช้บ่งความเป็นเหตุเป็นผล หรือลำดับเวลาของเหตุการณ์  แต่ในทางคณิตศาสตร์นั้น ค่าความจริงของประพจน์มักจะไม่ขึ้นกับเวลา จึงควรระมัดระวังในจุดนี้
\begin{definition}[implies]
$P\implies Q$ ``$P$ ส่อถึง $Q$'' หรือ ``ถ้า $P$ แล้ว $Q$'' เป็นจริง แปลว่า
\begin{itemize}
\item ถ้า $P$ เป็นจริง $Q$ ต้องเป็นจริง
\item ถ้า $P$ เป็นเท็จ $Q$ จะเป็นอะไรก็ได้
\end{itemize}
กล่าวคือ ประพจน์ผลลัพธ์จะเป็นจริงก็ต่อเมื่อประพจน์หน้าเป็นเท็จ หรือประพจน์หน้าเป็นจริงและประพจน์หลังเป็นจริง
\end{definition}

\begin{example}
ถ้าคุณรับรักผม ผมจะขอแต่งงานกับคุณ
\end{example}
กรณีที่น่าสนใจคือ คุณไม่รับรักผม แต่ผมขอแต่งงานกับคุณ ในทางคณิตศาสตร์ กรณีนี้ถือว่าไม่ผิดเงื่อนไข เพราะเราไม่ได้สัญญาว่า หากคุณไม่รับรักผม ผมจะทำอย่างไร

\begin{example}
อะไรก็เกิดขึ้นได้ ถ้ามีปาปริก้า
\end{example}
หากต้องการพิจารณาค่าความจริงของประพจน์นี้ กรณีที่เราต้องสนใจคือ ถ้ามีปาปริก้า แล้วทุกอย่างต้องเกิดขึ้นได้ ซึ่งค่าความจริงนี้อาจจะไม่สามารถบ่งชี้ชัดเจนได้ เนื่องจากสิ่งที่ยังไม่เกิดในขณะนี้ อาจจะเกิดในภายหลังก็ได้ และประพจน์ ``ทุกอย่างต้องเกิดขึ้นได้'' ไม่ได้ตั้งเงื่อนไขเวลาไว้

\begin{example}
ถ้า $0=1$ ผมจะให้ F ทุกคน
\end{example}
ประพจน์นี้เป็นจริง เพราะ $0\neq 1$ ดังนั้น ``ผมจะให้ F ทุกคน'' จะจริงหรือไม่จริงก็ได้ ประพจน์ทำนองนี้\emph{เป็นจริงโดยปริยาย} (vacuously true)

\begin{definition}[if and only if]
$P\iff Q$ ``$P$ ก็ต่อเมื่อ $Q$'' เป็นจริง แปลว่า
\begin{itemize}
    \item ถ้า $P$ เป็นจริง $Q$ ต้องเป็นจริง และ
    \item ถ้า $Q$ เป็นจริง $P$ ต้องเป็๋นจริง
\end{itemize}
กล่าวคือ ประพจน์ผลลัพธ์จะเป็นจริงก็ต่อเมื่อ implications ทั้งสองทิศทางของทั้งสองประพจน์ย่อยเป็นจริงทั้งคู่ นั่นคือ $P\iff Q$ เป็นจริง หาก $(P\implies Q)\wedge(Q\implies P)$ เป็นจริง
\end{definition}

ตัวดำเนินการ $\implies$ และ $\iff$ สามารถเขียนเป็นตารางค่าความจริงได้เช่นกัน
\[
\begin{array}{c|c|c|c|c}
P & Q & P\implies Q & Q\implies P & P\iff Q \\\hline
\true & \true & \true & \true & \true \\
\true & \false & \false & \true & \false \\
\false & \true & \true & \false & \false \\
\false & \false & \true & \true & \true
\end{array}
\]
\begin{example}
$\forall n\in\mathbb{Z}: n\geq 2\iff n^2\geq 4$

ประพจน์นี้เป็นเท็จ เนื่องจาก $n=-3$ ไม่ทำให้ $n^2\geq 4\implies n\geq 2$ นั่นคือ $n^2\geq 4$ แต่ $n\not\geq 2$ \enskip กล่าวคือ แม้ว่าในทิศทางขาไป $n\geq 2\implies n^2\geq 4$ จะเป็นจริงก็ตาม แต่ในทิศทางขากลับ $n^2\geq 4\implies n\geq 2$ นั้นไม่เป็นจริง จึงทำให้ทั้งประพจน์ไม่เป็นจริง
\end{example}

หาก $P\iff Q$ เราสามารถกล่าวได้ว่าประพจน์ทั้งสองนี้\emph{สมมูล} (equivalent) กัน \enskip ในบางครั้ง เราอาจใช้สัญลักษณ์ $\equiv$ แทน $\iff$ ได้

ตัวดำเนินการทางตรรกะที่ใช้กลับค่าความจริง เทียบเท่ากับคำว่า ``ไม่''
\begin{definition}[negation]
    $\neg P$ ``นิเสธของ $P$'' เป็นจริง แปลว่า $P$ เป็นเท็จ ซึ่งหมายความว่าเป็นไปไม่ได้ที่ $P$ จะเป็นจริง กล่าวคือ ถ้า $P$ เป็นจริงแล้ว จะเกิดสิ่งที่เป็นไปไม่ได้ขึ้น \enskip ดังนั้น $\neg P$ จึงสามารถตีความได้อีกแบบหนึ่งเป็น $P\implies\False$
\end{definition}
ตัวดำเนินการ $\neg$ สามารถเขียนเป็นตารางค่าความจริงได้ดังนี้
\[
\begin{array}{c|c}
P & \neg P \\\hline
\true & \false \\
\false & \true \\
\end{array}
\]
หาก $P$ เป็นไปไม่ได้ แสดงว่า ถ้า $P$ เป็นจริงขึ้นมา \False{} ก็จะเป็นจริงด้วย เพราะ \False{} เป็นสิ่งที่เป็นไปไม่ได้
\begin{exercise}
    เปรียบเทียบตารางค่าความจริงของ $\neg P$ และ $P\implies\False$ เพื่อตรวจสอบว่าประพจน์ทั้งสองนี้สมมูลกัน
\end{exercise}

\begin{example}\label{ex:neg-couple}
ประพจน์ที่ว่า แฟนฉันไม่มีใครอื่น สามารถเขียนได้อีกแบบคือ ไม่จริงที่ว่า แฟนฉันไปมีใครอื่น หรือ เป็นไปไม่ได้ที่แฟนฉันไปมีใครอื่น \enskip ดังนั้น ถ้าแฟนฉันไปมีใครอื่น แล้วสิ่งที่เป็นไปไม่ได้จะเกิดขึ้น \enskip แสดงว่า ถ้าแฟนฉันไปมีใครอื่น แล้ว \False{} ก็จะเกิดขึ้น กล่าวคือ \False{} จะเป็นจริง

หากให้ $P$ เป็นประพจน์ ``แฟนฉันไปมีใครอื่น'' จะได้ว่า $P\implies\False$ คือประพจน์ ``แฟนฉันต้องไม่มีใครอื่น'' เพราะถ้ามี สิ่งที่เป็นไปไม่ได้จะเกิดขึ้น
\end{example}

\subsection{Precedence and associativity}
เมื่อใช้ตัวดำเนินการทางตรรกะหลายตัวพร้อมกัน จะต้องมีลำดับการคำนวณค่าความจริงของประพจน์ โดยใช้กฎ\emph{การทำก่อน} (precedence) และกฎ\emph{การเปลี่ยนหมู่} (associativity)

แท้จริงแล้ว สองกฎนี้เป็นที่คุ้นเคยกันดีอยู่แล้วในการคำนวณผลลัพธ์ของนิพจน์เลขคณิต (arithmetic expressions) ที่ใช้ตัวดำเนินการทางเลขคณิต (arithmetic operators) โดย precedence และ associativity ของ arithmetic operators เป็นดังตารางที่~\ref{tab:arith-prec-assoc}
%
\begin{table}
    \centering
    \begin{tabular}{c|c}
        \bf operator & \bf associativity \\\hline
        $(x)$ & --- \\
        ${-x}$ & --- \\
        $x^y$ & right \\
        $x*y \quad x/y$ & left \\
        $x+y \quad x-y$ & left
    \end{tabular}
    \caption[precedence and associativity rules for arithmetic operators]{precedence and associativity rules for arithmetic operators, where operators towards the top of the table have higher precedence}
    \label{tab:arith-prec-assoc}
\end{table}
%
โดยตัวดำเนินการที่อยู่แถวต้นๆ ของตาราง จะมี precedence สูงกว่าตัวดำเนินการที่ตามๆ มา เช่น หากมีทั้งการบวกและการคูณใน expression เดียวกัน จะต้องทำการคูณก่อน \enskip ในขณะเดียวกัน หากใช้ตัวดำเนินการที่มี precedence เท่ากัน ลำดับการคำนวณจะเริ่มจากด้านที่ปรากฏในตาราง เช่น จะต้องทำการลบก่อนบวกใน $2-3+4$ ทำให้ได้ผลลัพธ์เป็น 3 และทำการยกกำลังจากบนลงล่างใน $2^{3^2}$ ทำให้ได้ผลลัพธ์เป็น 512 \enskip ทั้งนี้ หากต้องการเปลี่ยนแปลงลำดับการคำนวณ ให้ใช้วงเล็บ เช่น $2-(3+4)=-5$ และ $(2^3)^2=64$ \enskip อย่างไรก็ดี เราไม่จำเป็นต้องใช้วงเล็บกับตัวดำเนินการที่มีสมบัติการเปลี่ยนหมู่ เช่น $+$ เนื่องจากไม่ว่าจะคำนวณจากซ้ายไปขวาหรือขวาไปซ้าย ก็จะได้ผลลัพธ์เท่ากันเสมอ เช่น $(2+3)+4=2+3+4=2+(3+4)$

ในทำนองเดียวกัน นิพจน์เชิงตรรกะ (logical expressions) ก็มี precedence และ associativity ดังตารางที่~\ref{tab:bool-prec-assoc}
%
\begin{table}
    \centering
    \begin{tabular}{c|c}
        \bf operator & \bf associativity \\\hline
        $(P)$ & --- \\
        $\neg P$ & --- \\
        $P\wedge Q$ & left \\
        $P\vee Q$ & left \\
        $P\implies Q$ & right \\
        $P\iff Q$ & ---
    \end{tabular}
    \caption[precedence and associativity rules for logical operators]{precedence and associativity rules for logical operators, where operators towards the top of the table have higher precedence}
    \label{tab:bool-prec-assoc}
\end{table}

\begin{example}
    $P_1\implies P_2\vee P_3\wedge P_4\vee P_5\implies P_6\iff P_7\implies P_8$ มีลำดับการคำนวณเช่นเดียวกับ $(P_1\implies ((P_2\vee (P_3\wedge P_4)\vee P_5)\implies P_6))\iff (P_7\implies P_8)$ 
\end{example}

\begin{exercise}
    กรณีใดบ้างของ $P$, $Q$, และ $R$ ที่ทำให้ $P\implies(Q\implies R)$ และ $(P\implies Q)\implies R$ มีค่าความจริงที่แตกต่างกัน
\end{exercise}

\subsection{Negating logical operators}
การใช้นิเสธกับตัวดำเนินการทางตรรกะนั้น วิธีการที่ได้ผลและเข้าใจได้ง่าย คือการเขียนประพจน์ออกมาเป็นภาษาไทย แล้วใช้การตีความเพื่อแปลงรูป

\begin{tabular}{lcl}
$\neg (P\wedge Q)$ & $\equiv$ & ไม่จริงที่ว่า $P$ เป็นจริง และ $Q$ เป็นจริง \\
& $\equiv$ & $P$ ไม่จริง หรือ $Q$ ไม่จริง \\
& $\equiv$ & $\neg P\vee\neg Q$ \\
%
$\neg (P\vee Q)$ & $\equiv$ & ไม่จริงที่ว่า $P$ เป็นจริง หรือ $Q$ เป็นจริง \\
& $\equiv$ & $P$ และ $Q$ เป็นเท็จทั้งคู่ \\
& $\equiv$ & $\neg P\wedge\neg Q$ \\
%
$\neg (P\implies Q)$ & $\equiv$ & ไม่จริงที่ว่า ถ้า $P$ แล้ว $Q$ \\
& $\equiv$ & $P$ เป็นจริง (เกิดขึ้น) แต่ $Q$ ไม่เป็นจริง (ไม่เกิดขึ้น) \\
& $\equiv$ & $P\wedge\neg Q$
\end{tabular}
