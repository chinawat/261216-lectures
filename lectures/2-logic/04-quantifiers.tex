\section{Quantifiers}

\emph{ตัวบ่งปริมาณ} (quantifiers) ใช้ในการเชื่อมประพจน์จำนวนไม่จำกัดที่มีแบบอย่าง (pattern) เดียวกันเข้าด้วยกัน เพื่อบ่งบอกค่าความจริงของ predicates ที่กล่าวถึงกลุ่มของสิ่งของต่างๆ \enskip ในการใช้ quantifier แต่ละครั้งเราจะต้องทราบว่า\emph{โดเมน} (domain) ซึ่งก็คือเซตของสิ่งของที่เราสนใจจะพิจารณาคืออะไร เนื่องจากโดเมนที่ต่างกันอาจจะทำให้ค่าความจริงของประพจน์ที่ใช้ตัวบ่งปริมาณเปลี่ยนแปลงไปได้

ในหัวข้อย่อยต่อไปนี้ ให้ $P(x)$ เป็น predicate ที่มี $x$ เป็น\emph{ตัวแปรอิสระ} (free variable) กล่าวคือ $x$ ยังไม่ถูกผูกมัดด้วยตัวบ่งปริมาณใดๆ \enskip และให้ $D=\set{d_1,d_2,\ldots}$ เป็นโดเมนที่เราสนใจ

\subsection{Universal quantifier}
\emph{ตัวบ่งปริมาณสำหรับทุกตัว} (universal quantifier) เป็นการใช้ตัวดำเนินการ $\wedge$ เชื่อมประพจน์จำนวนไม่จำกัดเข้าด้วยกัน โดยใช้สัญลักษณ์ $\forall$ (for all)

\begin{definition}
$\forall x: P(x)$ จะเป็นจริง ก็ต่อเมื่อ $\bigwedge_{x\in D} P(x)$ เป็นจริง กล่าวคือ $P(d_1)\wedge P(d_2)\wedge\cdots$ เป็นจริง
\end{definition}
$\forall x: P(x)$ จะเป็นจริง ก็ต่อเมื่อ $d_i$ ทุกตัวใน $D$ ทำให้ $P(d_i)$ เป็นจริง

\begin{example}
ให้ $D$ เป็นเซตของประพจน์ \enskip ค่าความจริงของ \[\forall P: \True\wedge P \iff P\] เป็นจริง \enskip ประพจน์ดังกล่าวกล่าวว่า $\textrm{true}$ เป็นเอกลักษณ์ของการ $\wedge$ เนื่องจากนำไป and กับประพจน์ใดๆ (ในที่นี้คือ $P\in D$) ก็ได้ประพจน์นั้นๆ เป็นผลลัพธ์
\end{example}

\begin{example}
หาก $D=\emptyset$ จะได้ว่า $\forall x: P(x)=\bigwedge_{x\in\emptyset} P(x)$ นั่นคือ เป็นการนำประพจน์ 0 ตัวมา and ($\wedge$) กัน ผลลัพธ์ที่ได้คือ $\textrm{true}$ เนื่องจาก $\textrm{true}$ เป็นเอกลักษณ์ของการ and
\end{example}

\paragraph{Inference rules for universal quantification}
กฎการอนุมานที่เกี่ยวกับตัวบ่งปริมาณ $\forall$ มีดังนี้
\[
\inferrule*[Right=$\forall$-intro]
{\Gamma,d\in D\vdash P(d)}
{\Gamma\vdash\forall x: P(x)}
\]
Inference rule นี้ (มีอีกชื่อว่า \emph{universal generalization}) กล่าวว่า หากต้องการพิสูจน์ว่า $\forall x: P(x)$ เป็นจริง ให้สมมุติว่า $d$ เป็นสมาชิก\emph{ใดๆ} ของโดเมน แล้วพิสูจน์ว่า $P(d)$ เป็นจริง \enskip ในทางกลับกัน หากเราทราบว่า $P(d)$ เป็นจริงสำหรับทุกค่า $d\in D$ แล้วเราสามารถสรุปได้ว่า $\forall x: P(x)$ เป็นจริงด้วย

กฎนี้สามารถตีความได้คล้ายกับ \textsc{$\wedge$-intro} ที่กล่าวว่า หากจะพิสูจน์ $P\wedge Q$ ต้องพิสูจน์ว่าทั้ง $P$ และ $Q$ เป็นจริงทั้งคู่ \enskip ในที่นี้ หากจะพิสูจน์ $\forall x: P(x)$ เราต้องพิสูจน์ให้ได้ว่า predicate $P(x)$ ต้องเป็นจริงสำหรับ $x$ ทั้งหมดทุกตัวในโดเมน
%
\begin{example}
ให้โดเมนที่เราสนใจเป็นเซตของจำนวนเต็ม \enskip บทพิสูจน์ที่ว่าจำนวนเต็มทุกตัวที่หารด้วย 10 ลงตัวจะต้องเป็นเลขคู่ สามารถเขียนได้ดังนี้
\begin{pf}
ให้ $x$ เป็นจำนวนเต็มใดๆ \enskip ต้องพิสูจน์ว่า ถ้า $x$ หารด้วย 10 ลงตัว แล้ว $x$ จะเป็นเลขคู่ (\textsc{$\forall$-intro}) \enskip สมมุติว่า $x$ หารด้วย 10 ลงตัว ต้องพิสูจน์ว่า $x$ เป็นเลขคู่ (\textsc{$\Rightarrow$-intro}) \enskip เนื่องจาก $x$ หารด้วย 10 ลงตัว จะได้ว่ามีจำนวนเต็ม $y$ ที่ทำให้ $x=10y$ แต่นั่นแปลว่า $x=2(5y)$ กล่าวคือ $x$ ต้องเป็นเลขคู่
\end{pf}
\end{example}

\[
\inferrule*[Right=$\forall$-elim]
{\Gamma,\forall x: P(x),P(d_i)\vdash Q}
{\Gamma,\forall x: P(x)\vdash Q}
\]
Inference rule นี้ (มีอีกชื่อว่า \emph{universal specification}) กล่าวว่า หากทราบว่า $\forall x: P(x)$ เป็นจริง ก็สามารถเลือกแทนค่า $x$ ด้วยสมาชิก $d_i$ \emph{สักตัว}ในโดเมน แล้วเพิ่ม $P(d_i)$ เข้าไปในสมมุติฐานได้ \enskip นอกจากนี้ สังเกตว่า สมมุติฐาน $\forall x: P(x)$ ยังคงอยู่ เนื่องจากเราอาจจะจำเป็นต้องแทนค่า $x$ ด้วยสมาชิกตัวอื่นๆ ในโดเมนภายหลัง

กฎนี้สามารถตีความได้คล้ายกับ \textsc{$\wedge$-elim} ที่กล่าวว่า หากเราทราบว่า $P\wedge Q$ เป็นจริง เราก็ทราบว่า $P$ เป็นจริง และ $Q$ เป็นจริง \enskip ในที่นี้ เมื่อเราทราบแล้วว่า $\forall x: P(x)$ เป็นจริง ไม่ว่าจะแทนค่า $x$ ด้วยสมาชิกใดๆ ในโดเมนลงใน $P(x)$ ผลลัพธ์ที่ได้ก็ย่อมเป็นจริงอยู่ดี
%
\begin{example}
ให้โดเมนที่เราสนใจเป็นเซตของนักศึกษามหาวิทยาลัยเชียงใหม่ \enskip บทพิสูจน์ที่ว่า ถ้านักศึกษาทุกคนเดินขึ้นดอย แล้วน้องกอล์ฟซึ่งเป็นนักศึกษาจะต้องเดินขึ้นดอยด้วย สามารถเขียนได้ดังนี้
\begin{pf}
สมมุติว่านักศึกษาทุกคนเดินขึ้นดอย ต้องพิสูจน์ว่า น้องกอล์ฟซึ่งเป็นนักศึกษาจะต้องเดินขึ้นดอยเช่นกัน (\textsc{$\Rightarrow$-intro}) เนื่องจากนักศึกษาทุกคนเดินขึ้นดอย และน้องกอล์ฟเป็นนักศึกษา (กล่าวคือ น้องกอล์ฟเป็นสมาชิกในโดเมน) จะได้ว่า น้องกอล์ฟเดินขึ้นดอย (\textsc{$\forall$-elim})
\end{pf}
\end{example}

\subsection{Existential quantifier}
\emph{ตัวบ่งปริมาณสำหรับตัวมีจริง} (existential quantifier) เป็นการใช้ตัวดำเนินการ $\vee$ เชื่อมประพจน์จำนวนไม่จำกัดเข้าด้วยกัน โดยใช้สัญลักษณ์ $\exists$ (there exists)

\begin{definition}
$\exists x: P(x)$ จะเป็นจริง ก็ต่อเมื่อ $\bigvee_{x\in D} P(x)$ เป็นจริง กล่าวคือ $P(d_1)\vee P(d_2)\vee\cdots$ เป็นจริง
\end{definition}
$\exists x: P(x)$ จะเป็นจริง ก็ต่อเมื่อมี $d_i$ สักตัวใน $D$ ทำให้ $P(d_i)$ เป็นจริง

\begin{example}
หาก $D=\emptyset$ จะได้ว่า $\exists x: P(x)=\bigvee_{x\in\emptyset} P(x)$ นั่นคือ เป็นการนำประพจน์ 0 ตัวมา or ($\vee$) กัน ผลลัพธ์ที่ได้คือ $\textrm{false}$ เนื่องจาก $\textrm{false}$ เป็นเอกลักษณ์ของการ or
\end{example}

วิธีการพิสูจน์ $\exists$ ให้หาหลักฐานในโดเมนที่ทำให้ predicate เป็นจริง
%
\begin{example}
ให้โดเมนที่เราสนใจเป็นเซตของจำนวนเต็ม \enskip บทพิสูจน์ที่ว่า มีจำนวนเต็มที่เป็นเอกลักษณ์ของการบวก สามารถเขียนได้ดังนี้
\begin{pf}
0 คือจำนวนเต็มที่เป็นเอกลักษณ์ของการบวก
\end{pf}
\end{example}

\paragraph{Inference rules for existential quantification}
กฎการอนุมานบางส่วนที่เกี่ยวกับตัวบ่งปริมาณ $\exists$ มีดังนี้
\[
\inferrule*[Right=$\exists$-intro]
{\Gamma\vdash P(d_i)}
{\Gamma\vdash\exists x: P(x)}
\]
Inference rule นี้กล่าวว่า หากพิสูจน์ได้ว่า $P(d_i)$ เป็นจริง โดยที่ $d_i$ เป็นสมาชิก\emph{สักตัว}ในโดเมน ก็สามารถสรุปได้ว่า $\exists x: P(x)$ เป็นจริง

กฎนี้สามารถตีความได้คล้ายกับ \textsc{$\vee$-intro} ที่กล่าวว่า หากจะพิสูจน์ $P\vee Q$ ให้พิสูจน์ $P$ หรือ $Q$ อย่างใดอย่างหนึ่งก็พอ \enskip ในที่นี้ หากจะพิสูจน์ $\exists x: P(x)$ ให้พิสูจน์ว่ามี $x$ ตัวใดตัวหนึ่งในโดเมนที่ทำให้ $P(x)$ เป็นจริงก็เพียงพอ

\begin{example}
$\exists x\in\mathbb{Z}: \textrm{$x$ เป็นเลขคู่ และ $x$ เป็นจำนวนเฉพาะ}$
\begin{pf}
2 เป็นเลขคู่ และเป็นจำนวนเฉพาะ
\end{pf}
\end{example}

\begin{example}
$\exists x\in\mathbb{N}: 2^x=1$
\begin{pf}
0 ทำให้ $2^0=1$
\end{pf}
\end{example}

\subsection{Uniqueness quantification}
\emph{ตัวบ่งปริมาณสำหรับตัวมีจริงเพียงตัวเดียว} (uniqueness quantifier) เป็นการใช้ตัวบ่งปริมาณ $\exists!$ (there exists unique) เพื่อระบุว่า มีสมาชิกเพียงตัวเดียวเท่านั้นในโดเมนที่ทำให้ predicate เป็นจริง

\begin{definition}
$\exists! x: P(x)$ จะเป็นจริง ก็ต่อเมื่อ $\exists x: \left[P(x)\wedge \forall y: (P(y)\implies x=y)\right]$ เป็นจริง
\end{definition}
$\exists! x: P(x)$ จะเป็นจริง ก็ต่อเมื่อมีสมาชิก $d_i$ เพียงตัวเดียวใน $D$ ทำให้ $P(d_i)$ เป็นจริง กล่าวคือ หากมีสมาชิก $y$ ใดๆ ในโดเมนที่ทำให้ $P(y)$ เป็นจริงๆ แล้วสมาชิก $y$ นี้ต้องเป็นตัวเดียวกับ $d_i$

\subsection{Nested quantifiers}
ลำดับของตัวบ่งปริมาณนั้นมีความสำคัญ เพราะอาจเปลี่ยนแปลงความหมายของประพจน์ได้
\begin{example}
$\forall x\in\mathbb{Q}\exists y\in\mathbb{Q}: x\neq 0\implies xy=1$

ประพจน์นี้เป็นจริง เนื่องจากแต่ละค่า $x\neq 0$ มีค่า $y$ ซึ่งก็คือ $1/x$ ที่ทำให้ $xy=1$
\end{example}

\begin{example}
$\exists y\in\mathbb{Q}\forall x\in\mathbb{Q}: x\neq 0\implies xy=1$

ประพจน์นี้เป็นเท็จ เนื่องจากไม่สามารถหาค่า $y$ ที่นำไปคูณกับทุกค่า $x\neq 0$ แล้วได้ $1$
\end{example}

หากเขียนการใช้ตัวบ่งปริมาณเป็นประพจน์ในภาษาพูด จะเข้าใจได้ง่ายขึ้น \enskip ในตัวอย่างคู่ถัดไปนี้ ให้โดเมนที่เราสนใจเป็นนักศึกษามหาวิทยาลัยเชียงใหม่
\begin{example}
ทุกคนมีคนที่คู่ควร

ในกรณีนี้ หากแต่ละคนสามารถหาคนที่คู่ควรกับตนเองได้ ประพจน์นี้จะเป็นจริง

หากเขียนประพจน์นี้โดยใช้สัญลักษณ์ จะเริ่มด้วย $\forall x\exists y$

หากเขียนประพจน์นี้อีกแบบหนึ่ง จะได้ว่า ใครๆ ก็มีคนที่คู่ควร
\end{example}
เปรียบเทียบตัวอย่างที่แล้วกับตัวอย่างถัดไป
\begin{example}
มีคนที่ทุกคนคู่ควร

ในกรณีนี้ เราต้องหาคนเพียงคนเดียวที่คู่ควรกับทุกๆ คนในโดเมนให้ได้ ประพจน์นี้จึงจะเป็นจริง

หากเขียนประพจน์นี้โดยใช้สัญลักษณ์ จะเริ่มด้วย $\exists x\forall y$

หากเขียนประพจน์นี้อีกแบบหนึ่ง จะได้ว่า มีคนที่ใครๆ ก็คู่ควร
\end{example}

\subsection{Scope of a quantifier}
หากไม่มีการใช้วงเล็บ ขอบเขตของ quantifier แต่ละตัวจะครอบคลุมสัญลักษณ์ทุกตัวที่อยู่ทางด้านขวาของ quantifier นั้นๆ \enskip ดังนั้น หากต้องการจำกัดขอบเขตการใช้งานของ quantifiers ให้ใส่วงเล็บคร่อม

\subsection{Negating quantifiers}
การใช้นิเสธกับตัวบ่งปริมาณนั้น วิธีการที่ได้ผลและเข้าใจได้ง่าย คือการเขียนประพจน์ออกมาเป็นภาษาไทย แล้วใช้การตีความเพื่อแปลงรูป เช่นเดียวกับการใช้นิเสธกับตัวดำเนินการทางตรรกะที่เราได้เห็นมาแล้วก่อนหน้านี้

\begin{tabular}{lcl}
$\neg\forall x: P(x)$ & $\equiv$ & ไม่จริงที่ว่า ทุกค่า $x$ ในโดเมนทำให้ $P(x)$ เป็นจริง \\
& $\equiv$ & มีค่า $x$ ในโดเมนที่ทำให้ $P(x)$ ไม่เป็นจริง \\
& $\equiv$ & $\exists x: \neg P(x)$ \\
%
$\neg\exists x: P(x)$ & $\equiv$ & ไม่จริงที่ว่า มีค่า $x$ ในโดเมนที่ทำให้ $P(x)$ เป็นจริง \\
& $\equiv$ & ทุกค่า $x$ ในโดเมนไม่ทำให้ $P(x)$ เป็นจริง \\
& $\equiv$ & $\forall x: \neg P(x)$
\end{tabular}

\section{Predicate formalization}
ก่อนหน้านี้ เราได้ทำความรู้จักตัวดำเนินการทางตรรกะต่างๆ รวมถึงตัวบ่งปริมาณที่ใช้ในการกล่าวถึงค่าความจริงของกลุ่มของสิ่งของ \enskip ลำดับถัดไป จะยกตัวอย่างการเขียนประพจน์และ predicates ต่างๆ เป็นเชิงรูปนัย

ในตัวอย่างต่อไปนี้ ให้โดเมนที่เราสนใจเป็นเซตของจำนวนตรรกยะ

\begin{example}
$\mathrm{divides}(a,b)$: ``$a$ หาร $b$ ลงตัว'' กล่าวคือ $a$ และ $b$ เป็นจำนวนเต็ม ซึ่งมีจำนวนเต็มที่คูณกับ $a$ แล้วได้ผลลัพธ์เป็น $b$ สามารถเขียนเป็นเชิงรูปนัยได้ดังนี้

เนื่องจากโดเมนที่เราสนใจเป็นจำนวนเต็ม แต่ถ้า $a$ จะหาร $b$ ลงตัวได้ แสดงว่า $a$ และ $b$ ต้องเป็นจำนวนเต็มเท่านั้น เราจึงต้องกำหนดขอบเขตของ $a$ และ $b$ ก่อนว่าเป็นจำนวนเต็ม กล่าวคือ $a\in\mathbb{Z}\wedge b\in\mathbb{Z}$ \enskip กล่าวอีกนัยหนึ่ง หาก $a$ หรือ $b$ ไม่ใช่จำนวนเต็ม ก็ไม่มีทางที่ $a$ จะหาร $b$ ได้ลงตัว

ถัดไป ต้องมีจำนวนเต็มที่นำมาคูณกับ $a$ แล้วได้ $b$ เป็นผลลัพธ์ \enskip ก่อนอื่น เรียกจำนวนเต็มนี้ว่า $c$ \enskip เช่นเดียวกับ $a$ และ $b$ เราต้องกำหนดขอบเขตของ $c$ ว่าเป็นจำนวนเต็ม \enskip จากนั้น $c$ ที่เราต้องการต้องนำไปคูณกับ $a$ แล้วได้ $b$ ด้วย \enskip ทั้งหมดนี้สามารถเขียนได้เป็น \[c\in\mathbb{Z}\wedge b=ac\] อย่างไรก็ดี เราต้องการแค่ $c$ เพียงตัวเดียวที่ทำให้เงื่อนไขข้างต้นเป็นจริง ดังนั้น เราจึงใช้ตัวบ่งปริมาณ $\exists$ เพื่อผูกมัดตัวแปร $c$ ไม่ให้เป็นอิสระอีกต่อไป กล่าวคือ \[\exists c:c\in\mathbb{Z}\wedge b=ac\]

เมื่อรวมเงื่อนไขทั้งหมดที่มี จะได้ว่า $\mathrm{divides}(a,b)$ เขียนเป็นรูปนัยได้ดังนี้ \[a\in\mathbb{Z}\wedge b\in\mathbb{Z}\wedge\exists c:c\in\mathbb{Z}\wedge b=ac\] สังเกตว่า ตัวแปรอิสระใน predicate ดังกล่าวคือ $a$ และ $b$ เนื่องจากไม่มีตัวบ่งปริมาณมาผูกมัด

นอกจากนี้ เราสามารถเขียน predicate ดังกล่าวโดยย่อได้ โดยสังเกตว่าหากสิ่งที่ตาม $\exists$ มาทันทีนั้นเป็น conjunction ที่ประกอบไปด้วยเงื่อนไขที่กล่าวถึงตัวแปรที่ถูกผูกมัด แล้วเราสามารถนำเงื่อนไขดังกล่าวมากำหนดขอบเขตของตัวแปรพร้อมกับ $\exists$ ได้ดังนี้
\[a\in\mathbb{Z}\wedge b\in\mathbb{Z}\wedge\exists c\in\mathbb{Z}: b=ac\]
กล่าวคือ ส่วนที่ระบุว่า $c$ ต้องเป็นจำนวนเต็มนั้นได้ถูกพ่วงไปพร้อมกับ $\exists$ ซึ่งกล่าวว่าเราจะพิจารณาเฉพาะจำนวนเต็มเท่านั้น ภายใต้โดเมนของจำนวนตรรกยะ

อย่างไรก็ดี หากเราเปลี่ยนโดเมนที่เราสนใจเป็นเซตของจำนวนเต็ม predicates ที่กล่าวว่าตัวแปรตัวหนึ่งๆ ต้องเป็นจำนวนเต็มก็ไม่มีความสำคัญอีกต่อไป เนื่องจาก predicates ดังกล่าวไม่ได้ลดขอบเขตหรือเปลี่ยนแปลงค่าความจริงของ $\mathrm{divides}$ แต่อย่างใด \enskip ดังนั้น เราสามารถละเงื่อนไขดังกล่าว ทำให้ผลลัพธ์เปลี่ยนแปลงเป็น \[\exists c: b=ac\] สังเกตว่า ตัวแปรอิสระใน predicate ดังกล่าวก็ยังเป็น $a$ และ $b$ เหมือนเดิม
\end{example}
ข้อสังเกตที่ได้จากตัวอย่างข้างต้น คือ โดเมนที่เราสนใจนั้น สามารถทำให้ผลลัพธ์ของ predicate ที่เราจะเขียนขึ้นมาแตกต่างกันได้ โดยโดเมนที่แตกต่างกัน อาจทำให้บางส่วนของ predicate ที่เรามีอยู่นั้นเป็นจริงเสมอ จึงไม่จำเป็นที่จะต้องเขียนส่วนนั้นๆ อีกต่อไป

\begin{example}
$\mathrm{prime}(n)$: ``$n$ เป็นจำนวนเฉพาะ'' กล่าวคือ $n$ เป็นจำนวนเต็มบวกที่มากกว่า 1 และไม่มีจำนวนเต็มบวกที่ไม่ใช่ 1 และ $n$ ที่หาร $n$ ลงตัว สามารถเขียนเป็นเชิงรูปนัยได้ดังนี้

เริ่มแรก เราต้องกำหนดขอบเขตของ $n$ ก่อนว่าเป็นจำนวนเต็มบวกที่มากกว่า 1 \[n\in\mathbb{Z}\wedge n>1\]

ถัดไป ต้องเพิ่มเงื่อนไขที่ว่า ไม่มีจำนวนเต็มบวกที่ไม่ใช่ 1 และ $n$ ที่หาร $n$ ลงตัว \enskip กล่าวอีกนัยหนึ่ง จำนวนเต็มบวกที่ตัวที่ไม่ใช่ 1 และ $n$ ต้องหาร $n$ ไม่ลงตัว \enskip ก่อนอื่น เรียกจำนวนเต็มบวกที่จะนำไปหารว่า $d$ จะได้ว่า หาก $d$ ผ่านเงื่อนไขว่าเป็นจำนวนเต็มบวก และ $d$ มีค่านอกเหนือจาก 1 และ $n$ แล้วเราจึงจะค่อยนำ $d$ ไปทดสอบว่าหาร $n$ ลงตัวหรือไม่ ซึ่งสามารถเขียนเป็นรูปนัยได้ดังนี้ \[d\in\mathbb{Z}\wedge d>0\wedge d\neq 1\wedge d\neq n \implies \neg\mathrm{divides}(d,n)\] นั่นคือ หาก $d$ ไม่ใช่จำนวนเต็ม ไม่เป็นบวก เท่ากับ 1 หรือเท่ากับ $n$ แล้ว $d$ ดังกล่าวจะไม่ผ่านเงื่อนไขที่จะนำไปทดสอบต่อไป \enskip กล่าวอีกนัยหนึ่ง เราจะไม่พิจารณา $d$ ที่ไม่ผ่านเงื่อนไขทั้งสี่ข้างต้น \enskip ดังนั้น $d$ ที่จะทำให้ implication ข้างต้นเป็นเท็จได้ จะต้องผ่านเงื่อนไขทั้งสี่ก่อน แต่เมื่อนำไปหาร $n$ แล้วทำให้หารลงตัว \enskip ในท้ายที่สุด เราต้องการกล่าวว่า $d$ ทุกตัวที่ตรงตามเงื่อนไขทั้งสี่ ต้องหาร $n$ ไม่ลงตัว แสดงว่าเราต้องใช้ตัวบ่งปริมาณ $\forall$ เพื่อผูกมัดตัวแปร $d$ ไม่ให้เป็นอิสระอีกต่อไป \[\forall d: d\in\mathbb{Z}\wedge d>0\wedge d\neq 1\wedge d\neq n \implies \neg\mathrm{divides}(d,n)\]

เมื่อรวมเงื่อนไขทั้งหมดที่มี จะได้ว่า $\mathrm{prime}(n)$ เขียนเป็นรูปนัยได้ดังนี้ \[n\in\mathbb{Z}\wedge n>1\wedge\forall d: d\in\mathbb{Z}\wedge d>0\wedge d\neq 1\wedge d\neq n \implies \neg\mathrm{divides}(d,n)\]
สังเกตว่า ตัวแปรอิสระใน predicate ดังกล่าวคือ $n$ เนื่องจากไม่มีตัวบ่งปริมาณมาผูกมัด

ใน predicate ข้างต้น เราทดสอบว่า $d$ ผ่านเงื่อนไขทั้งสี่ในเวลาเดียวกัน \enskip อย่างไรก็ดี เราสามารถแบ่งการทดสอบออกเป็นสองส่วนโดยแยก implications ได้ดังนี้ \[n\in\mathbb{Z}\wedge n>1\wedge\forall d: d\in\mathbb{Z}\wedge d>0 \implies d\neq 1\wedge d\neq n \implies \neg\mathrm{divides}(d,n)\]
หากจะเปรียบเทียบการทดสอบในลักษณะดังกล่าวกับการเขียนโปรแกรม จะได้ว่า การทดสอบในแบบแรกคล้ายคลึงกับการเขียนโปรแกรมโดยใช้คำสั่งเงื่อนไขคำสั่งเดียว โดยทดสอบหลายๆ อย่างในเวลาเดียวกันดังนี้
\begin{lstlisting}[language=Java,numbers=none]
if (B1 && B2) { ... }
\end{lstlisting}
ส่วนการทดสอบในแบบหลังนั้นคล้ายคลึงกับการใช้คำสั่งเงื่อนไขสองคำสั่ง โดยแยกทดสอบทีละอย่างดังนี้
\begin{lstlisting}[language=Java,numbers=none]
if (B1) {
  if (B2) { ... }
}
\end{lstlisting}
จะเห็นว่า โปรแกรมทั้งสองทำงานเหมือนกันทุกประการ \enskip ในทำนองเดียวกัน predicates ทั้งสองข้างต้นจึงมีความหมายเหมือนกันทุกประการด้วย

นอกจากนี้ เราสามารถเขียน predicate ดังกล่าวโดยย่อได้ โดยสังเกตว่าหากสิ่งที่ตาม $\forall$ มาทันทีนั้นเป็น implication ที่กล่าวถึงตัวแปรที่ถูกผูกมัด แล้วเราสามารถนำเงื่อนไขตั้งต้นมากำหนดขอบเขตของตัวแปรพร้อมกับ $\forall$ ได้ดังนี้ \[n\in\mathbb{Z}\wedge n>1\wedge\forall d\in\mathbb{Z}^+: d\neq 1\wedge d\neq n \implies \neg\mathrm{divides}(d,n)\]

สุดท้ายนี้ เราสามารถเขียน predicate ที่ตามหลัง $\forall$ โดยใช้ contrapositive ได้ดังนี้ด้วย
\[n\in\mathbb{Z}\wedge n>1\wedge \forall d\in\mathbb{Z}^+: \mathrm{divides}(d,n)\implies d=1\vee d=n\]
หากเขียนเป็นภาษาพูด จะได้ว่า $n$ เป็นจำนวนเฉพาะ ก็ต่อเมื่อ $n$ เป็นจำนวนเต็มบวก โดยที่จำนวนเต็มบวก ($d$) ใดๆ ที่หาร $n$ ลงตัวต้องเป็น 1 หรือ $n$ เท่านั้น \enskip สังเกตว่า ตัวแปรอิสระใน predicate ดังกล่าวก็ยังเป็น $n$ เหมือนเดิม

อย่างไรก็ดี หากโดเมนที่เราสนใจเป็นเซตของจำนวนเต็ม เราสามารถลดรูป predicate ข้างต้นโดยตัดพจน์ที่เป็นจริงเสมอออกไป ได้ผลลัพธ์เป็นดังนี้
\begin{align*}
& n>1\wedge \forall d>0: \mathrm{divides}(d,n)\implies d=1\vee d=n \\
\equiv\ & n>1\wedge \forall d: d>0 \implies \mathrm{divides}(d,n)\implies d=1\vee d=n \\
\equiv\ & n>1\wedge \forall d:  d>0 \wedge \mathrm{divides}(d,n) \implies d=1\vee d=n
\end{align*}
สังเกตว่า มีเพียงตัวแปร $n$ เท่านั้นที่เป็นตัวแปรอิสระ เนื่องจากตัวแปรอื่นๆ นั้นถูกผูกมัดด้วยตัวบ่งปริมาณแล้ว
\end{example}

\begin{example}
$\textrm{additive-identity}$: มีจำนวนเต็มที่นำไปบวกกับจำนวนเต็มใดๆ ก็ตาม แล้วได้ผลลัพธ์เป็นจำนวนเต็มใดๆ นั้น สามารถเขียนเป็นเชิงรูปนัยได้ดังนี้

ก่อนอื่น ให้ $x$ เป็นจำนวนเต็มที่เราต้องการ และให้ $y$ เป็นจำนวนเต็มใดๆ ที่จะนำ $x$ ไปบวก \enskip จะได้ว่า $x$ และ $y$ ต้องผ่านคุณสมบัติที่ว่า $x+y=y$ \enskip จะเห็นว่า ณ จุดนี้ ตัวแปรอิสระคือ $x$ และ $y$

ถัดไป เราต้องการให้ $y$ ทุกตัวที่เป็นจำนวนเต็มผ่านคุณสมบัติดังกล่าว \enskip ดังนั้น เราต้องจำกัดขอบเขตของ $y$ และใช้ตัวบ่งปริมาณ $\forall$ ผูกมัด $y$ ดังนี้ \[\forall y: y\in\mathbb{Z}\implies x+y=y\] ณ จุดนี้ ตัวแปรอิสระคือ $x$ เนื่องจาก $y$ ได้ถูกผูกมัดไปแล้ว

ท้ายที่สุด เราต้องการจำนวนเต็ม $x$ ตัวเดียวที่ทำให้คุณสมบัติดังกล่าวเป็นจริง \enskip ดังนั้น เราต้องจำกัดขอบเขตของ $x$ และใช้ตัวบ่งปริมาณ $\exists$ ผูกมัด $x$ ดังนี้ \[\exists x: x\in\mathbb{Z}\wedge(\forall y: y\in\mathbb{Z}\implies x+y=y)\]
เราสามารถเขียนย่อ implication ที่ตามหลัง $\forall$ ได้ในลักษณะเดียวกันกับตัวอย่างที่แล้ว ได้ผลลัพธ์เป็น \[\exists x: x\in\mathbb{Z}\wedge\forall y\in\mathbb{Z}: x+y=y\]

ในทำนองเดียวกันกับ $\forall$ หากสิ่งที่ตาม $\exists$ มาทันทีนั้นเป็น conjunction (เงื่อนไขที่ใช้ $\wedge$) ที่กล่าวถึงตัวแปรที่ถูกผูกมัด เราก็สามารถนำเงื่อนไขดังกล่าวมากำหนดขอบเขตของตัวแปรพร้อมกับ $\exists$ ได้ดังนี้ \[\exists x\in\mathbb{Z}\ \forall y\in\mathbb{Z}: x+y=y\] สังเกตว่า ไม่มีตัวแปรอิสระเหลืออยู่ในผลลัพธ์ดังกล่าวเลย

หากโดเมนที่เราสนใจเปลี่ยนเป็น $\mathbb{Z}$ เราสามารถลดรูป predicate ข้างต้นได้ดังนี้
\[\exists x\forall y: x+y=y\]
\end{example}

\begin{example}
$\textrm{unique-additive-identity}$: มี additive identity เพียงตัวเดียวเท่านั้น สามารถเขียนเป็นเชิงรูปนัยได้ดังนี้

ก่อนหน้านี้ เรามี predicate $\textrm{additive-identity}$ ที่กล่าวว่ามีเอกลักษณ์การบวกอยู่ โดยเราใช้ตัวบ่งปริมาณ $\exists$ กำกับว่ามีเอกลักษณ์ดังกล่าว \enskip หากจะกล่าวว่ามี additive identity เพียงตัวเดียวเท่านั้น สามารถเปลี่ยนตัวบ่งปริมาณ $\exists$ เป็น $\exists!$ ได้ดังนี้
\[\exists! x\ \forall y: x+y=y\]
อย่างไรก็ดี เราสามารถแปลง predicate ข้างต้นให้ใช้ $\exists$ เท่านั้น โดยคงความหมายเดิมได้ ด้วยการเพิ่มเงื่อนไขที่ว่า ถ้ามีเอกลักษณ์การบวกอื่นๆ (ที่ชื่อว่า $z$) แล้ว $z$ จะต้องเป็นตัวเดียวกันกับเอกลักษณ์การบวกที่เรามีอยู่แล้ว \enskip กล่าวอีกนัยหนึ่ง เอกลักษณ์การบวกที่เป็นไปได้ทั้งหมด มีเพียงตัวเดียวเท่านั้น

เงื่อนไขที่บังคับให้มีเอกลักษณ์การบวกอย่างมากเพียงตัวเดียวเท่านั้น (ถ้ามี) สามารถเขียนเป็น predicate ได้ดังนี้
\[\forall z\in\mathbb{Z}: z+y=y \implies z=x\]
เมื่อนำเงื่อนไขข้างต้นไปรวมกับเงื่อนไขก่อนหน้านี้ ซึ่งกล่าวว่ามีเอกลักษณ์การบวกอย่างน้อยหนึ่งตัว ก็จะได้เงื่อนไขที่กล่าวว่า มีเอกลักษณ์การบวกหนึ่งตัวพอดี ไม่ขาดไม่เกิน ดังนี้
\[\exists x\in\mathbb{Z}\ \forall y\in\mathbb{Z}: x+y=y\wedge\forall z\in\mathbb{Z}: z+y=y \implies z=x\]
สังเกตว่า ไม่มีตัวแปรอิสระเหลืออยู่ใน predicate ข้างต้นเลย
\end{example}

\begin{exercise}
    เติมวงเล็บในประพจน์ $\textrm{unique-additive-identity}$ ข้างต้นเพื่อระบุขอบเขตของตัวบ่งปริมาณแต่ละตัวให้ชัดเจน
\end{exercise}

\begin{exercise}
ใน predicate ต่อไปนี้ \[(\exists x: x>0\wedge y=x+1)\implies\forall z: yz=x\] มีตัวแปรที่ถูกใช้งาน 6 ตำแหน่ง \enskip ในแต่ละตำแหน่ง ให้พิจารณาว่าตัวแปรนั้นๆ อิสระหรือไม่
\iffalse
\begin{itemize}
\item ในตำแหน่งแรก ตัวแปร $x$ นั้นไม่อิสระ (bound) เนื่องจาก $x$ ตัวนี้มี $\exists$ มาผูกมัดแล้ว
\item ในตำแหน่งที่สอง ตัวแปร $y$ นั้นอิสระ เนื่องจากไม่มีตัวบ่งปริมาณมาผูกมัด $y$
\item ในตำแหน่งที่สาม ตัวแปร $x$ นั้นไม่อิสระ เช่นเดียวกับตำแหน่งที่ 1
\item ในตำแหน่งที่สี่ ตัวแปร $y$ นั้นอิสระ เช่นเดียวกับตำแหน่งที่ 2
\item ในตำแหน่งที่ห้า ตัวแปร $z$ นั้นไม่อิสระ เนื่องจาก $z$ มี $\forall$ มาผูกมัดแล้ว
\item ในตำแหน่งสุดท้าย ตัวแปร $x$ นั้นอิสระ เนื่องจาก $\exists$ ที่ผูกมัด $x$ ก่อนหน้านี้ไม่ได้ครอบคลุมมาถึง $x$ ตัวนี้ \enskip กล่าวอีกนัยหนึ่ง $x$ ตัวนี้อยู่นอกเหนือขอบเขต (scope) ของ $\exists$ ที่ผูกมัด $x$ ตัวอื่นๆ นั่นคือ $x$ ตัวนี้ไม่มีตัวบ่งปริมาณมาผูกมัด
\end{itemize}
\fi
\end{exercise}

การใช้ตัวบ่งปริมาณ เกิดขึ้นในบริบทอื่นๆ ในวิชาคณิตศาสตร์และคอมพิวเตอร์ ดังในนิยามต่อไปนี้
\begin{definition}
ฟังก์ชัน $f(x)$ เป็น $O(g(x))$ หาก $\exists n_0\in\mathbb{R}\ \exists k\in\mathbb{R}^+\ \forall n\geq n_0: |f(n)|\leq kg(n)$
\end{definition}
นิยามนี้กล่าววว่า ฟังก์ชัน $f$ โตไม่เร็วกว่าฟังก์ชัน $g$ ในอัตราส่วนที่ไม่เกินค่าคงที่ $k$ เมื่อค่าของ input นั้นเป็นอย่างน้อย $n_0$

\begin{definition}
ฟังก์ชัน $f(x)$ เป็น $\Omega(g(x))$ หาก $\exists n_0\in\mathbb{R}\ \exists k\in\mathbb{R}^+\ \forall n\geq n_0: kg(n)\leq f(n)$
\end{definition}
นิยามนี้กล่าววว่า ฟังก์ชัน $f$ โตไม่ช้ากว่าฟังก์ชัน $g$ ในอัตราส่วนที่ไม่ต่ำกว่าค่าคงที่ $k$ เมื่อค่าของ input นั้นเป็นอย่างน้อย $n_0$
    
\begin{definition}
ฟังก์ชัน $f(x)$ เป็น $\Theta(g(x))$ หาก \[\exists n_0\in\mathbb{R}\ \exists k_b\in\mathbb{R}^+\ \exists k_t\in\mathbb{R}^+\ \forall n\geq n_0: k_bg(n)\leq f(n)\leq k_tg(n)\]
\end{definition}
นิยามนี้กล่าววว่า ฟังก์ชัน $f$ โตอัตราเดียวกันกับฟังก์ชัน $g$ ในอัตราส่วนที่อยู่ระหว่างค่าคงที่ $k_b$ และ $k_t$ เมื่อค่าของ input นั้นเป็นอย่างน้อย $n_0$

\begin{definition}
$\lim_{x\to k}{f(x)}=\ell$ หาก $\forall\varepsilon>0\ \exists\delta>0\ \forall x: 0<|x-k|<\delta\implies|f(x)-l|<\varepsilon$
\end{definition}
นิยามนี้กล่าวว่า ลิมิตของฟังก์ชัน $f$ ณ จุด $k$ มีค่าเป็น $\ell$ หากไม่ว่าเราจะกำหนดช่วงของค่าของฟังก์ชันที่ใกล้ $\ell$ มากเพียงใด (ต่างจาก $\ell$ น้อยกว่า $\varepsilon$) เราสามารถหาช่วงของค่า input ของฟังก์ชันที่อยู่ใกล้ $k$ ได้ (ต่างจาก $k$ น้อยกว่า $\delta$) ที่ทำให้ค่าของฟังก์ชันนี้อยู่ในช่วง $(\ell-\varepsilon,\ell+\varepsilon)$ เสมอ (ยกเว้นค่าของฟังก์ชันที่ $k$ ซึ่งไม่จำเป็นต้องอยู่ในช่วงนี้)
