\section{Proof by cases}

\emph{การพิสูจน์โดยแยกกรณี} (proof by cases) เป็นกลวิธีหนึ่งที่ใช้ในการพิสูจน์ประพจน์ต่างๆ โดยมักจะเหมาะสมเมื่อมีกรณีต่างๆ ที่ต้องพิจารณา แต่ต้องการข้อสรุปอย่างเดียวกัน

ในการพิสูจน์โดยแยกกรณี ให้พิจารณากรณีทั้งหมดที่เป็นไปได้ โดยในแต่ละกรณี ให้แสดงว่ากรณีนั้นๆ ให้ผลลัพธ์ตามที่ต้องการ เมื่อทุกกรณีที่เราพิจารณาให้ผลลัพธ์ตามที่ต้องการเหมือนกันหมด การพิสูจน์ก็เป็นอันเสร็จสิ้น

\begin{theorem}
กำหนดให้ $a$ และ $b$ เป็นจำนวนจริง จะได้ว่า $\min(a,b)+\max(a,b)=a+b$

\begin{pf}
เนื่องจากฟังก์ชัน $\min$ และ $\max$ ให้ผลลัพธ์ต่างๆ กันไปตามค่าของ $a$ และ $b$ เราจะทำการพิสูจน์โดยแยกกรณี โดยมี 3 กรณีด้วยกัน
\begin{enumerate}
\item $a<b$: ในกรณีนี้ $\min(a,b)=a$ และ $\max(a,b)=b$ จะได้ว่า $\min(a,b)+\max(a,b)=a+b$

\item $a>b$: ในกรณีนี้ $\min(a,b)=b$ และ $\max(a,b)=a$ จะได้ว่า $\min(a,b)+\max(a,b)=b+a=a+b$

\item $a=b$: ในกรณีนี้ $\min(a,b)=\max(a,b)=a=b$ จะได้ว่า $\min(a,b)+\max(a,b)=a+b$
\end{enumerate}
เนื่องจากทั้ง 3 กรณีได้ผลเป็นอย่างเดียวกัน กล่าวคือ $\min(a,b)+\max(a,b)=a+b$ และทั้ง 3 กรณีนี้ครอบคลุมกรณีทั้งหมดที่เป็นไปได้ จึงสรุปได้ว่า $\min(a,b)+\max(a,b)=a+b$ ตามที่ต้องการ
\end{pf}
ในบทพิสูจน์ข้างต้นนั้น เราได้ทำการพิสูจน์ประพจน์ต่อไปนี้
\[\forall a,b\in\mathbb{R}: a<b\vee a>b\vee a=b \implies \min(a,b)+\max(a,b)=a+b\]
ซึ่งมีความหมายเดียวกันกับประพจน์ที่ให้มาในทฤษฎีบทเริ่มแรก
\end{theorem}

แท้จริงแล้ว การพิสูจน์โดยแยกกรณี เป็นกฎการอนุมานกฎหนึ่งของตัวดำเนินการ $\vee$

\paragraph{Inference rules for disjunctions}
กฎการอนุมานเพิ่มเติมที่เกี่ยวกับตัวดำเนินการ $\vee$ มีดังนี้
%
\[
\inferrule*[Right=$\vee$-elim]
{\Gamma,P\vdash R \\ \Gamma,Q\vdash R}
{\Gamma, P\vee Q\vdash R}
\]
Inference rule นี้กล่าวว่า ถ้าเรามี $P\vee Q$ เป็นสมมุติฐาน แล้วต้องการพิสูจน์ $R$ ให้แยกกรณีว่า
\begin{itemize}
\item $P$ เป็นจริง แล้วพิสูจน์ $R$
\item $Q$ เป็นจริง แล้วพิสูจน์ $R$
\end{itemize}
หากสามารถพิสูจน์ได้ทั้งคู่ ก็สรุปได้ว่า ไม่ว่า $P$ หรือ $Q$ จะเป็นจริง $R$ จะต้องเป็นจริงด้วย \enskip จะเห็นว่า กฎนี้เป็นกฎที่อธิบายวิธีการพิสูจน์โดยแยกกรณี (proof by cases) ในตัวอย่างข้างต้นนั่นเอง

\paragraph{Inference rules for existential quantification}
กฎการอนุมานเพิ่มเติมที่เกี่ยวกับตัวบ่งปริมาณ $\exists$ มีดังนี้
%
\[
\inferrule*[Right=$\exists$-elim]
{\Gamma,d\in D,P(d)\vdash Q}
{\Gamma,\exists x: P(x)\vdash Q}
\]
Inference rule นี้กล่าวว่า หากทราบว่า $\exists x: P(x)$ เป็นจริง แล้วต้องการพิสูจน์ $Q$ ให้สมมุติว่า $d$ เป็นสมาชิก\emph{ใดๆ} ของโดเมนที่ทำให้ $P(d)$ เป็นจริง แสดงพยายามสรุปว่า $Q$ เป็นจริง

กฎนี้สามารถตีความได้คล้ายกับ \textsc{$\vee$-elim} ที่กล่าวว่า หากเราทราบว่า $P\vee Q$ เป็นจริง แล้วต้องการพิสูจน์ $R$ ให้แยกกรณีว่า $P$ เป็นจริงแล้วพิสูจน์ $R$ และแยกอีกกรณีว่า $Q$ เป็นจริงแล้วพิสูจน์ $R$ เนื่องจากเราอาจจะไม่ทราบแน่ชัดว่า $P$ หรือ $Q$ ที่เป็นจริงกันแน่ \enskip ในที่นี้ หากเราทราบว่า $\exists x: P(x)$ เป็นจริง แล้วต้องการพิสูจน์ $Q$ ให้แยกกรณีพิจารณาว่า $P(x)$ เป็นจริงสำหรับ $x$ แต่ละตัวในโดเมน แล้วพิสูจน์ $Q$ เนื่องจากเราอาจจะไม่ทราบแน่ชัดว่า $x$ ตัวใดในโดเมนจะทำให้ $P(x)$ เป็นจริง
%
\begin{example}
ให้โดเมนที่เราสนใจเป็นเซตของจำนวนเต็ม \enskip บทพิสูจน์ที่ว่า ถ้ามีจำนวนเต็มที่มากกว่า 61 แล้วจะมีจำนวนเต็มที่มากกว่า 47 สามารถเขียนเป็นบทพิสูจน์ได้ดังนี้
\begin{pf}
สมมุติว่ามีจำนวนเต็มที่มากกว่า 61 ต้องพิสูจน์ว่า มีจำนวนเต็มที่มากกว่า 47 (\textsc{$\Rightarrow$-intro}) \enskip ให้ $x$ เป็นจำนวนเต็มใดๆ ที่มากกว่า 61 ต้องพิสูจน์ว่า มีจำนวนเต็มที่มากกว่า 47 (\textsc{$\exists$-elim}) \enskip เนื่องจาก $x$ มากกว่า 61 จะได้ว่า $x$ มากกว่า 47 ด้วย \enskip ดังนั้น มีจำนวนเต็มที่มากกว่า 47 (\textsc{$\exists$-intro})
\end{pf}
\end{example}

\begin{theorem}
มีจำนวน\emph{อตรรกยะ} (irrationals) $a$ และ $b$ ที่ทำให้ $a^b\in\mathbb{Q}$

\begin{pf}
เพียงแค่เราสามารถหาค่าของ $a$ และ $b$ ที่ตรงตามเงื่อนไขได้ การพิสูจน์นี้ก็จะเสร็จสิ้นสมบูรณ์

อันดับแรก เนื่องจาก $\sqrt{2}$ เป็นจำนวนอตรรกยะ ให้พิจารณาว่า $\sqrt{2}^{\sqrt{2}}$ เป็นจำนวนตรรกยะหรือไม่ ในที่นี้ แม้ว่าเราจะไม่สามารถบ่งชี้ไปอย่างใดอย่างหนึ่งได้อย่างแน่ชัด แต่เราจะใช้การพิสูจน์โดยแยกกรณีว่า ไม่ว่าจะกรณีใดก็ตาม เราสามารถหา $a$ และ $b$ ตามที่ต้องการได้
\begin{enumerate}
\item $\sqrt{2}^{\sqrt{2}}$ เป็นจำนวนตรรกยะ ในกรณีนี้ เราได้ค่าของ $a$ และ $b$ ที่ทำให้ $a^b$ เป็นจำนวนตรรกยะ กล่าวคือ $a=b=\sqrt{2}$
\item $\sqrt{2}^{\sqrt{2}}$ ไม่เป็นจำนวนตรรกยะ นั่นคือ $\sqrt{2}^{\sqrt{2}}$ เป็นจำนวนอตรรกยะ ในกรณีนี้ พิจารณา $\left(\sqrt{2}^{\sqrt{2}}\right)^{\sqrt{2}}$ จะได้ว่า \[\left(\sqrt{2}^{\sqrt{2}}\right)^{\sqrt{2}}=\sqrt{2}^2=2\] ซึ่งเป็นจำนวนตรรกยะ ดังนั้น ในกรณีนี้ เราได้ค่าของ $a$ และ $b$ ที่ทำให้ $a^b$ เป็นจำนวนตรรกยะ กล่าวคือ $a=\sqrt{2}^{\sqrt{2}}$ และ $b=\sqrt{2}$
\end{enumerate}
ในทั้งสองกรณี เราสามารถหาค่า $a$ และ $b$ ตามเงื่อนไขได้ จึงสรุปได้ว่า มีจำนวนอตรรกยะ $a$ และ $b$ ที่ทำให้ $a^b\in\mathbb{Q}$ ตามที่ต้องการ
\end{pf}
แท้จริงแล้ว ในบทพิสูจน์ข้างต้น เราได้ทำการพิสูจน์ประพจน์
\[(\sqrt{2}^{\sqrt{2}}\in\mathbb{Q}\vee\sqrt{2}^{\sqrt{2}}\notin\mathbb{Q})\implies\exists a,b\in\mathbb{R}-\mathbb{Q}: a^b\in\mathbb{Q}\]
ซึ่งมีความหมายเดียวกับประพจน์ที่กำหนดมาให้ในทฤษฎีบท
\end{theorem}
จะเห็นว่า เทคนิคการพิสูจน์โดยแยกกรณีนั้น เราทำการพิสูจน์ประพจน์ที่แตกต่างไปจากประพจน์ที่กำหนดให้ แต่ยังคงความหมายเดิมอยู่ \enskip ความท้าทายของการพิสูจน์โดยทั่วไปนั้น คือการเลือกเฟ้นประพจน์ที่เราสามารถนำมาใช้ในการให้เหตุผลได้สะดวกยิ่งขึ้น ดังที่ได้สาธิตในตัวอย่างข้างต้นนี้ \enskip ความท้าทายดังกล่าวเป็นเหตุผลหลักที่ทำให้มนุษย์ยังมีบทบาทสำคัญในการเขียนบทพิสูจน์ เนื่องจากเรายังไม่สามารถใช้คอมพิวเตอร์สร้างบทพิสูจน์ได้ทั้งหมดโดยอัตโนมัติ \enskip อย่างไรก็ดี แม้ว่าคอมพิวเตอร์จะยังไม่สามารถเขียนบทพิสูจน์ได้เอง แต่ก็มีโปรแกรมช่วยเขียนบทพิสูจน์ต่างๆ ที่ทำให้มนุษย์มั่นใจขึ้นว่า บทพิสูจน์ที่ตนเองคิดค้นขึ้นมานั้นถูกต้องและครอบคลุมทุกกรณีที่เป็นไปได้
