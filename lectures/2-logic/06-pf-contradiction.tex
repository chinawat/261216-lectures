\section{Proof by contradiction}

\emph{การพิสูจน์โดยข้อขัดแย้ง} (proof by contradiction) เป็นอีกกลวิธีหนึ่งที่ใช้ในการพิสูจน์ โดยมักจะเหมาะสมเมื่อสามารถแสดงได้ว่า นิเสธของประพจน์ที่ต้องการพิสูจน์นั้นเป็นไปไม่ได้ ซึ่งแสดงว่าประพจน์ที่ต้องการพิสูจน์นั้นเป็นจริง

การพิสูจน์ว่าประพจน์ $P$ เป็นจริงนั้น เทียบเท่ากับการพิสูจน์ว่า $\true\implies P$ เป็นจริง เนื่องจากทั้งสองประพจน์สมมูลกัน (สามารถตรวจสอบได้จากตารางค่าความจริง) แต่เนื่องจาก $\true\implies P$ มี contrapositive เป็น $\neg P\implies\false$ จะได้ว่า หากต้องการพิสูจน์ $P$ ก็สามารถพิสูจน์ $\neg P\implies\false$ ได้ เนื่องจากทั้งสองประพจน์สมมูลกัน

Proofs by contradiction เป็นการให้เหตุผลที่มีนิเสธเข้ามาเกี่ยวข้อง

\paragraph{Inference rules for negations}
กฎการอนุมานที่เกี่ยวกับตัวดำเนินการ $\neg$ มีดังนี้
\[
\inferrule*[Right=$\neg$-intro]
{\Gamma,P\vdash Q \\ \Gamma,P\vdash\neg Q}
{\Gamma\vdash \neg P}
\]
Inference rule นี้กล่าวว่า หากเราต้องการพิสูจน์ว่า $\neg P$ กล่าวคือ $P$ เป็นเท็จ ให้สมมุติว่า $P$ เป็นจริง แล้วพยายามหา $Q$ ที่เราสามารถพิสูจน์ได้ว่าเป็นจริงและเท็จทั้งคู่ \enskip กล่าวอีกนัยหนึ่ง หากเรามี $P$ เป็นสมมุติฐาน แล้วเราสามารถพิสูจน์ว่าสิ่งใดสิ่งหนึ่ง ($Q$ ในที่นี้) เป็นจริงและเป็นเท็จได้ทั้งคู่ แสดงว่า สมมุติฐาน $P$ ที่เรามีอยู่นั้นไม่สมเหตุสมผล หรือเป็นไปไม่ได้ \enskip ดังนั้น จึงสรุปได้ว่า $P$ ต้องเป็นเท็จ
% กฎนี้เป็นกฎที่อธิบายวิธีการพิสูจน์โดยข้อขัดแย้ง (proof by contradiction) ดังที่จะได้กล่าวต่อไป

อย่างไรก็ดี กฎ \textsc{$\neg$-intro} นี้ไม่จำเป็นต้องสร้างขึ้นมาเป็นกฎใหม่ เนื่องจากเราสามารถสร้างกฎนี้ได้จาก inference rules ที่เรามีอยู่ก่อนหน้านี้ โดยพึงระลึกว่า $\neg P\equiv P\implies\textrm{false}$ \enskip Proof tree ของกฎ \textsc{$\neg$-intro} สามารถเขียนได้ดังนี้
\[
\inferrule*[Right=$\implies$-intro]
{
\inferrule*[Right=$\implies$-elim,leftskip=3in,rightskip=3in]
{\Gamma,P\vdash Q \\ \Gamma,P\vdash Q\implies\textrm{false}}
{\Gamma,P\vdash \textrm{false}}
}
{\Gamma\vdash P\implies \textrm{false}}
\]
หากเรามี $P$ เป็นสมมุติฐาน แล้วเราสรุปได้ว่า $Q$ เป็นจริง และ $Q\implies\textrm{false}$ เป็นจริง กล่าวคือ $Q$ เป็นเท็จ เราสามารถใช้กฎ \textsc{$\Rightarrow$-elim} สรุปได้ว่า $\textrm{false}$ เป็นจริง \enskip ดังนั้น เราสามารถใช้กฎ \textsc{$\Rightarrow$-intro} สรุปได้ว่า หากไม่ตั้ง $P$ เป็นสมมุติฐาน เราสามารถพิสูจน์ว่า $P\implies\textrm{false}$ ได้ ซึ่งแปลว่า $\neg P$ ต้องเป็นจริงนั่นเอง

\[
\inferrule*[Right=$\neg$-elim]
{\Gamma\vdash\textrm{false}}
{\Gamma\vdash P}
\]
Inference rule นี้กล่าวว่า หากเราพิสูจน์ได้ว่า $\textrm{false}$ เป็นจริง แล้วเราจะสรุปได้ว่าประพจน์ใดๆ ($P$ ในที่นี้) ก็ย่อมต้องเป็นจริงด้วย \enskip กล่าวอีกนัยหนึ่ง ถ้าสิ่งที่เป็นไปไม่ได้เกิดเป็นไปได้ขึ้นมา สิ่งอื่นๆ ที่เหลือก็ย่อมเป็นไปได้โดยปริยาย
%
\begin{example}
Proof tree ต่อไปนี้เป็นบทพิสูจน์ว่า $(P\vee Q)\wedge\neg Q \implies P$
\[
\inferrule*[Right=$\implies$-intro]
{
  \inferrule*[Right=$\wedge$-elim,leftskip=5in,rightskip=5in]
  {
    \inferrule*[Right=$\vee$-elim,leftskip=5in,rightskip=5in]
    {
      \inferrule*[right=axiom]
      { }
      {P,\neg Q\vdash P}
      \\
      \inferrule*[Right=$\neg$-elim]
      {
        \inferrule*[Right=$\implies$-elim,leftskip=1in,rightskip=1in]
        {
          \inferrule*[right=axiom]
          { }
          {Q,\neg Q\vdash Q}
          \\
          \inferrule*[Right=axiom]
          { }
          {Q,\neg Q\vdash Q\implies\textrm{false}}
        }
        {Q,\neg Q\vdash \textrm{false}}
      }
      {Q,\neg Q\vdash P}
    }
    {P\vee Q,\neg Q\vdash P}
  }
  {(P\vee Q)\wedge\neg Q\vdash P}
}
{\vdash (P\vee Q)\wedge\neg Q \implies P}
\]
หากจะเขียน proof tree ดังกล่าวเป็นร้อยแก้ว จะได้ดังนี้

สมมุติว่า $P\vee Q$ และ $\neg Q$ เป็นจริง ต้องพิสูจน์ว่า $P$ เป็นจริง \enskip เนื่องจาก $P\vee Q$ จึงพิสูจน์โดยแยกกรณีดังนี้ \enskip ในกรณีแรก $P$ เป็นจริง จึงสรุปได้ทันทีว่า $P$ เป็นจริง \enskip ในกรณีที่สอง $Q$ เป็นจริง แต่เนื่องจาก $\neg Q$ เป็นจริงด้วย จึงเป็นไปไม่ได้ (สามารถสรุปได้ว่า $P$ เป็นจริงโดยปริยาย) ดังนั้น จึงสรุปได้ว่า $P$ เป็นจริงในทุกกรณี
\end{example}

\[
\inferrule*[Right=$\neg\neg$-elim]
{\Gamma\vdash\neg\neg P}
{\Gamma\vdash P}
\]
Inference rule นี้กล่าวว่า หากเราพิสูจน์ได้ว่า $\neg\neg P$ เป็นจริง แล้วเราจะสรุปได้ว่า $P$ ต้องเป็นจริงด้วย \enskip ทั้งนี้ inference rules อื่นๆ ที่เรามีอยู่ไม่สามารถกำจัด negations ที่ซ้อนกันออกไปได้ จึงจำเป็นต้องสร้างกฎนี้ขึ้นมาเพิ่มเติมเพื่อขั้นตอนนี้โดยเฉพาะ \enskip กฎนี้เป็นกฎที่อยู่ใน classical logic เท่านั้น และมักจะพบใน proof by contradiction

ในการพิสูจน์ประพจน์ $P$ โดยข้อขัดแย้ง ให้ตั้งสมมุติฐานว่า $\neg P$ เป็นจริง แล้วพยายามแสดงว่าความเท็จจะเป็นจริงได้ กล่าวคือ เกิดความเป็นไปไม่ได้ขึ้นเนื่องจากมีบางอย่างขัดแย้งกันโดยสิ้นเชิง

\begin{example}
หากต้องการพิสูจน์ $P\implies Q$ โดยข้อขัดแย้ง ให้พิสูจน์ว่า $\neg(P\implies Q)\implies\false$ ซึ่ง $\neg(P\implies Q)$ มีค่าความจริงดังตารางนี้
\[
\begin{array}{c|c|c|c}
P & Q & P\implies Q & \neg(P\implies Q) \\\hline
\true & \true & \true & \false \\
\true & \false & \false & \true \\
\false & \true & \true & \false \\
\false & \false & \true & \false \\
\end{array}
\]

จะเห็นว่า กรณีที่เราต้องให้ความสนใจมีเพียงกรณีเดียว กล่าวคือ เมื่อ $P$ เป็นจริง และ $Q$ เป็นเท็จ ดังนั้น ในการพิสูจน์นี้ ให้ตั้งสมมุติฐานว่า $P\wedge\neg Q$ เป็นจริง แล้วพยายามสร้างข้อขัดแย้ง
\end{example}

\begin{theorem}\label{thm:sq-e-e}
ให้ $a\in\mathbb{N}$ ถ้า $a^2$ เป็นเลขคู่ แล้ว $a$ เป็นเลขคู่
\begin{pf}
By contradiction. สมมุติว่า $a^2$ เป็นเลขคู่ แต่ $a$ ไม่เป็นเลขคู่ กล่าวคือ $a$ เป็นเลขคี่

เนื่องจาก $a$ เป็นเลขคี่ จะได้ว่า มี $x\in\mathbb{Z}$ ที่ทำให้ $a=2x+1$ ดังนั้น $a^2=(2x+1)^2=4x^2+4x+1$

ในขณะเดียวกัน อีกสมมุติฐานหนึ่งกล่าวว่า $a^2$ เป็นเลขคู่ ซึ่งแปลว่ามี $y\in\mathbb{Z}$ ที่ทำให้ $a^2=2y$

ดังนั้น $2y=4x^2+4x+1$ นั่นคือ $y=4x^2+4x+\frac{1}{2}$ ซึ่งไม่เป็นจำนวนเต็ม ขัดแย้งกับสมมุติฐานว่า $y\in\mathbb{Z} \lightning$

จะได้ว่า ถ้า $a^2$ เป็นเลขคู่ แล้ว $a$ เป็นเลขคู่ ตามที่ต้องการ
\end{pf}
\end{theorem}

\begin{theorem}
$\sqrt{2}$ เป็นจำนวนอตรรกยะ
\begin{pf}
By contradiction. สมมุติว่า $\sqrt{2}$ เป็นจำนวนตรรกยะ จะได้ว่ามี $a,b\in\mathbb{Z}$ ที่ทำให้ $\sqrt{2}=\frac{a}{b}$ นอกจากนี้ เนื่องจากเศษส่วนจำนวนเต็มใดๆ สามารถทำให้เป็นเศษส่วนอย่างต่ำได้ หากมี $a,b$ หลายชุด ให้เลือกชุดที่เป็นเศษส่วนอย่างต่ำด้วย

ยกกำลังสองทั้งสองข้าง จะได้ $2=\frac{a^2}{b^2}$ นั่นคือ $2b^2=a^2$ ซึ่งแปลว่า $a^2$ เป็นเลขคู่ จาก Theorem~\ref{thm:sq-e-e} จะได้ว่า $a$ เป็นเลขคู่ จากนิยามของเลขคู่ เราสามารถเขียน $a=2x$ โดยที่ $x\in\mathbb{Z}$ ได้ จะได้ว่า $a^2=(2x)^2=4x^2$ ดังนั้น $2b^2=a^2=4x^2$ นั่นคือ $b^2=2x^2$ ซึ่งแปลว่า $b^2$ เป็นเลขคู่

จาก Theorem~\ref{thm:sq-e-e} จะได้ว่า $b$ เป็นเลขคู่ด้วย จะเห็นได้ว่า เราค้นพบว่า ทั้ง $a$ และ $b$ เป็นเลขคู่ ขัดแย้งกับสมมุติฐานว่า $\frac{a}{b}$ เป็นเศษส่วนอย่างต่ำ เนื่องจากเราสามารถใช้ 2 หารได้ทั้งเศษและส่วน $\lightning$

ดังนั้น $\sqrt{2}$ เป็นจำนวนอตรรกยะ
\end{pf}

\end{theorem}
%
\begin{example}
Proof tree ต่อไปนี้เป็นบทพิสูจน์ว่า $P\vee\neg P$
{\small
\[
\hspace{1in}
\inferrule*[Right=$\neg\neg$-elim]
{
  \inferrule*[Right=$\neg$-intro,leftskip=1.5in,rightskip=1.5in]
  {
    \inferrule*[right=$\vee$-intro-2]
    {
      \inferrule*[Right=$\neg$-intro,leftskip=5in,rightskip=5in]
      {
        \inferrule*[right=$\vee$-intro-1]
        {
          \inferrule*[Right=axiom]
          { }
          {\neg(P\vee\neg P),P\vdash P}
        }
        {\neg(P\vee\neg P),P\vdash P\vee\neg P}
        \\
        \inferrule*[Right=axiom]
        { }
        {\neg(P\vee\neg P),P\vdash\neg(P\vee\neg P)}
      }
      {\neg(P\vee\neg P)\vdash \neg P}
    }
    {\neg(P\vee\neg P)\vdash P\vee\neg P}
    \\
    \inferrule*[Right=axiom]
    { }
    {\neg(P\vee\neg P)\vdash\neg(P\vee\neg P)}
  }
  {\vdash \neg\neg(P\vee\neg P)}
}
{\vdash P\vee\neg P}
\]
}
หากจะเขียน proof tree ดังกล่าวเป็นร้อยแก้ว จะได้ดังนี้

Proof by contradiction. \enskip สมมุติว่า $P\vee\neg P$ เป็นเท็จ กล่าวคือ $\neg(P\vee\neg P)$ เป็นจริง \enskip ก่อนอื่น จะพิสูจน์ว่า $\neg P$ เป็นจริง กล่าวคือ ถ้า $P$ เป็นจริงแล้วจะเกิดข้อขัดแย้งขึ้น \enskip สมมุติว่า $P$ เป็นจริง จะได้ว่า $P\vee\neg P$ เป็นจริงด้วย แต่เนื่องจากเรามีสมมุติฐานว่า $\neg(P\vee\neg P)$ ด้วย จึงเป็นไปไม่ได้ ดังนั้น $P$ ต้องไม่เป็นจริง กล่าวคือ $\neg P$ เป็นจริง \enskip ลำดับถัดไป เนื่องจาก $\neg P$ เป็นจริง จึงสรุปได้ว่า $P\vee\neg P$ เป็นจริงด้วย แต่ก็ขัดแย้งกับสมมุติฐานเดิมที่ว่า $\neg(P\vee\neg P)$ เป็นจริงอยู่ดี ดังนั้น จึงเป็นไปไม่ได้ที่ $\neg(P\vee\neg P)$ จะเป็นจริง $\lightning$ นั่นคือ $P\vee\neg P$ ต้องเป็นจริงนั่นเอง

สังเกตว่า บทพิสูจน์ในรูปร้อยแก้วดังกล่าวอาจจะทำความเข้าใจได้ยากกว่าปกติ เนื่องจากเป็นการพิสูจน์สิ่งที่ชัดเจนอยู่แล้วว่าต้องเป็นจริง (อย่างน้อยในระบบการให้เหตุผลแบบ classical logic ที่เรามักจะคุ้นเคยกัน) \enskip อย่างไรก็ดี ความยากของบทพิสูจน์นี้อยู่ตรงที่เราไม่สามารถตั้งสมมุติฐานว่า $P\vee\neg P$ เป็นจริงได้ทันที จึงต้องใช้วิธีการที่ค่อนข้างซับซ้อนในการสร้างกรณีขึ้นมาว่า ไม่ว่า $P$ จะเป็นจริงหรือไม่ ก็เป็นไปไม่ได้ที่ $\neg(P\vee\neg P)$ จะเป็นจริง
\end{example}

\begin{example}
ในตัวอย่างนี้ จะแสดงว่า $\inferrule{\Gamma\vdash P}{\Gamma\vdash\neg\neg P}$ ไม่จำเป็นต้องสร้างเป็นกฎใหม่ขึ้นมา เนื่องจากเราสามารถใช้กฎที่มีอยู่ในการเขียน proof tree ได้ดังนี้
\[
\inferrule*[Right=$\neg$-intro]
{
  \inferrule*[right=weakening]
  {\Gamma\vdash P}
  {\Gamma,\neg P\vdash P}
  \\
  \inferrule*[Right=axiom]
  { }
  {\Gamma,\neg P\vdash\neg P}
}
{\Gamma\vdash\neg\neg P}
\]
Proof by contradiction. \enskip สมมุติว่า $\neg P$ เป็นจริง \enskip เนื่องจากเราได้พิสูจน์แล้วว่า $P$ เป็นจริงด้วย จึงเกิดข้อขัดแย้งขึ้น $\lightning$ ดังนั้น $\neg P$ ต้องเป็นเท็จ กล่าวคือ $\neg\neg P$ ต้องเป็นจริง
\end{example}
%
ในตัวอย่างดังกล่าว มีการใช้กฎที่ละสมมุติฐานที่ไม่จำเป็นต้องใช้ทิ้งไป ซึ่งเขียนในรูปทั่วไปได้ดังนี้
\[
\inferrule*[Right=weakening]
{\Gamma\vdash P}
{\Gamma,\Delta\vdash P}
\]
Inference rule นี้กล่าวว่า หากเราใช้สมมุติฐาน $\Gamma$ ในการพิสูจน์ $P$ ได้แล้ว แม้ว่าจะมีสมมุติฐาน $\Delta$ มาให้เราเพิ่มเติม เราก็ยังสามารถพิสูจน์ $P$ ได้อยู่ดี \enskip กล่าวอีกนัยหนึ่ง หาก $\Delta$ ไม่จำเป็นต้องใช้ในการพิสูจน์ $P$ ก็สามารถละ $\Delta$ ทิ้งไปได้

\begin{example}
ให้โดเมนที่เราสนใจเป็นเซตของผู้ที่มี CMU account \enskip หากเราทราบว่า ถ้ามีนักศึกษามหาวิทยาลัยเชียงใหม่ที่ไม่ได้เดินขึ้นดอย แล้วสำนักทะเบียนฯ จะปรับปรุงระบบลงทะเบียนใหม่ พิจารณาการให้เหตุผลต่อไปนี้
\begin{itemize}
\item หากเราทราบเพิ่มเติมว่า นักศึกษามหาวิทยาลัยเชียงใหม่ทุกคนเดินขึ้นดอย แล้วต้องการสรุปว่า สำนักทะเบียนฯ จะไม่ปรับปรุงระบบลงทะเบียนใหม่ จะไม่สมเหตุสมผล \enskip ก่อนอื่น นิยาม predicates ดังนี้
\begin{itemize}
\item $P(x)$: $x$ เป็นนักศึกษามหาวิทยาลัยเชียงใหม่
\item $Q(x)$: $x$ เดินขึ้นดอย
\item $R$: สำนักทะเบียนฯ ปรับปรุงระบบลงทะเบียนใหม่
\end{itemize}
ข้อมูลที่เรามีอยู่สามารถเขียนเป็นสัญลักษณ์ได้ดังนี้
\begin{itemize}
\item $(\exists x: P(x)\wedge\neg Q(x))\implies R$
\item $\forall x: P(x)\implies Q(x)$
\end{itemize}
หากจะพยายามสรุปว่า $\neg R$ จะไม่สมเหตุสมผล เนื่องจากไม่มีเงื่อนไขที่กำหนดให้ $R$ เป็นเท็จได้ \enskip ทั้งนี้ สังเกตว่า $\forall x: P(x)\implies Q(x)$ เป็นนิเสธของ $\exists x: P(x)\wedge\neg Q(x)$ แต่การที่เงื่อนไขตั้งต้นของ implication เป็นเท็จ ไม่ได้บังคับให้ผลลัพธ์ต้องเป็นจริงหรือเท็จอย่างใดอย่างหนึ่ง \enskip ในที่นี้ หากนักศึกษาทุกคนเดินขึ้นดอย แล้วสำนักทะเบียนฯ ก็ยังสามารถปรับปรุงระบบลงทะเบียนได้อยู่

\item หากเราทราบเพิ่มเติมว่า สำนักทะเบียนฯ ปรับปรุงระบบลงทะเบียนใหม่ ก็ยังไม่จำเป็นที่จะต้องมีนักศึกษาที่ไม่ได้เดินขึ้นดอย เพราะไม่มีเงื่อนไขบังคับว่าถ้าปรับปรุงระบบ อะไรจะต้องเกิดขึ้นบ้าง \enskip ในที่นี้ก็เช่นเดียวกัน หากนักศึกษาทุกคนเดินขึ้นดอย แล้วสำนักทะเบียนฯ ก็ยังสามารถปรับปรุงระบบลงทะเบียนได้อยู่

\item หากเราทราบเพิ่มเติมว่า สำนักทะเบียนฯ ไม่ปรับปรุงระบบลงทะเบียนใหม่ จะสรุปได้ว่า ไม่เป็นจริงที่ว่า มีนักศึกษาที่ไม่ได้เดินขึ้นดอย เพราะมีเงื่อนไขบังคับว่า ถ้ามีนักศึกษาที่ไม่เดินขึ้นดอย แล้วจะต้องปรับปรุงระบบ \enskip ดังนั้น นักศึกษาทุกคนต้องเดินขึ้นดอย ซึ่งแปลว่าหากน้องกอล์ฟเป็นนักศึกษา น้องกอล์ฟก็จะต้องเดินขึ้นดอยด้วย
\end{itemize}
\end{example}

\section{Proof of uniqueness}

\emph{การพิสูจน์ความเป็นหนึ่งเดียว} (proof of uniqueness) เป็นวิธีที่ใช้ในการยืนยันว่า ค่าใดๆ ที่ทำให้ประพจน์ที่เรากำลังสนใจนั้นเป็นจริง มีเพียงค่าเดียวเท่านั้นที่เป็นไปได้ \enskip วิธีการโดยคร่าวๆ คือ ให้สมมุติว่าค่าดังกล่าวที่เป็นไปได้สองค่า แล้วพยายามให้เหตุผลและสรุปว่าแท้ที่จริงแล้ว ค่าสองค่านี้จำเป็นต้องเท่ากัน \enskip อีกวิธีหนึ่งที่สามารถทำได้ คือใช้การพิสูจน์โดยข้อขัดแย้ง โดยเริ่มจากการตั้งสมมุติฐานว่ามีสองค่าที่แตกต่างกัน ที่ทำให้ประพจน์เป็นจริง แล้วพยายามหาข้อขัดแย้ง ซึ่งจะทำให้เราสามารถสรุปได้เช่นกันว่า มีค่าที่ทำให้ประพจน์เป็นจริงที่แตกต่างกันไม่ได้

\begin{theorem}
$\exists! x\in\mathbb{Z}\ \forall y\in\mathbb{Z}: xy=x$
\begin{pf}
สมมุติว่ามีค่า $x$ ข้างต้นที่เป็นไปได้สองค่า โดยเรียกว่า $x_1,x_2\in\mathbb{Z}$ ซึ่งทำให้ $x_1y=x_1$ และ $x_2y=x_2$ เมื่อ $y$ เป็นจำนวนเต็มใดๆ \enskip จะได้ว่า $x_1y-x_1=0$ และ $x_2y-x_2=0$ \enskip ดังนั้น $x_1y-x-1=x_2y-x_2$ กล่าวคือ $x_1(y-1)=x_2(y-1)$ ซึ่งแสดงว่า $(x_1-x_2)(y-1)=0$ จึงสรุปได้ว่า $x_1=x_2$ หรือ $y=1$ \enskip อย่างไรก็ดี เนื่องจาก $y$ จะเป็นจำนวนเต็มใดๆ ก็ได้ ไม่จำเป็นต้องเป็น 1 เสมอไป จึงจำเป็นที่ $x_1$ ต้องเท่ากับ $x_2$ \enskip นั่นคือ มี $x$ เพียงตัวเดียวที่คูณกับจำนวนเต็มใดๆ แล้วได้ $x$
\end{pf}
ในบทพิสูจน์ข้างต้นนั้น เราไม่ได้ทำการพิสูจน์ว่ามี $x$ ที่ว่าอยู่จริง แต่เราเพียงแค่ให้เหตุผลว่า จะมี $x$ ที่มีคุณสมบัติดังกล่าวเกินกว่า 1 ค่าไม่ได้ \enskip หากจะทำการพิสูจน์ว่า $x$ นี้มีอยู่จริง สามารถให้เหตุผลเพิ่มเติมได้ว่า $x=0$ เป็นค่าที่ตรงตามคุณสมบัติดังกล่าว
\end{theorem}
