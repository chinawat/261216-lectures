\chapter{Collections}

บ่อยครั้งในวิชาคณิตศาสตร์ เราต้องสร้างแบบจำลองต่างๆ โดยแบบจำลองแต่ละแบบมักจะมีคุณสมบัติที่แตกต่างกันไป \enskip ในหัวข้อนี้ จะกล่าวถึงแบบจำลองที่ใช้กันอย่างแพร่หลาย รวมทั้งแจกแจงคุณสมบัติของแบบจำลองแต่ละชนิด เพื่อที่เราจะได้เลือกใช้ให้เหมาะสมกับปัญหาที่เราพยายามแก้ไข

เมื่อใดก็ตามที่เราจำเป็นต้องจัดกลุ่มสิ่งของให้เป็นหมวดหมู่ (collection) ผลลัพธ์ที่ได้นั้นมักจะมีคุณสมบัติอยู่สองแบบที่เราจะให้ความสนใจ ได้แก่
\begin{itemize}[]
\item สมาชิกใน collection นั้นซ้ำกันได้หรือไม่
\item ลำดับของสมาชิกใน collection นั้นมีความสำคัญหรือไม่
\end{itemize}
คุณสมบัติข้างต้น ทำให้เราจำแนก collections ได้เป็น 4 ชนิดใหญ่ๆ ดังนี้
\begin{itemize}[]
\item sequences (หรือ lists) เป็น collections ที่มีสมาชิกซ้ำได้ และลำดับของสมาชิกมีความสำคัญ
\item multisets (หรือ bags) เป็น collections ที่มีสมาชิกซ้ำได้ แต่ลำดับของสมาชิกนั้นไม่สำคัญ
\item sorted sets เป็น collections ที่สมาชิกซ้ำกันไม่ได้ แต่ลำดับของสมาชิกมีความสำคัญ \enskip sorted sets นั้นมักจะไม่พบโดยทั่วไปในวิชาคณิตศาสตร์เชิงทฤษฎี แต่อาจมีความจำเป็นต้องใช้ในการแก้ปัญหาคอมพิวเตอร์ \enskip ด้วยเหตุนี้ ภาษาโปรแกรมบางภาษาจึงสร้างโครงสร้างข้อมูลที่มีคุณสมบัตินี้เพื่ออำนวยความสะดวกให้แก่ผู้พัฒนาซอฟต์แวร์
\item sets เป็น collections ที่สมาชิกซ้ำกันไม่ได้ และลำดับของสมาชิกนั้นไม่สำคัญ
\end{itemize}
Sets นั้นเป็นแบบจำลองพื้นฐานที่พบได้โดยทั่วไป จึงจะกล่าวถึงนิยามและทฤษฎีที่เกี่ยวข้อง รวมไปถึงตัวอย่างการใช้งานโดยละเอียดเป็นอันดับถัดไป

\section{Sets}

\emph{เซต} (set) เป็นกลุ่มของสิ่งต่างๆ ที่
\begin{itemize}
\item ไม่นับซ้ำ (repetition free)
\item ไม่มีลำดับ (unordered)
\end{itemize}
กล่าวคือ สิ่งใดสิ่งหนึ่งจะอยู่ในเซตได้เพียงครั้งเดียว หรือไม่อยู่ในเซตนั้นๆ \enskip นอกจากนี้ ลำดับของสิ่งต่างๆ ในเซตไม่สำคัญ
%
\begin{example}
$\set{x\in\mathbb{Z}: 2\mid x}$ คือเซตที่มีสมาชิกเป็นเลขคู่
\end{example}

\begin{definition}
หาก $x$ อยู่ในเซต $A$ จะเรียก $x$ ว่า\emph{สมาชิก} (element) ของ $A$ (denoted $x\in A$)
\end{definition}
%
\begin{example}
$0\in\mathbb{N}$, $-1\notin\mathbb{N}$, $\frac{1}{2}\notin\mathbb{Z}$, $\sqrt{2}\notin\mathbb{Q}$
\end{example}

\begin{definition}
$A\cup B\triangleq\set{x\mid x\in A\vee x\in B}$
\end{definition}
เขียนเป็น Venn diagram ได้ดังรูปที่~\ref{fig:union}
%
\begin{figure}
\centering
\begin{tikzpicture}[scale=0.8]
    \draw[outline] \vennRect;
    \draw[filled] \vennCircA \vennCircB;
    \node[anchor=south] at (current bounding box.north) {$A \cup B$};
\end{tikzpicture}
\caption{Venn diagram for set union}
\label{fig:union}
\end{figure}

\begin{definition}
$A\cap B\triangleq\set{x\mid x\in A\wedge x\in B}$
\end{definition}
เขียนเป็น Venn diagram ได้ดังรูปที่~\ref{fig:intersection}
%
\begin{figure}
\centering
\begin{tikzpicture}[scale=0.8]
    \draw[outline] \vennRect;
    \begin{scope}
        \clip \vennCircA;
        \fill[filled] \vennCircB;
    \end{scope}
    \draw[outline] \vennCircA \vennCircB;
    \node[anchor=south] at (current bounding box.north) {$A \cap B$};
\end{tikzpicture}
\caption{Venn diagram for set intersection}
\label{fig:intersection}
\end{figure}

\begin{definition}
$A-B\triangleq\set{x\mid x\in A\wedge x\notin B}$
\end{definition}
เขียนเป็น Venn diagram ได้ดังรูปที่~\ref{fig:difference}
%
\begin{figure}
\centering
\begin{tikzpicture}[scale=0.8]
    \draw[outline] \vennRect;
    \begin{scope}
        \clip \vennCircA;
        \draw[filled, even odd rule] \vennCircA \vennCircB;
    \end{scope}
    \draw[outline] \vennCircA \vennCircB;
    \node[anchor=south] at (current bounding box.north) {$A-B$};
\end{tikzpicture}
\caption{Venn diagram for set difference}
\label{fig:difference}
\end{figure}

\begin{definition}
ให้ $D$ เป็นโดเมนที่เราสนใจ \enskip $\overline{A}\triangleq D-A$
\end{definition}
เขียนเป็น Venn diagram ได้ดังรูปที่~\ref{fig:complement}
%
\begin{figure}
\centering
\begin{tikzpicture}[scale=0.8]
    \draw[outline] \vennRect;
    \begin{scope}
        \clip \vennRect;
        \draw[filled, even odd rule] \vennRect \vennCircA;
    \end{scope}
    \node[anchor=south] at (current bounding box.north) {$\overline{A}$};
\end{tikzpicture}
\caption{Venn diagram for set complement}
\label{fig:complement}
\end{figure}

ถัดไป จะกล่าวถึงความสัมพันธ์ระหว่างเซต

\begin{definition}
$A$ เป็น\emph{เซตย่อย} (subset) ของ $B$ [denoted $A\subseteq B$] หาก $\forall x: x\in A\implies x\in B$
\end{definition}
หากเขียนความสัมพันธ์ของเซต $A$ และ $B$ ในนิยามดังกล่าวเป็น\emph{แผนภาพเวนน์} (Venn diagram) จะได้ดังรูปที่~\ref{fig:subset}
\begin{figure}
\centering
\begin{tikzpicture}[scale=0.8]
\draw[outline] \vennRect;
\draw[outline] {{(0.5,0) circle (0.75cm)} node {$A$}};
\draw[outline] {{(0:1cm) circle (1.75cm)}};
\node at (2,0) {$B$};
\node[anchor=south] at (current bounding box.north) {$A \subseteq B$};
\end{tikzpicture}
\caption{Venn diagram for subset}
\label{fig:subset}
\end{figure}
%
\begin{example}
$\mathbb{N}\subseteq\mathbb{Q}$
\end{example}
\begin{example}
ให้ $A=\set{1,2,\set{3,4}}$ จะได้ว่า
\begin{itemize}
\item $\set{1,2}\subseteq A$
\item $\set{3,4}\in A$
\item $\set{\set{3,4}}\subseteq A$
\item $\emptyset\subseteq A$ \enskip ทั้งนี้ เนื่องจาก $\forall x: x\in\emptyset\implies x\in A$ เนื่องจาก $x\in\emptyset$ นั้นเป็นจริงโดยปริยาย
\end{itemize}
\end{example}

\begin{theorem}
ให้ $A=\set{x\in\mathbb{Z}: 6\mid x}$ และ $B=\set{x\in\mathbb{Z}: 2\mid x}$ จะได้ว่า $A\subseteq B$
\begin{pf}
ต้องพิสูจน์ว่า $\forall x: x\in A\implies x\in B$ \enskip สมมุติว่า $x\in A$ กล่าวคือ $6\mid x$ แสดงว่ามี $x\in\mathbb{Z}$ ที่ทำให้ $x=6y$ นั่นคือ $x=2(3y)$ แต่นั่นแปลว่า $2\mid x$ จึงสรุปได้ว่า $x\in B$

ดังนั้น $A\subseteq B$ ตามที่ต้องการ
\end{pf}
\end{theorem}

\begin{theorem}
ให้ $A=\set{x\in\mathbb{Z}\mid x=y+z\ \textup{โดยที่ $y,z$ เป็นเลขคู่}}$ และ $B=\set{x\in\mathbb{Z}\mid x\ \textup{เป็นเลขคู่}}$ จะได้ว่า $A\subseteq B$
\begin{pf}
ให้ $x\in A$ นั่นคือ $x=y+z$ (ตามนิยามของ $A$) โดยที่ $y,z$ เป็นเลขคู่ \enskip จาก Theorem~2 ใน Lecture~3 จะได้ว่า $x$ เป็นเลขคู่ด้วย นั่นคือ $x\in B$ (ตามนิยามของ $B$) \enskip ดังนั้น เนื่องจากเราได้พิสูจน์ว่า $x\in B$ จากสมมุติฐานว่า $x\in A$ จะได้ว่า $A\subseteq B$ ตามที่ต้องการ
\end{pf}
\end{theorem}

\begin{theorem}
ให้ $a,b\in\mathbb{Z}$ \enskip ให้ $A=\set{x\in\mathbb{Z}:a\mid x}$ และ $B=\set{x\in\mathbb{Z}:b\mid x}$ จะได้ว่า \[A\subseteq B\iff b\mid a\]
\begin{pf}
($\Rightarrow$) เนื่องจาก $A\subseteq B$ จะได้ว่า $\forall x\in A: x\in B$ \enskip นอกจากนี้ $a\in A$ เพราะ $a\mid a$ \enskip ดังนั้น $a\in B$ เพราะ $A\subseteq B$ \enskip นั่นคือ $b\mid a$ ตามนิยามของเซต $B$

ในที่นี้ $b\mid a$ เป็น\emph{เงื่อนไขที่จำเป็น} (necessary condition) ของ $A\subseteq B$ กล่าวคือ หาก $A\subseteq B$ แล้ว $b\mid a$ อย่างแน่นอน

($\Leftarrow$) เนื่องจาก $b\mid a$ จะได้ว่า $a=bn$ โดยที่ $n\in\mathbb{Z}$ \enskip ให้ $x\in A$ กล่าวคือ $a\mid x$ จะได้ว่า $x=ak$ โดยที่ $k\in\mathbb{Z}$ \enskip ดังนั้น $x=bnk$ นั่นคือ $b\mid x$ ซึ่งแปลว่า $x\in B$ ตามนิยามของ $B$ \enskip เพราะฉะนั้น $A\subseteq B$

ในที่นี้ $b\mid a$ เป็น\emph{เงื่อนไขที่เพียงพอ} (sufficient condition) ของ $A\subseteq B$ กล่าวคือ แค่เพียง $b\mid a$ ก็พอที่ทำให้ $A\subseteq B$
\end{pf}
\end{theorem}

\begin{definition}[set equality]
$A=B$ หาก $\forall x: x\in A\iff x\in B$
\end{definition}
%
\begin{theorem}
$A=B$ iff $A\subseteq B$ และ $B\subseteq A$
\begin{pf}
($\Rightarrow$) ให้ $A=B$ กล่าวคือ $\forall x: x\in A\implies x\in B$ ต้องพิสูจน์ว่า $A\subseteq B$ และ $B\subseteq A$

พิจารณา $x$ ใดๆ \enskip เนื่องจาก $x\in A\iff x\in B$ (จาก $A=B$) จะได้ว่า
\begin{itemize}
\item $x\in A\implies x\in B$ นั่นคือ $A\subseteq B$
\item $x\in B\implies x\in A$ นั่นคือ $B\subseteq A$
\end{itemize}
ดังนั้น $A\subseteq B$ และ $B\subseteq A$ ตามที่ต้องการ

($\Leftarrow$) สมมุติว่า $A\subseteq B$ และ $B\subseteq A$ ต้องพิสูจน์ว่า $A=B$

พิจารณา $x$ ใดๆ ต้องพิสูจน์ว่า $x\in A\iff x\in B$\enskip เนื่องจาก
\begin{itemize}
\item $x\in A\implies x\in B$ (จาก $A\subseteq B$)
\item $x\in B\implies x\in A$ (จาก $B\subseteq A$)
\end{itemize}
จะได้ว่า $x\in A\iff x\in B$ กล่าวคือ $A=B$ นั่นเอง ตามที่ต้องการ
\end{pf}
\end{theorem}

\subsection{Connections between logic and sets}

ตัวดำเนินการเกี่ยวกับเซต และความสัมพันธ์ระหว่างเซต สามารถเชื่อมโยงกับตัวดำเนินการทางตรรกะได้ดังนี้
\begin{itemize}[]
\item ตัวดำเนินการ and ($\wedge$) นั้นเทียบเคียงได้กับตัวดำเนินการ intersection ($\cap$) เนื่องจากประพจน์ทั้งสองต้องเป็นจริง ประพจน์ผลลัพธ์จึงจะเป็นจริง และสมาชิกต้องอยู่ในทั้งสองเซต สมาชิกดังกล่าวจึงจะอยู่ใน intersection
\item ตัวดำเนินการ or ($\vee$) นั้นเทียบเคียงได้กับตัวดำเนินการ union ($\cup$) เนื่องจากประพจน์เพียงตัวใดตัวหนึ่งต้องเป็นจริง ประพจน์ผลลัพธ์จึงจะเป็นจริง และสมาชิกต้องอยู่ในเซตใดเซตหนึ่ง สมาชิกดังกล่าวจึงจะอยู่ใน union
\item ตัวดำเนินการ not ($\neg$) นั้นเทียบเคียงได้กับตัวดำเนินการ complement ($\overline{\cdot}$) เนื่องจากประพจน์ตั้งต้นต้องเป็นจริง ประพจน์ผลลัพธ์จึงจะเป็นเท็จ และสมาชิกต้องอยู่ในเซตตั้งต้น สมาชิกดังกล่าวจึงจะไม่อยู่ใน complement
\end{itemize}
จะเห็นว่า ตัวดำเนินการทั้งสามคู่ที่กล่าวมาข้างต้นนั้น รับข้อมูลนำเข้าเป็นประพจน์ หรือเซต และให้ผลลัพธ์เป็นประพจน์ หรือเซต เช่นเดียวกับข้อมูลนำเข้า
\begin{itemize}[]
\item ตัวดำเนินการ implication ($\implies$) นั้นเทียบเคียงได้กับความสัมพันธ์ subset ($\subseteq$) เนื่องจากประพจน์ผลลัพธ์จะเป็นจริงหากประพจน์ซ้ายเป็นจริงและประพจน์ขวาเป็นจริง หรือหากประพจน์ขวาเป็นเท็จ เช่นเดียวกับการที่คุณสมบัติ subset จะเป็นจริงหากสมาชิกอยู่ในเซตหน้าและเซตหลัง หรือสมาชิกไม่อยู่ในเซตหน้า
\item ตัวดำเนินการ iff ($\iff$) นั้นเทียบเคียงได้กับความสัมพันธ์ equality ($=$) เนื่องจากประพจน์สองตัวจะสมมูลกัน หากมีค่าความจริงเหมือนกัน เช่นเดียวกับเซตสองเซตที่เท่ากัน หากมีสมาชิกเหมือนกัน
\end{itemize}
จะเห็นว่า ความสัมพันธ์ทั้งสองคู่หลังนี้ รับข้อมูลนำเข้าเป็นประพจน์ หรือเซต แต่ให้ผลลัพธ์เป็นประพจน์เสมอ (นั่นคือ ผลลัพธ์เป็นจริงหรือเท็จ)

ความเชื่อมโยงดังที่ได้กล่าวมา สามารถสรุปได้ดังตารางต่อไปนี้
\begin{center}
\begin{tabular}{cc||cc}
\multicolumn{2}{c||}{\bf logic} & \multicolumn{2}{c}{\bf sets} \\ \hline
ชนิดตัวดำเนินการ & logical operator & set operator & ชนิดตัวดำเนินการ \\ \hline\hline
\ldelim\{{2}{4cm}[$\mathrm{Prop}\times\mathrm{Prop}\to\mathrm{Prop}$ ] & $\wedge$ & $\cap$ & \rdelim\}{2}{3.5cm}[ $\mathrm{Set}\times\mathrm{Set}\to\mathrm{Set}$] \\
& $\vee$ & $\cup$ & \\
$\mathrm{Prop}\to\mathrm{Prop}$ & $\neg$ & $\overline{\cdot}$ & $\mathrm{Set}\to\mathrm{Set}$ \\ \hline
\ldelim\{{2}{4cm}[$\mathrm{Prop}\times\mathrm{Prop}\to\mathrm{Prop}$ ] & $\implies$ & $\subseteq$ & \rdelim\}{2}{3.5cm}[ $\mathrm{Set}\times\mathrm{Set}\to\mathrm{Prop}$] \\
& $\iff$ & $=$ &
\end{tabular}
\end{center}

\subsection{Power sets}

\begin{definition}
ให้ $A$ เป็นเซต \enskip \emph{เซตกำลัง} (power set) ของ $A$ (denoted $\pow(A)$) คือเซต $\set{X\mid X\subseteq A}$
\end{definition}
บางครั้ง เราอาจจะเห็น power set ของ $A$ ที่เขียนด้วย $\mathcal{P}(A)$ หรือ $2^A$
%
\begin{example}
$\pow(\set{1,2})=\set{\emptyset,\set{1},\set{2},\set{1,2}}$
\end{example}
%
\begin{theorem}
$x\in A\iff\set{x}\in\pow(A)$
\begin{pf}
จากนิยามของ $\subseteq$ และ power set จะได้ว่า
$x\in A\iff\set{x}\subseteq A\iff\set{x}\in\pow(A)$
\end{pf}
\end{theorem}
ทั้งนี้ ในบทพิสูจน์ข้างต้น $x\in A\iff\set{x}\subseteq A$ อาจจะยังไม่ชัดเจนมากนัก จึงอาจเขียนเป็น\emph{บทตั้ง} (lemma) พร้อมบทพิสูจน์ที่ชัดเจนขึ้นได้ดังนี้
\begin{lemma}
$x\in A\iff\set{x}\subseteq A$
\begin{pf}
($\Rightarrow$): สมมุติว่า $x\in A$ ต้องพิสูจน์ว่า $\set{x}\subseteq A$ กล่าวคือ $\forall y: y\in\set{x}\implies y\in A$ \enskip ให้ $y\in\set{x}$ แสดงว่ามีค่าของ $y$ ที่เป็นไปได้เพียงตัวเดียวเท่านั้น กล่าวคือ $y=x$ \enskip ดังนั้น $y\in A$ เนื่องจาก $x\in A$ \qquad\yea

($\Leftarrow$): สมมุติว่า $\set{x}\subseteq{A}$ ต้องพิสูจน์ว่า $x\in A$ \enskip เนื่องจาก $\set{x}\subseteq A$ แสดงว่า $\forall y: y\in\set{x}\implies y\in A$ \enskip หากเราเลือกให้ $y$ มีค่าเป็น $x$ จะเห็นว่า $y\in\set{x}$ \enskip ดังนั้น $x\in A$ ตามต้องการ \qquad\yea
\end{pf}
\end{lemma}

\begin{theorem}
ให้ $A$ และ $B$ เป็นเซต \enskip $\pow(A)\cup\pow(B)\subseteq\pow(A\cup B)$
\begin{pf}
จากนิยามของ $\subseteq$ หากให้ $x\in\pow(A)\cup\pow(B)$ เราต้องพิสูจน์ว่า $x\in\pow(A\cup B)$ \enskip เนื่องจาก $x\in\pow(A)\cup\pow(B)$ แสดงว่า $x\in\pow(A)$ หรือ $x\in\pow(B)$ ตามนิยามของ $\cup$ \enskip พิสูจน์โดยแยกกรณีดังนี้
\begin{enumerate}[]
\item\label{pf:powset-union-subset-l} $x\in\pow(A)$: จะได้ว่า $x\subseteq A$ (ตามนิยามของ power set) \enskip ดังนั้น $x\subseteq A\cup B$
\item $x\in\pow(B)$: ในกรณีนี้ เราสามารถใช้เหตุผลเช่นเดียวกับในกรณีที่~\ref{pf:powset-union-subset-l} โดยสลับบทบาทของ $A$ และ $B$ \enskip จึงสรุปได้ว่า $x\in\pow(A\cup B)$ เช่นกัน
\end{enumerate}
ในทั้งสองกรณี เราสามารถสรุปได้ว่า $x\in\pow(A\cup B)$ \enskip ดังนั้น $\pow(A)\cup\pow(B)\subseteq\pow(A\cup B)$ ตามที่เราต้องการ
\end{pf}
\end{theorem}

\section{Lists}
\emph{รายการ} (lists) เป็น collections อีกชนิดหนึ่งที่มีสมบัติดังนี้
\begin{itemize}[]
\item สมาชิกซ้ำได้ (repetitions allowed)
\item ลำดับสำคัญ (ordered)
\end{itemize}
ในส่วนนี้ เราจะพิจารณาเฉพาะ lists ที่กำหนดความยาวแน่นอนก่อน \enskip Lists ในลักษณะนี้เรียกว่า \emph{tuples} ซึ่งสามารถสร้างขึ้นได้โดยใช้ Cartesian product
\begin{definition}
ให้ $A$ และ $B$ เป็นเซต \enskip \emph{ผลคูณคาร์ทีเซียน} (Cartesian product) ของ $A$ และ $B$ [denoted $A\times B$] คือเซตของ\emph{คู่อันดับ} (ordered pairs) ทั้งหมด โดยที่ตัวหน้าของคู่อันดับเป็นสมาชิกของ $A$ และตัวหลังของคู่อันดับเป็นสมาชิกของ $B$ กล่าวคือ
\[A\times B\triangleq\set{(a,b)\mid a\in A\wedge b\in B}\]
\end{definition}
%
\begin{example}\label{ex:Cartesian}
ให้ $A=\set{\str{d},\str{p}}$ และ $B=\set{\str{c},\str{x}}$ จะได้ว่า $A\times B=\set{(\str{d},\str{c}),(\str{d},\str{x}),(\str{p},\str{c}),(\str{p},\str{x})}$ และ $B\times A=\set{(\str{c},\str{d}),(\str{c},\str{p}),(\str{x},\str{d}),(\str{x},\str{p})}$
\end{example}
จะเห็นว่า $A\times B\neq B\times A$ กล่าวคือ Cartesian product ไม่มีสมบัติการสลับที่

เมื่อเราสร้าง ordered pairs ได้แล้ว เราสามารถสร้าง tuples ได้โดยใช้ Cartesian products ซ้ำกันหลายๆ ครั้ง ดังนี้ \enskip ให้ $A$, $B$, และ $C$ เป็นเซต และเราต้องการสร้าง tuples ที่มีความยาว 3 โดยตำแหน่งแรกเป็นสมาชิกของ $A$ ตำแหน่งที่สองเป็นสมาชิกของ $B$ และตำแหน่งสุดท้ายเป็นสมาชิกของ $C$ \enskip เริ่มแรก ให้หา Cartesian product $A\times B$ ก่อน จากนั้นนำผลลัพธ์ที่ได้ไปทำ Cartesian product กับ $C$ อีกครั้งหนึ่ง ได้ผลลัพธ์ดังนี้
\[(A\times B)\times C =\set{(x,c)\mid x\in A\times B\wedge c\in C}\]
แต่เนื่องจาก $x\in A\times B$ จะเห็นว่า $x$ นั้นต้องอยู่ในรูปคู่อันดับ $(a,b)$ โดยที่ $a\in A$ และ $b\in B$ กล่าวคือ
\[(A\times B)\times C =\set{((a,b),c)\mid (a\in A\wedge b\in B)\wedge c\in C}\]
อย่างไรก็ดี เราสามารถลบวงเล็บชั้นในที่ล้อมรอบ $a$ และ $b$ ออกได้โดยไม่เสียตำแหน่งของ $a$ และ $b$ ในผลลัพธ์ \enskip นอกจากนี้ Cartesian product นั้นมีสมบัติการเปลี่ยนหมู่ กล่าวคือ
\[(A\times B)\times C = A\times B\times C = A\times (B\times C)\]
ดังนั้น เราจึงเขียนเซตผลลัพธ์ใหม่ได้เป็น
\[A\times B\times C = \set{(a,b,c)\mid a\in A\wedge b\in B\wedge c\in C}\]
ซึ่งทำให้เราได้ tuples ที่มีความยาวเป็น 3 \enskip หากต้องการ tuples ที่มีความยาวมากกว่านี้ สามารถใช้วิธีเดียวกัน โดยเพิ่มจำนวนเซตที่นำมา Cartesian product กัน

ทั้งนี้ หากเราต้องการสร้าง $n$-tuples (tuples ที่มีความยาว $n$) โดยที่สมาชิกในแต่ละตำแหน่งนั้นมาจากเซต $A$ ทั้งหมด เราสามารถเขียนสัญกรณ์ (notation) ได้เป็น $A^n\triangleq\underbrace{A\times A\times\cdots\times A}_{\textrm{$n$ ตัว}}$

\section{Relations}

\emph{ความสัมพันธ์} (relation) เชื่อมโยงสิ่งสองสิ่งเข้าด้วยกัน
%
\begin{example}
$a$ กับ $b$ สัมพันธ์กันถ้า $a$ แต่งงานกับ $b$
\end{example}
\begin{example}
$a$ กับ $b$ สัมพันธ์กันถ้า $a$ ชอบ $b$
\end{example}
%
จะเห็นว่า ลำดับก่อนหลังในความสัมพันธ์มีความสำคัญ

\begin{definition}
ให้ $A,B$ เป็นเซต \enskip \emph{ความสัมพันธ์จาก $A$ ไป $B$} (relation from $A$ to $B$) คือเซต $R\subseteq A\times B$

เรียก $A$ ว่า\emph{โดเมน} (domain) และ $B$ ว่า\emph{โดเมนร่วมเกี่ยว} (codomain)\footnote{ไม่ใช้คำว่า range เนื่องจากอาจจะทำให้สับสนกับ image ได้} ของ $R$
\end{definition}
%
\begin{example}\label{ex:rel-a-b}
ให้ $A=\set{\str{d},\str{p}}$ และ $B=\set{\str{c},\str{x}}$ (จาก Example~\ref{ex:Cartesian}) \enskip $R=\set{(\str{d},\str{c}),(\str{p},\str{x})}\subseteq A\times B$ 
\end{example}
เมื่อมีความสัมพันธ์ $R$ from $A$ to $B$ ที่สามารถเขียนแจกแจงสมาชิกในรูปของคู่อันดับได้ เราสามารถวาด\emph{กราฟ} (graph) ของ $R$ ได้ โดยแจกแจงสมาชิกของ domain (เซต $A$) ไว้ด้านซ้าย และสมาชิกของ codomain (เซต $B$) ไว้ด้านขวา แล้วโยงลูกศรเชื่อมจาก $a$ ไป $b$ หาก $(a,b)\in R$ ดังตัวอย่างต่อไปนี้ ซึ่งแสดงกราฟความสัมพันธ์ของ $R$ ใน Example~\ref{ex:rel-a-b}
\begin{center}
\begin{tikzpicture}
\node (d) {\str{d}};
\node (p) [below=of d] {\str{p}};
\node (c) [right=of d] {\str{c}};
\node (x) [right=of p] {\str{x}};

\draw[arrow] (d) -- (c);
\draw[arrow] (p) -- (x);
\end{tikzpicture}
\end{center}
ในขณะเดียวกัน หากมีกราฟของความสัมพันธ์ใดๆ เราก็สามารถเขียนแจกแจงเป็นเซตได้เช่นกัน
%
\begin{example}
ให้ $A=\set{1,2,3,4}$ และ $B=\set{a,b,c,d}$ \enskip หากมีกราฟความสัมพันธ์ $R$ ดังนี้
\begin{center}
\begin{tikzpicture}
\node (A1) {$1$};
\node (A2) [below of=A1] {$2$};
\node (A3) [below of=A2] {$3$};
\node (A4) [below of=A3] {$4$};

\node (Ba) [right=1in of A1] {$a$};
\node (Bb) [below of=Ba] {$b$};
\node (Bc) [below of=Bb] {$c$};
\node (Bd) [below of=Bc] {$d$};

\path (A1) -- node[above=2ex] {$R$} (Ba);

\draw[->] (A1) -- (Ba);
\draw[->] (A1) -- (Bb);
\draw[->] (A3) -- (Bc);
\draw[->] (A4) -- (Bb);
\end{tikzpicture}
\end{center}
จะได้ว่า $R=\set{(1,a),(1,b),(3,c),(4,b)}$ \enskip นอกจากนี้ เราสามารถนับจำนวนลูกศรขาออกจากสมาชิกแต่ละตัวทางด้านซ้าย และจำนวนลูกศรขาเข้าไปยังสมาชิกแต่ละตัวทางด้านขวาได้ ตัวอย่างเช่น $1$ มี 2 ลูกศรออก, $2$ มี 0 ลูกศรออก, $b$ มี 2 ลูกศรเข้า, และ $c$ มี 1 ลูกศรเข้า
\end{example}

\begin{example}
ให้ $A$ เป็นเซตของนักศึกษา และ $B$ เป็นเซตของวิชา \enskip $R=\set{(a,b)\mid\text{$a$ ลงทะเบียนวิชา $b$}}$
\end{example}

\begin{example}
ให้ $A$ เป็นเซตผู้ชาย และ $B$ เป็นเซตของผู้หญิง \enskip $R=\set{(a,b)\mid\text{$a$ ชอบ $b$}}$
\end{example}

\begin{definition}
ให้ $A$ เป็นเซต \enskip \emph{ความสัมพันธ์บนเซต $A$} (relation on $A$) คือเซต $R\subseteq A\times A$
\end{definition}
%
\begin{example}
ให้ $A=\set{\str{d},\str{p}}$ (จาก Example~\ref{ex:Cartesian}) \enskip \[R=\set{(\str{d},\str{p}),(\str{p},\str{d})}\subseteq A\times A=\set{(\str{d},\str{d}),(\str{d},\str{p}),(\str{p},\str{d}),(\str{p},\str{p})}\]
\end{example}
%
\begin{example}
$\leq$ is a relation on $\mathbb{Z}$.

\[\leq\ =\set{(0,0),(0,1),(-1,0),(-2,-1),(2019,2562),\ldots}\]
\end{example}

\begin{example}
$=$, $<$, $>$, $\geq$, $\neq$ are relations on $\mathbb{Z}$, $\mathbb{Q}$, $\mathbb{R}$
\end{example}

\begin{example}
ให้ $A$ เป็นเซตของ \textit{homo sapiens sapiens} \enskip $R=\set{(a_1,a_2)\mid\text{$a_1$ ชอบ $a_2$}}$ is a relation on $A$
\end{example}

\begin{definition}
ให้ $R$ เป็น relation \enskip \emph{ความสัมพันธ์ผกผัน} (inverse) ของ $R$ (denoted $R^{-1}$) คือเซต \[R^{-1}\triangleq\set{(b,a)\mid(a,b)\in R}\]
\end{definition}
%
\begin{example}
ให้ $R=\set{(\str{d},\str{c}),(\str{p},\str{x})}$ จะได้ว่า $R^{-1}=\set{(\str{c},\str{d}),(\str{x},\str{p})}$
\end{example}

\begin{example}
ให้ $R=\set{(a_1,a_2)\mid\text{$a_1$ ชอบ $a_2$}}$ จะได้ว่า $R^{-1}=\set{(a_1,a_2)\mid\text{$a_2$ ชอบ $a_1$}}$
\end{example}

\begin{definition}
ให้ $R\subset A\times B$ และ $a\in A$ \enskip \emph{ภาพ} (image) ของ $a$ ภายใต้ $R$ (denoted $R(a)$) คือเซต \[R(a)\triangleq\set{b\in B\mid (a,b)\in R}\]
\end{definition}
\begin{definition}
ให้ $R\subset A\times B$ และ $A'\subseteq A$ \enskip \emph{image of $A'$ under $R$} (denoted $R(A')$) คือเซต
\begin{align*}
R(A')&\triangleq\set{b\in B\mid\exists a\in A': (a,b)\in R} \\
&=\bigcup_{a\in A'}{R(a)}
\end{align*}
\end{definition}
%
\begin{example}
ให้ $R=\set{(1,4),(1,5),(2,6),(3,6)}$ จะได้ว่า
\begin{itemize}
\item $R(1)=\set{4,5}$
\item $R(2)=\set{6}$
\item $R^{-1}(4)=\set{1}$
\item $R^{-1}(6)=\set{2,3}$
\item $R(\set{1,2})=\set{4,5,6}$
\item $R(\set{2,3})=\set{6}$
\item $R^{-1}(\set{4,5})=\set{1}$
\item $R^{-1}(\set{6})=\set{2,3}$
\item $R(A)=\set{4,5,6}$
\end{itemize}
\end{example}
จะเห็นว่า image ของสมาชิกหรือเซตย่อยของโดเมนนั้น ก็คือ subset ของ codomain ที่มีลูกศรชี้ไปหาจากสมาชิกหรือเซตย่อยที่เรากำลังพิจารณาอยู่ \enskip หากเซตย่อยที่เรากำลังพิจารณานั้นเป็นทั้งโดเมน จะได้ผลลัพธ์เป็นสมาชิกใน codomain ที่ปรากฎอยู่ในความสัมพันธ์ ซึ่งอาจจะไม่ครบทุกตัวก็ได้ \enskip ด้วยเหตุนี้ เราจึงหลีกเลี่ยงการใช้คำว่า range ในการอธิบายความสัมพันธ์ เนื่องจากเราอาจจะตีความว่า range เป็นสมาชิกที่เราสามารถสร้างลูกศรชี้ไปหาได้ (ซึ่งมีความหมายเดียวกับ codomain) หรืออาจจะตีความว่า range เป็นสมาชิกที่มีลูกศรในความสัมพันธ์ชี้ไปหาก็ได้ (ซึ่งมีความหมายเดียวกับ image)

\subsection{Properties of relations}

ใน Definitions~\ref{def:reflexive}--\ref{def:transitive} ให้ $R$ เป็น relation on $A$ \enskip เราสามารถนิยามคุณสมบัติของความสัมพันธ์แบบต่างๆ ได้ดังนี้

\begin{definition}\label{def:reflexive}
$R$ มีสมบัติ\emph{สะท้อน} (reflexive) หาก \[\forall x\in A: \R{x}{x}\]
\end{definition}
($\R{x}{x}$ เป็นอีกวิธีหนึ่งในการเขียน $(x,x)\in R$)

\begin{definition}
$R$ มีสมบัติ\emph{ไม่สะท้อน} (irreflexive) หาก \[\forall x\in A: \R[/\!\!\!R]{x}{x}\]
\end{definition}

\begin{definition}
$R$ มีสมบัติ\emph{สมมาตร} (symmetric) หาก \[\forall x,y\in A: \R{x}{y}\implies\R{y}{x}\]
\end{definition}

\begin{definition}
$R$ มีสมบัติ\emph{ปฏิสมมาตร} (antisymmetric) หาก \[\forall x,y\in A: \R{x}{y}\wedge\R{y}{x}\implies x=y\]
\end{definition}

\begin{definition}\label{def:transitive}
$R$ มีสมบัติ\emph{ถ่ายทอด} (transitive) หาก \[\forall x,y,z\in A: \R{x}{y}\wedge\R{y}{z}\implies\R{x}{z}\]
\end{definition}

\begin{example}
ความสัมพันธ์ต่างๆ ต่อไปนี้มีสมบัติดังปรากฏในตาราง
\begin{center}
\begin{longtable}{c|c||c|c|c|c|c}
$A$ & relation & \rotatedhdr{reflexive} & \rotatedhdr{irreflexive} & \rotatedhdr{symmetric} & \rotatedhdr{antisymmetric} & \rotatedhdr{transitive} \\ \hline\hline
\endhead
\multirow{3}{*}{$\mathbb{Z}$}
 & $=$ & \yea & \nay & \yea & \yea & \yea \\ \cline{2-7}
 & $\leq$ & \yea & \nay & \nay & \yea & \yea \\ \cline{2-7}
 & $<$ & \nay & \yea & \nay & \yea\footnote{เป็นจริงโดยปริยาย (vacuously true) เนื่องจากไม่มีจำนวนเต็ม $x$ และ $y$ ที่ $x<y$ และ $y<x$ ทั้งคู่} & \yea \\ \hline
\multirow{2}{*}{\it homo sapiens sapiens}
 & แต่งงาน\footnote{หาก $a$ แต่งงานกับ $b$ หมายถึง $a$ จดทะเบียนสมรสกับ $b$} & \nay & \yea & \yea & \nay & \nay \\ \cline{2-7}
 & ชอบ & \yea & \nay & \nay & \nay & \nay
\end{longtable}
\end{center}
\end{example}

ใน Definitions~\ref{def:function}--\ref{def:bijection} ให้ $R$ เป็น relation from $A$ to $B$ 

\begin{definition}\label{def:function}
$R$ เป็น\emph{ฟังก์ชัน} (function) หากสมาชิกทุกตัวใน $A$ มี $\leq 1$ ออก กล่าวคือ \[\forall a\in A: |R(a)|\leq 1\]
\end{definition}

\begin{definition}
$R$ นั้น\emph{ทุกส่วน} (total) หากสมาชิกทุกตัวใน $A$ มี $\geq 1$ ออก กล่าวคือ \[\forall a\in A: |R(a)|\geq 1\]
\end{definition}

\begin{definition}
$R$ นั้น\emph{หนึ่งต่อหนึ่ง}\footnote{หากใช้ภาษาไทย อาจจะสับสนกับ bijection ได้} (injection) หากสมาชิกทุกตัวใน $B$ มี $\leq 1$ เข้า กล่าวคือ \[\forall b\in B: |R^{-1}(b)|\leq 1\]
\end{definition}

\begin{definition}
$R$ นั้น\emph{ทั่วถึง} (surjection) หากสมาชิกทุกตัวใน $B$ มี $\geq 1$ เข้า กล่าวคือ \[\forall b\in B: |R^{-1}(b)|\geq 1\]
\end{definition}

\begin{definition}\label{def:bijection}
$R$ นั้น\emph{หนึ่งต่อหนึ่งทั่วถึง}\footnote{หากใช้ภาษาไทย อาจจะลืมว่าเป็น total function ได้} (bijection) หากสมาชิกทุกตัวใน $A$ มี $=1$ ออก และสมาชิกทุกตัวใน $B$ มี $=1$ เข้า กล่าวคือ \[\forall a\in A: |R(a)|= 1 \wedge \forall b\in B: |R^{-1}(b)|= 1\]
\end{definition}

\begin{example}
ความสัมพันธ์ต่างๆ ต่อไปนี้มีสมบัติดังปรากฏในตาราง
\begin{center}
\begin{longtable}{c|c|c||c|c|c|c|c}
$A$ & $B$ & relation & \rotatedhdr{function} & \rotatedhdr{total} & \rotatedhdr{injection} & \rotatedhdr{surjection} & \rotatedhdr{bijection} \\ \hline\hline
\endhead
\multirow{4}{0.6in}{\centering $\mathbb{R}$} & \multirow{4}{0.6in}{\centering $\mathbb{R}$}
 & $\R{x}{2x}$ & \yea & \yea & \yea & \yea & \yea \\ \cline{3-8}
 & & $\R{x}{\frac{1}{x}}$ & \yea & \nay\footnote{หากเป็น function ที่ไม่ total จะเรียกว่า\emph{ฟังก์ชันบางส่วน} (partial function)} & \yea & \nay & \nay \\ \cline{3-8}
 & & $\R{x^2}{x}$ & \nay & \nay & \yea & \yea & \nay \\ \cline{3-8}
 & & $\leq$ & \nay & \yea & \nay & \yea & \nay \\ \hline
\multicolumn{2}{c|}{\multirow{2}{*}{\it homo sapiens sapiens}}
 & แต่งงาน\footnote{สมบัติต่างๆ อาจขึ้นกับความเชื่อส่วนบุคคล โปรดใช้วิจารณญาณในการอ่าน} & \yea & \nay & \yea & \nay & \nay \\ \cline{3-8}
\multicolumn{2}{c|}{}
 & ชอบ & \nay & \nay & \nay & \nay & \nay
\end{longtable}
\end{center}
\end{example}

\subsection{Relational compositions}
เราสามารถนำความสัมพันธ์มาเกี่ยวโยงกันได้
\begin{definition}
ให้ $R$ เป็น relation from $A$ to $B$ และ $S$ เป็น relation from $B$ to $C$ \enskip \emph{ความสัมพันธ์ประกอบ} (composition) ของ $R$ และ $S$ (denoted $S\circ R$) คือเซต \[S\circ R\triangleq\set{(a,c)\mid\exists b\in B: \R{a}{b} \wedge \R[S]{b}{c}}\]
\end{definition}
นั่นคือ เราสามารถสร้างความสัมพันธ์ที๋มีลูกศรโยงจากสมาชิกในเซต $A$ ไปยังสมาชิกในเซต $C$ ได้ หากมีสมาชิกในเซต $B$ ที่เป็นตัวกลางเชื่อมสมาชิกทั้งสองเซตก่อนหน้านี้ กล่าวคือ มีลูกศรโยงจากสมาชิกในเซต $A$ ไปยังสมาชิกในเซต $B$ และมีลูกศรโยงจากสมาชิกตัวนี้ในเซต $B$ ไปยังสมาชิกในเซต $C$
%
\begin{example}
นิยาม $R$ และ $S$ ตามกราฟต่อไปนี้
\begin{center}
\begin{tikzpicture}
\node (A1) {$\pi$};
\node (A2) [below of=A1] {$\sqrt{2}$};

\node (B1) [right=1in of A1] {$1$};
\node (B2) [below of=B1] {$2$};
\node (B3) [below of=B2] {$3$};
\node (B4) [below of=B3] {$4$};

\node (Ca) [right=1in of B1] {$a$};
\node (Cb) [below of=Ca] {$b$};
\node (Cc) [below of=Cb] {$c$};
\node (Cd) [below of=Cc] {$d$};

\path (A1) -- node[above=2ex] {$R$} (B1) -- node[above=2ex] {$S$} (Ca);

\draw[->] (A1) -- (B2);
\draw[->] (A1) -- (B4);
\draw[->] (A2) -- (B1);

\draw[->] (B1) -- (Ca);
\draw[->] (B1) -- (Cb);
\draw[->] (B3) -- (Cc);
\draw[->] (B4) -- (Cb);
\end{tikzpicture}
\end{center}
จะได้ว่า $S\circ R=\set{(\sqrt{2},a),(\sqrt{2},b),(\pi,b)}$
\end{example}
%
\begin{example}
ให้ $f(x)=x+1$ และ $g(x)=x^2$ จะได้ว่า
\begin{align*}
(g\circ f)(x)
&= g(f(x)) \\
&= g(x+1) \\
&= (x+1)^2
\end{align*}
หากเขียนในรูปความสัมพันธ์ จะได้ว่า $\R[f]{x}{(x+1)}$ และ $\R[g]{x}{x^2}$ กล่าวคือ $\R[g]{(x+1)}{(x+1)^2}$ \enskip ดังนั้น \[\R[(g\circ f)]{x}{(x+1)^2}\]
\end{example}

\begin{example}
ให้ $A$ เป็นเซตของนักศึกษามหาวิทยาลัยเชียงใหม่, $B$ เป็นเซตของกระบวนวิชาในมหาวิทยาลัยเชียงใหม่, และ $C$ เป็นเซตของอาจารย์มหาวิทยาลัยเชียงใหม่ \enskip ให้ $R\subseteq A\times B$ เป็นความสัมพันธ์ระหว่างนักศึกษา $a$ กับกระบวนวิชา $b$ โดยที่ $\R{a}{b}$ หาก $a$ ลงทะเบียนเรียนวิชา $b$ ในภาคเรียนนี้, และ $S\subseteq B\times C$ เป็นความสัมพันธ์ระหว่างกระบวนวิชา $b$ กับอาจารย์ $c$ โดยที่ $\R[S]{b}{c}$ หาก $c$ สอนวิชา $b$ ในภาคเรียนนี้ \enskip จะได้ว่า
\begin{itemize}[]
\item $R(A)$ (image of $A$ under $R$) เป็นเซตของวิชาที่มีนักศึกษาลงทะเบียนในภาคเรียนนี้ (นั่นคือ วิชาใดมีลูกศรชี้มาจากนักศึกษาบ้างในความสัมพันธ์ $R$)
\item $S^{-1}(\text{ชิน})$ (image of ชิน under $S^{-1}$) เป็นเซตของวิชาที่ อ.ชินสอนในภาคเรียนนี้ (นั่นคือ วิชาใดมีลูกศรชี้ไปหา อ.ชินบ้างในความสัมพันธ์ $S$)
\item $S\circ R\subseteq A\times C$ เป็นความสัมพันธ์ที่เชื่อมโยงนักศึกษากับอาจารย์ โดยที่ $\R[S\circ R]{a}{c}$ หากนักศึกษา $a$ เรียนกับอาจารย์ $c$
\item $R\circ R^{-1}\subseteq B\times B$ เป็นความสัมพันธ์ที่เชื่อมโยงสองวิชาเข้าด้วยกัน โดยที่ $\R[R\circ R^{-1}]{b_1}{b_2}$ หากวิชา $b_1$ และ $b_2$ ไม่สามารถจัดสอบปลายภาคพร้อมกันได้ เนื่องจากมีนักศึกษาที่ลงทะเบียนวิชา $b_1$ และ $b_2$ ทั้งคู่ (หรือ $b_1$ และ $b_2$ นี้เป็นวิชาเดียวกัน)
\item $R^{-1}\circ R\subseteq A\times A$ เป็นความสัมพันธ์ที่เชื่อมโยงนักศึกษาสองค้นเข้าด้วยกัน โดยที่ $\R[R^{-1}\circ R]{a_1}{a_2}$ หากนักศึกษา $a_1$ และ $a_2$ ลงทะเบียนเรียนวิชาเดียวกัน (หรือ $a_1$ และ $a_2$ เป็นนักศึกษาคนเดียวกัน)
\end{itemize}
\end{example}

