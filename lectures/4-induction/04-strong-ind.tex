
\section{Strong induction}

\emph{การพิสูจน์โดยอุปนัยอย่างแรง} (proof by strong induction) เป็นอีกกลวิธีที่ใช้ในการพิสูจน์ว่า predicate $P(n)$ นั้นเป็นจริงสำหรับจำนวนนับ $n$ ทุกตัว โดยใช้ axiom คล้ายกับ induction

Axiom ที่จะใช้ในการพิสูจน์ มีดังนี้
\begin{axiom}[strong induction]
ให้ $P(n)$ เป็น predicate โดยที่ $n\in\mathbb{N}$ \enskip ถ้าสองเงื่อนไขต่อไปนี้เป็นจริง
\begin{enumerate}
\item $P(0)$ เป็นจริง และ
\item\label{ax:strong-ind} $\forall n\in\mathbb{N}:\left(P(0)\wedge P(1)\wedge\cdots\wedge P(n)\right)\implies P(n+1)$ เป็นจริง
\end{enumerate}
แล้ว $\forall n\in\mathbb{N}:P(n)$ จะเป็นจริงด้วย
\end{axiom}
จุดที่แตกต่างไปจาก induction axiom ก็คือสมมุติฐานในข้อที่~\ref{ax:strong-ind} ข้างต้น ซึ่งใน induction นั้นเราสามารถใช้ $P(n)$ เป็นสมมุติฐานได้เพียงอย่างเดียว แต่ใน strong induction นั้น เราสามารถใช้ $P(0),P(1),\ldots,P(n)$ เป็นสมมุติฐานได้ทั้งหมด อย่างไรก็ดี เราอาจจะไม่ได้ใช้สมมุติฐานทุกตัวที่เรามีจนครบในบทพิสูจน์ก็เป็นได้

ขั้นตอนการเขียนบทพิสูจน์โดยอุปนัยอย่างแรง เปลี่ยนแปลงไปจากขั้นตอนของ induction ดังนี้
\begin{itemize}
\item By strong induction.
\item \emph{ขั้นอุปนัย} (inductive step): ตั้งสมมุติฐานว่า $P(0),P(1),\ldots,P(n)$ เป็นจริง เพื่อใช้ในการพิสูจน์ว่า $P(n+1)$ เป็นจริง
\end{itemize}

\begin{theorem}
หากต้องการแบ่งกลุ่มให้มีสมาชิกกลุ่มละ 3 หรือ 4 คน จะมีวิธีแบ่งกลุ่มเสมอหากมีคนอย่างน้อย 6 คน
\begin{pf}
By strong induction.  ให้ $P(n)\triangleq$ มีวิธีแบ่งกลุ่มคน $n$ คน ให้มีสมาชิกกลุ่มละ 3 หรือ 4 คน
\begin{itemize}
\item {\bf Base case}: ($n=6$) \quad แบ่ง 6 คนเป็น 2 กลุ่ม กลุ่มละ 3 คน \quad\yea
\item {\bf Inductive step}: สมมุติว่ามีวิธีแบ่งกลุ่มคน $6,7,\ldots,n$ คน ต้องพิสูจน์ว่ามีวิธีแบ่งกลุ่มคน $n+1$ คน (เมื่อ $n\geq 6$)
\begin{itemize}
\item $n=6$: ต้องพิสูจน์ว่ามีวิธีแบ่งกลุ่มคน 7 คน (โดยเรารู้ว่ามีวิธีแบ่งกลุ่มคน 6 คน) \enskip แบ่งได้เป็น 2 กลุ่ม กลุ่มหนึ่งมี 3 คน อีกกลุ่มหนึ่งมี 4 คน (ในที่นี้ เราไม่ได้ใช้ I.H. แต่อย่างใด)
\item $n=7$: ต้องพิสูจน์ว่ามีวิธีแบ่งกลุ่มคน 8 คน (โดยเรารู้ว่ามีวิธีแบ่งกลุ่มคน $6,7$ คน) \enskip แบ่งได้เป็น 2 กลุ่ม กลุ่มละ 4 คน (ในที่นี้ เราไม่ได้ใช้ I.H. แต่อย่างใดเช่นกัน)
\item $n\geq 8$: ต้องพิสูจน์ว่าแบ่ง $n+1$ คนได้ (โดยรู้ว่าแบ่ง $6,7,\ldots,n$ คนได้) \enskip เนื่องจาก $n-2\geq 6$ จาก I.H. จะได้ว่า มีวิธีแบ่งกลุ่มคน $n-2$ คน \enskip ดังนั้น หากต้องการแบ่งกลุ่มคน $n+1$ คน ให้ดึง 3 คนออกมาก่อน จะเหลือ $n-2$ คน \enskip แบ่ง $n-2$ คนนี้ตาม I.H. แล้วจับ 3 คนที่ดึงออกมาอยู่กลุ่มเดียวกัน ก็จะได้วิธีแบ่งกลุ่มคน $n+1$ คน ตามที่ต้องการ \quad\yea
\end{itemize}
\end{itemize}
\end{pf}
ณ จุดนี้ อาจจะเกิดข้อสงสัยขึ้นว่า ทุกครั้งที่มีคนอย่างน้อย 8 คน เราดึงคนออกไป 3 คนเพื่อจัดเป็นกลุ่มทุกครั้ง ซึ่งดูเหมือนว่าจะไม่ค่อยมีกลุ่มที่มี 4 คนเลย อย่างไรก็ดี เราสามารถดึง 4 คนออกมาก่อนได้เช่นกัน แต่ถ้า $n=8$ เราจะยังไม่สามารถใช้ induction hypothesis ได้ เนื่องจาก $(n+1)-4=n-3$ นั้นยังน้อยเกินไป \enskip ดังนั้น หากต้องการดึง 4 คนออกมาก่อน จะต้องเขียนแยกกรณีที่ $n=8$ ออกมา กล่าวคือ ต้องแบ่งกรณีที่ $n\geq 8$ ข้างต้นออกเป็นดังต่อไปนี้
\begin{itemize}
\item $n=8$: ต้องพิสูจน์ว่ามีวิธีแบ่งกลุ่มคน 9 คน (โดยเรารู้ว่ามีวิธีแบ่งกลุ่มคน $6,7,8$ คน) \enskip แบ่งได้เป็น 3 กลุ่ม กลุ่มละ 3 คน (ในที่นี้ เราไม่ได้ใช้ I.H. แต่อย่างใดเช่นกัน)
\item $n\geq 9$: ต้องพิสูจน์ว่าแบ่ง $n+1$ คนได้ (โดยรู้ว่าแบ่ง $6,7,\ldots,n$ คนได้) \enskip เนื่องจาก $n-3\geq 6$ จาก I.H. จะได้ว่า มีวิธีแบ่งกลุ่มคน $n-3$ คน \enskip ดังนั้น หากต้องการแบ่งกลุ่มคน $n+1$ คน ให้ดึง 4 คนออกมาก่อน จะเหลือ $n-3$ คน \enskip แบ่ง $n-3$ คนนี้ตาม I.H. แล้วจับ 4 คนที่ดึงออกมาอยู่กลุ่มเดียวกัน ก็จะได้วิธีแบ่งกลุ่มคน $n+1$ คน ตามที่ต้องการ
\end{itemize}
\end{theorem}

\subsection{The unstacking game}

ในตัวอย่างนี้ จะพิจารณาเกมที่ให้แบ่งกองไม้สูง $n$ ชิ้น ออกเป็นสองกอง จากนั้น ในแต่ละกอง ให้แบ่งกองไม้ออกในลักษณะเดียวกันไปเรื่อยๆ จนกระทั่งไม่มีกองใดมีไม้มากกว่า 1 ชิ้นเลย

การคิดคะแนนในเกมนี้ จะคิดเมื่อมีการแบ่งกองไม้เกิดขึ้น โดยคะแนนในแต่ละตาคำนวณได้จากผลคูณของจำนวนชิ้นไม้ในแต่ละกองที่เพิ่งแบ่งออกมา และคะแนนรวมทั้งหมดคือผลรวมของคะแนนในแต่ละตา

ตัวอย่างเช่น หากเริ่มด้วยกองไม้ 4 ชิ้น และตาแรกแบ่งออกเป็นกองละ 2 ชิ้น จะได้ว่า คะแนนในตาแรกเป็น $2\cdot 2=4$ จากนั้น ในแต่ละกองที่แบ่งออกมาสามารถแบ่งเป็นกองละ 1 ชิ้นได้เท่านั้น โดยใช้ทั้งหมด 2 ตา จะได้ว่า ตาหนึ่งๆ ได้คะแนน $1\cdot 1=1$ ดังนั้น คะแนนรวมทั้งหมดในการเล่นวิธีนี้คือ $4+1+1=6$

\begin{theorem}
ทุกๆ กลยุทธ์ในการแบ่งกองไม้ $n$ ชิ้นจนหมด (เมื่อ $n\geq 1$) จะได้คะแนนเท่ากัน (เรียกคะแนนนี้ว่า $S(n)$)
\begin{pf}[Failed proof]
By strong induction.  ให้ $P(n)\triangleq$ ทุกๆ กลยุทธ์ในการแบ่งกองไม้ $n$ ชิ้นจนหมด จะได้คะแนน $S(n)$
\begin{itemize}
\item {\bf Base case}: ($n=1$) \quad วิธีการแบ่งกองไม้ 1 ชิ้น ย่อมมีวิธีเดียว คือ ไม่แบ่ง ดังนั้น ทุกๆ กลยุทธ์ก็ย่อมได้คะแนนเท่ากันคือ $S(0)$ \quad\yea
\item {\bf Inductive step}: สมมุติว่าทุกๆ กลยุทธ์ในการแบ่งกองไม้ $1$ ชิ้นจนหมด ได้คะแนนเท่ากันคือ $S(1)$; ทุกๆ กลยุทธ์ในการแบ่งกองไม้ $2$ ชิ้นจนหมด ได้คะแนนเท่ากันคือ $S(2)$; $\ldots$; และทุกๆ กลยุทธ์ในการแบ่งกองไม้ $n$ ชิ้นจนหมด ได้คะแนนเท่ากันคือ $S(n)$ \enskip ต้องพิสูจน์ว่า ทุกๆ กลยุทธ์ในการแบ่งกองไม้ $n+1$ ชิ้นจนหมด จะได้คะแนน $S(n+1)$

หากแบ่งกองไม้ $n+1$ ชิ้นเป็นสองกอง โดยที่กองหนึ่งมี $k$ ชิ้น จะได้ว่า อีกกองหนึ่งมี $n+1-k$ ชิ้น ดังนั้น ในตานี้ จะได้คะแนน $k(n+1-k)$ ทันที จากนั้น ต้องแบ่งกองไม้แต่ละกองไปเรื่อยๆ จนแบ่งต่อไม่ได้อีก \enskip By I.H., จะได้ว่า คะแนนรวมที่จะได้จากการแบ่งไม้ $k$ ชิ้น คือ $S(k)$ และคะแนนรวมที่จะได้จากการแบ่งไม้ $n+1-k$ ชิ้น คือ $S(n+1-k)$ ดังนั้น คะแนนรวมที่จะได้จากการแบ่งกองไม้ $n+1$ ชิ้นคือ \[S(n+1)=k(n+1-k)+S(k)+S(n+1-k)\] \ldots
\end{itemize}
\end{pf}

จะเห็นว่า เกิดการติดขัดในบทพิสูจน์ขึ้น เนื่องจากในแต่กลยุทธ์นั้น $k$ ไม่เท่ากัน เพราะเราอาจจะแบ่งกองไม้เป็นสองกองที่มีขนาดต่างๆ กัน แต่เราไม่มีวิธีการเปรียบเทียบคะแนนในแต่ละกลยุทธ์ นอกจากนี้ คะแนนในตาแรกก็ยังขึ้นกับ $k$ อีกด้วย \enskip ปัญหาที่เกิดขึ้นนี้สืบเนื่องมาจาก induction hypothesis ที่เรามีอยู่นั้นไม่แข็งแรงพอ ซึ่งมีวิธีแก้ไขคือเขียน predicate ใหม่ให้มีเงื่อนไขหรือข้อผูกมัดให้มากขึ้น \enskip ในที่นี้ หากเราหาสูตรที่คำนวณคะแนนได้ออกมาอย่างแน่ชัด ก็เพียงพอที่จะใช้พิสูจน์ได้โดยไม่ติดขัด \enskip โปรดสังเกตว่า หาก induction hypothesis พูดถึงคะแนนรวมที่คำนวณออกมาได้อย่างเฉพาะเจาะจง ก็ย่อมเกิดข้อผูกมัดที่ทำให้ induction hypothesis นั้นแข็งแรงขึ้น ซึ่งจะทำให้เราสามารถตั้งสมมุติฐานได้รัดกุมมากขึ้นด้วย และหากสามารถคำนวณคะแนนให้เฉพาะเจาะจงได้ ก็แปลว่าคะแนนที่จะได้นั้นก็ย่อมเท่ากัน ดั่งที่เราต้องการจะพิสูจน์ตั้งแต่ต้น
\end{theorem}

\begin{theorem}
ทุกๆ กลยุทธ์ในการแบ่งกองไม้ $n$ ชิ้นจนหมด (เมื่อ $n\geq 1$) จะได้คะแนนเท่ากัน คือ $S(n)=\frac{n(n-1)}{2}$
\begin{pf}
By strong induction.  ให้ $P(n)\triangleq$ ทุกๆ กลยุทธ์ในการแบ่งกองไม้ $n$ ชิ้นจนหมด จะได้คะแนน $S(n)=\frac{n(n-1)}{2}$
\begin{itemize}
\item {\bf Base case}: ($n=1$) \quad วิธีการแบ่งกองไม้ 1 ชิ้น ย่อมมีวิธีเดียว คือ ไม่แบ่ง ซึ่งไม่ได้คะแนน และเราต้องการพิสูจน์ว่า ทุกๆ กลยุทธ์ก็ย่อมได้คะแนนเท่ากันคือ $S(0)=\frac{0(1-0)}{2}=0$ ซึ่งถูกต้อง \quad\yea
\item {\bf Inductive step}: สมมุติว่าทุกๆ กลยุทธ์ในการแบ่งกองไม้ $k$ ชิ้นจนหมด โดยที่ $1\leq k\leq n$ จะได้คะแนนเท่ากันคือ $S(k)=\frac{k(k-1)}{2}$ \enskip ต้องพิสูจน์ว่า ทุกๆ กลยุทธ์ในการแบ่งกองไม้ $n+1$ ชิ้นจนหมด จะได้คะแนน $S(n+1)=\frac{(n+1)n}{2}$

หากแบ่งกองไม้ $n+1$ ชิ้นเป็นสองกอง โดยที่กองหนึ่งมี $k$ ชิ้น จะได้ว่า อีกกองหนึ่งมี $n+1-k$ ชิ้น ดังนั้น ในตานี้ จะได้คะแนน $k(n+1-k)$ ทันที จากนั้น ต้องแบ่งกองไม้แต่ละกองไปเรื่อยๆ จนแบ่งต่อไม่ได้อีก \enskip By I.H., จะได้ว่า ไม่ว่าจะใช้กลยุทธ์ใดก็ตาม คะแนนรวมที่จะได้จากการแบ่งไม้ $k$ ชิ้น คือ $S(k)=\frac{k(k-1)}{2}$ และคะแนนรวมที่จะได้จากการแบ่งไม้ $n+1-k$ ชิ้น คือ $S(n+1-k)=\frac{(n+1-k)(n-k)}{2}$ (สังเกตว่า $k$ และ $n+1-k$ ทั้งคู่นั้นมีค่าอย่างน้อย 1 และอย่างมาก $n$ ซึ่งตรงตามเงื่อนไขของ I.H. ที่เราสมมุติไว้) ดังนั้น คะแนนรวมที่จะได้จากการแบ่งกองไม้ $n+1$ ชิ้นคือ
\begin{align*}
S(n+1)
&=k(n+1-k)+S(k)+S(n+1-k) \\
&=k(n+1-k)+\frac{k(k-1)}{2}+\frac{(n+1-k)(n-k)}{2} \\
&=(n+1-k)\left[k+\frac{(n-k)}{2}\right]+\frac{k(k-1)}{2} \\
&=(n+1-k)\left[\frac{n+k}{2}\right]+\frac{k(k-1)}{2} \\
&=\frac{((n+1)-k)(n+k)}{2}+\frac{k(k-1)}{2} \\
&=\frac{(n+1)n-kn+k(n+1)-k^2+k(k-1)}{2} \\
&=\frac{(n+1)n-kn+kn+k-k^2+k^2-k}{2} \\
&=\frac{(n+1)n}{2} \quad\textup{\yea}
\end{align*}
\end{itemize}
\end{pf}
\end{theorem}

\subsection{Induction vs strong induction}
ในการพิสูจน์โดยอุปนัยอย่างแรงนั้น สมมุติฐานที่เรามีในขั้นอุปนัย (inductive step) นั้นมีมากกว่าในการพิสูจน์โดยอุปนัย \enskip จากข้อเท็จจริงนี้ เราสามารถสรุปได้ว่า หากเราสามารถใช้ induction พิสูจน์ประพจน์หนึ่งๆ ได้ ก็ย่อมสามารถใช้ strong induction พิสูจน์ได้เช่นกัน \enskip แต่ในทางกลับกัน หากเราสามารถใช้ strong induction พิสูจน์ประพจน์หนึ่งๆ ได้ แล้วเราจะสามารถใช้ induction พิสูจน์ประพจน์นั้นๆ ได้หรือไม่

หากพิจารณาโดยผิวเผิน อาจจะเห็นว่า การพิสูจน์โดยอุปนัยอย่างแรงนั้นน่าจะทำให้เราเขียนบทพิสูจน์ได้มากกว่าเดิม กล่าวคือ อาจจะมี propositions บางตัวที่เราไม่สามารถพิสูจน์โดยใช้ induction ได้ แต่สามารถพิสูจน์โดยใช้ strong induction ได้ อย่างไรก็ดี strong induction นั้นไม่ทำให้เราพิสูจน์ได้มากขึ้นแต่อย่างใด แต่เป็นเพียงเครื่องมืออำนวยความสะดวกให้เราเขียนบทพิสูจน์ได้รวดเร็วและกระชับยิ่งขึ้น \enskip กล่าวอีกนัยหนึ่ง หากใช้ strong induction พิสูจน์อะไรได้ ก็สามารถใช้ induction พิสูจน์ได้เช่นเดียวกัน แต่บทพิสูจน์ที่ใช้ induction นั้นอาจจะมีความซับซ้อนมากกว่า ดังที่จะได้สาธิตต่อไปนี้

ในการพิสูจน์โดยใช้ strong induction นั้น เราต้องการพิสูจน์ว่า predicate $P(n)$ หนึ่งๆ นั้นเป็นจริงสำหรับทุกค่า $n\in\mathbb{N}$ โดยใน inductive step เรามีสมมุติฐานคือ $\forall n'\leq n: P(n')$ \enskip เราจะแปลงบทพิสูจน์นี้เป็นบทพิสูจน์โดยใช้ induction โดยจะพิสูจน์ predicate $Q(n)\triangleq\forall n'\leq n: P(n')$ ว่าเป็นจริงสำหรับทุกค่า $n\in\mathbb{N}$
\begin{itemize}
\item {\bf Base case}: ($n=0$) \quad ต้องพิสูจน์ว่า $Q(0)$ เป็นจริง กล่าวคือ $\forall n'\leq 0: P(n')$ ซึ่งมี $n'$ เพียงค่าเดียวที่เป็นไปได้คือ $0$ \enskip ดังนั้น ในการพิสูจน์ว่า $Q(0)$ เป็นจริง จึงใช้บทพิสูจน์เดียวกันกับการแสดงว่า $P(0)$ เป็นจริงในบทพิสูจน์โดยใช้ strong induction ที่เรามีอยู่เดิม
\item {\bf Inductive step}: ในขั้นตอนนี้ เราต้องพิสูจน์ว่า $Q(n+1)$ เป็นจริง โดยสมมุติฐานที่เรามีคือ $Q(n)$ ซึ่งแปลว่า $\forall n'\leq n: P(n')$ กล่าวคือ $P(0),P(1),\ldots,P(n)$ เป็นจริง \enskip ดังนั้น หากจะพิสูจน์ว่า $Q(n+1)$ เป็นจริง กล่าวคือ $\forall n'\leq n+1: P(n')$ เราจึงต้องพิสูจน์ว่า $P(n+1)$ เป็นจริง (เนื่องจากเราทราบว่า $P(0),\ldots,P(n)$ เป็นจริงอยู่แล้วจากสมมุติฐาน) ซึ่งเราสามารถใช้บทพิสูจน์เดียวกันกับการแสดงว่า $P(n+1)$ เป็นจริงในบทพิสูจน์โดยใช้ strong induction ที่เรามีอยู่เดิม
\end{itemize}
ดังนั้น บทพิสูจน์ข้างต้นจึงเป็นบทพิสูจน์ที่ใช้วิธี induction แต่สามารถพิสูจน์ประพจน์ที่พิสูจน์ได้โดยวิธี strong induction ซึ่งแปลว่าวิธี strong induction นั้นไม่ได้ทำให้เราสามารถพิสูจน์ประพจน์ได้จำนวนมากขึ้นกว่าเดิมแต่อย่างใด
