\section{Invariants}

\emph{ตัวยืนยง} (invariant) เป็นคุณสมบัติที่เป็นจริงในทุกขั้นตอนการทำงานของกระบวนการ (process) อาทิ โปรแกรมคอมพิวเตอร์ เครื่องจักรกล คอมพิวเตอร์ และ state machines

\begin{example}
หากคอมพิวเตอร์คือกระบวนการที่เรากำลังพิจารณา อาจจะเขียนว่า invariant ที่เราต้องการคือ การประมวลผลทุกคำสั่งจะไม่ทำให้เครื่องค้าง
\end{example}
อย่างไรก็ดี invariants ที่เราเขียนขึ้นมานั้นย่อมไม่มีประโยชน์ หากไม่สามารถรับประกันได้ว่าจะเป็นจริง ดังนั้น เราจำเป็นต้องเขียนบทพิสูจน์กำกับ \enskip ในตัวอย่างข้างต้น บทพิสูจน์นั้นเขียนได้ยากมาก เนื่องจากการประมวลผลคำสั่งหนึ่งๆ มีหลายปัจจัยด้วยกันที่อาจจะก่อให้เกิดปัญหาขึ้น ทั้งนี้ การพยายามรับประกันคุณสมบัติดังกล่าว เป็นหัวข้อที่นักวิจัยยังคงค้นคว้าหาวิธีการอยู่แม้ในปัจจุบัน

\begin{example}
ในกรณีประตูเลื่อนอัตโนมัติ ประตูเปิดก็ต่อเมื่อมีคนอยู่ใกล้ประตู กล่าวคือ มีคนอยู่ด้านนอก ด้านใน หรือทั้งคู่ของประตู
\begin{pf}
($\Leftarrow$) \quad ถ้ามีคนอยู่ใกล้ประตู แปลว่า นอก/ใน/both เป็นจริง พิจารณาแต่ละกรณีว่าประตูจะทำอะไร
\begin{itemize}
\item นอก: หากประตูปิดอยู่จะเปิด แต่หากเปิดอยู่ก็จะเปิดต่อไป
\item ใน: เช่นเดียวกับกรณีข้างต้น
\item both: เช่นเดียวกับกรณีข้างต้น
\end{itemize}
ดังนั้น ประตูจะเปิดเสมอในทุกกรณี

($\Rightarrow$) \quad ถ้าประตูเปิด พิจารณากรณีที่ทำให้ประตูเปิด ตาม state diagram ใน Example~\ref{ex:door-diagram} จะได้ว่า มี นอก/ใน/both เท่านั้น ดังนั้น ต้องมีคนอยู่ใกล้ประตู
\end{pf}
\end{example}

\begin{example}
ในกรณีหุ่นยนต์เดินทแยง เราต้องการทราบว่า หุ่นยนต์นั้นจะสามารถเดินจากพิกัด $(0,0)$ ไปยังพิกัด $(1,0)$ ได้หรือไม่

หากเราสามารถหา invariant ที่อธิบาย states ทั้งหมดที่หุ่นยนต์สามารถไปถึงได้ แล้วพิกัดเป้าหมายนั้นสอดคล้องกับ invariant ดังกล่าว จะได้ว่า มีวิธีเดินไปยังพิกัดเป้าหมาย แต่ถ้าพิกัดเป้าหมายนั้นไม่สอดคล้องกับ invariant แสดงว่าไม่มีวิธีเดินที่ต้องการ
\end{example}

ให้ predicate $\textup{Even-sum}((x,y))\triangleq x+y$ เป็นเลขคู่ \enskip เราจะพิสูจน์ว่า Even-sum นี้เป็นตัวยืนยงที่คงสภาพ (preserved invariant) ภายใต้ transition ใดๆ ของหุ่นยนต์เดินทแยงนี้ โดยเริ่มจาก\emph{บทตั้ง} (lemma) ซึ่งจะใช้ในการพิสูจน์ theorem ในภายหลัง \enskip บทตั้งนี้กล่าวว่า หาก invariant เป็นจริงในสถานะปัจจุบัน แล้ว invariant จะเป็นจริงในสถานะต่อไปด้วย

\begin{lemma}\label{lemma:even-sum-trans}
ให้ $q,r$ เป็น states \enskip ถ้า $q\to r$ เป็น transition สำหรับหุ่นยนต์ และ $\textup{Even-sum}(q)$ แล้ว $\textup{Even-sum}(r)$
\begin{pf}
หาก $q=(x,y)$ เป็น state ที่ $\textup{Even-sum}(q)$ จะได้ว่า $x+y$ เป็นเลขคู่ \enskip ถ้า $q\to r$ เป็น transition จะได้ว่า $r$ เป็นไปได้ 4 กรณี
\begin{itemize}
\item $r=(x+1,y+1)$: ในที่นี้ $(x+1)+(y+1)=x+y+2$ แต่เนื่องจาก $x+y$ เป็นเลขคู่ จะได้ว่า $x+y+2$ เป็นเลขคู่ด้วย ดังนั้น $\textup{Even-sum}(r)$
\item $r=(x+1,y-1)$: ในที่นี้ $(x+1)+(y-1)=x+y$ ซึ่งเป็นเลขคู่
\item $r=(x-1,y+1)$: คล้ายกับกรณีที่แล้ว
\item $r=(x-1,y-1)$: $(x-1)+(y-1)=x+y-2$ ซึ่งเป็นเลขคู่เช่นกัน
\end{itemize}
ดังนั้น ในทุกกรณี จะได้ว่า $\textup{Even-sum}(r)$
\end{pf}
\end{lemma}

ต่อไป จะใช้บทตั้งช่วยในการพิสูจน์ทฤษฎีบทหลักว่า ไม่ว่าหุ่นยนต์จะเดินไปกี่ครั้งก็ตามจาก start state จะได้ว่า ตำแหน่งที่หุ่นยนต์ไปถึงได้ย่อมมีคุณสมบัติ Even-sum

\begin{theorem}\label{thm:even-sum-all}
ถ้า $r$ เป็น state ที่หุ่นยนต์ไปถึงได้จาก start state $(0,0)$ แล้ว $\textup{Even-sum}(r)$
\begin{pf}
By induction on the number of moves.  ให้ $P(n)\triangleq$ ถ้า $r$ เป็น state ที่หุ่นยนต์ไปถึงได้จาก start state โดยเดิน $n$ ครั้ง แล้ว $\textup{Even-sum}(r)$
\begin{itemize}
\item {\bf Base case}: ($n=0$) \quad หากเดิน 0 ครั้ง ก็ยังอยู่ที่ $r=(0,0)$ ซึ่ง $0+0=0$ เป็นเลขคู่ \quad\yea
\item {\bf Inductive step}: สมมุติว่า $P(n)$ เป็นจริง ต้องพิสูจน์ว่า ถ้า $r$ เป็น state ที่หุ่นยนต์ไปถึงได้จาก start state โดยเดิน $ืn+1$ ครั้ง แล้ว $\textup{Even-sum}(r)$

เนื่องจาก $r$ ไปถึงได้ใน $n+1$ transitions จะได้ว่ามี state $q$ ที่หุ่นยนต์ไปถึงก่อนหน้า $r$ หนึ่งก้าว กล่าวคือ $q$ ไปถึงได้ใน $n$ transitions \enskip จากนั้น หุ่นยนต์เดินจาก $q$ ไป $r$ โดยใช้ transition $q\to r$ \enskip By I.H., $\textup{Even-sum}(q)$ เป็นจริง \enskip เนื่องจาก $q\to r$ เป็น transition และ $\textup{Even-sum}(q)$ เป็นจริง จาก Lemma~\ref{lemma:even-sum-trans} จะได้ว่า $\textup{Even-sum}(r)$ เป็นจริงด้วย กล่าวคือ $P(n+1)$ เป็นจริง \quad\yea
\end{itemize}
ดังนั้น if $r$ is reachable in $n$ transitions, then $\textup{Even-sum}(r)$ กล่าวคือ state ที่ไปถึงได้ทุก state มีสมบัติ Even-sum นี้
\end{pf}
\end{theorem}

สุดท้ายนี้ เราสามารถสรุปได้ว่า ไม่มีหนทางที่หุ่นยนต์จะเดินไปยังพิกัด $(1,0)$ ได้ โดยเขียนเป็น\emph{บทแทรก} (corollary) ซึ่งสามารถสรุปได้ง่ายๆ จากทฤษฎีบทหลัก

\begin{corollary}
หุ่นยนต์จะเดินไป $(1,0)$ ไม่ได้
\begin{pf}
จาก Theorem~\ref{thm:even-sum-all} จะได้ว่า หุ่นยนต์เดินไปยัง state ที่มีสมบัติ Even-sum ได้เท่านั้น แต่ $1+0=1$ เป็นเลขคี่ ดังนั้น state $(1,0)$ ไม่มีสมบัติ Even-sum กล่าวคือ หุ่นยนต์ไปถึงไม่ได้
\end{pf}
\end{corollary}
