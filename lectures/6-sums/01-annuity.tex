\section{Annuity}

สมมุติว่าเราได้ไปออกรายการโทรทัศน์ที่มีเงินรางวัล jackpot แล้วเราได้ jackpot แต่ทางรายการให้เราเลือกวิธีการจ่ายเงินรางวัลจากสองวิธีดังนี้
\begin{itemize}
\item จ่าย $k$ บาทตอนนี้เป็นก้อนใหญ่ก้อนเดียว
\item จ่ายปีละ 5 แสนบาท เป็นเวลา 20 ปี
\end{itemize}
เราควรจะเลือกรับเงินรางวัลด้วยวิธีใด

การตัดสินใจของเรานั้นย่อมขึ้นกับค่าของ $k$ กล่าวคือ ถ้า $k$ มีค่ามากๆ วิธีแรกน่าจะทำให้เราได้กำไรมากกว่า แต่ถ้า $k$ มีค่ามากไม่พอ วิธีที่สองน่าจะดีกว่าสำหรับเรา \enskip คำถามที่ตามมาก็คือ ค่าของ $k$ ที่ทำให้วิธีการทั้งสองไม่ดีไปกว่ากันคือเท่าใด \enskip ค่า $k$ ดังกล่าวคือ\emph{ค่าที่เท่าทุน} (break-even value)

ลักษณะการจ่ายเงินรางวัลในแบบที่สองข้างต้นเรียกว่า annuity
\begin{definition}
\emph{เงินรายปีที่จ่าย $n$ ปี ปีละ $m฿$} ($n$-year, $m฿$-payment annuity) จ่าย $m฿$ ทุกๆ ต้นปี เป็นเวลา $n$ ปี
\end{definition}
ตัวอย่างอื่นๆ ของ annuity ได้แก่เบี้ยประกันชีวิต ที่เราต้องจ่ายเป็นประจำทุกๆ ปีเป็นมูลค่าเท่าๆ กัน หรือการผ่อนชำระหนี้ ซึ่งรวมถึงค่าเช่าต่างๆ ที่ต้องจ่ายเป็นงวดๆ เดือนละครั้งตามระยะเวลาที่กำหนด (ในที่นี้ เราต้องจ่ายทุกๆ ต้นเดือนแทนที่จะจ่ายทุกๆ ต้นปี)

หากจะคำนวณ break-even value เราต้องคำนึงว่า หากเรามีเงิน ณ ขณะนี้ เราสามารถนำไปบริหารจัดการให้เกิดดอกออกผลได้ เช่น ฝากธนาคาร ลงทุนในตลาดหลักทรัพย์ หรือหาซื้ออสังหาริมทรัพย์เพื่อเก็งกำไร \enskip กล่าวคือ เงินที่เรามีอยู่ตอนนี้สามารถสร้างกำไรหรือดอกเบี้ยได้หากใช้งานให้เกิดประโยชน์สูงสุด \enskip นับจากจุดนี้ เราจะตั้งสมมุติฐานว่า เราสามารถสร้างกำไรจากเงินที่เรามีอยู่ตอนนี้ในอัตราดอกเบี้ยที่คงที่ (fixed interest rate) ตลอดระยะเวลาที่เรากำลังพิจารณา โดยให้ $p$ เป็นอัตราดอกเบี้ย หรือ\emph{ผลตอบแทนร้อยละต่อปี} (annual percentage yield) [APY] ดังกล่าว
%
\begin{example}
หากอัตราดอกเบี้ยเป็น 6\% จะได้ว่า $p=0.06$
\end{example}

เมื่อทราบอัตราดอกเบี้ยแล้ว เราสามารถคำนวณค่าเงินที่เราจะมีได้ในภายภาคหน้า \enskip สมมุติว่า ณ วันนี้เรามีเงิน $100฿$ แล้วนำไปบริหารจัดการโดยได้กำไรในอัตรา $p$ ต่อปี \enskip ค่าเงินสุทธิหลังจากปีต่างๆ เป็นดังนี้
\begin{itemize}
\item เมื่อเวลาผ่านไปหนึ่งปี เราจะมีเงินต้นคือ $100฿$ บวกกับดอกเบี้ยที่ได้ คือ $(100\cdot p)฿$ \enskip ดังนั้น เงินรวมสุทธิหลัง 1 ปีคือ $100(1+p)฿$
\item เมื่อเวลาผ่านไปสองปี เงินที่เรามีหลังจากหนึ่งปีก็จะได้ดอกเบี้ยเข้ามาสมทบ กล่าวคือ หากเงินต้นเป็น $100(1+p)฿$ เราจะได้ดอกเบี้ย $100(1+p)\cdot p฿$ \enskip ดังนั้น เงินรวมสุทธิหลัง 2 ปีคือ $100(1+p)(1+p)฿=100(1+p)^2฿$
\item เมื่อเวลาผ่านไปสามปี เงินที่เรามีหลังจากสองปีก็จะได้ดอกเบี้ยเข้ามาสมทบ กล่าวคือ หากเงินต้นเป็น $100(1+p)^2 ฿$ เราจะได้ดอกเบี้ย $100(1+p)^2\cdot p ฿$ \enskip ดังนั้น เงินรวมสุทธิหลัง 3 ปีคือ $100(1+p)^2(1+p)฿=100(1+p)^3฿$
\item จากแบบอย่างข้างต้น เราสามารถสรุปได้ว่า หากวันนี้เรามีเงิน $100฿$ และเราได้ดอกเบี้ยในอัตรา $p$ ต่อปี จะได้ว่า เงินรวมสุทธิหลัง $n$ ปีคือ $100(1+p)^n฿$
\item ดังนั้น หากวันนี้เรามีเงิน $x฿$ และเราได้ดอกเบี้ยในอัตรา $p$ ต่อปี จะได้ว่า เงินรวมสุทธิหลัง $n$ ปีคือ $x(1+p)^n฿$
\end{itemize}
ในทางกลับกัน เราสามารถคำนวณได้ว่า หากเราต้องการจะมีเงินรวมสุทธิ $100฿$ ในภายภาคหน้า เราควรจะเริ่มต้นด้วยเงินเท่าใด ณ เวลานี้ \enskip จากข้างต้น หากวันนี้เรามีเงิน $x฿$ และเราได้ดอกเบี้ยในอัตรา $p$ ต่อปี แล้วเงินรวมสุทธิหลัง $n$ ปีคือ $x(1+p)^n฿$ \enskip ดังนั้น หากต้องการให้เงินรวมสุทธิหลัง $n$ ปีเป็น $x฿$ เราต้องเริ่มต้นด้วยเงิน $\frac{x}{(1+p)^n}฿$ ในวันนี้ กล่าวคือ
\begin{itemize}
\item หากเริ่มด้วย $\frac{100}{1+p}฿$ วันนี้ เมื่อเวลาผ่านไปหนึ่งปี ดอกเบี้ยที่จะได้ก็คือ $\frac{100}{1+p}p฿$ ดังนั้น เงินรวมสุทธิหลัง 1 ปีคือ $\frac{100}{1+p}(1+p)฿=100฿$
\item หากเริ่มด้วย $\frac{100}{(1+p)^2}฿$ วันนี้ เมื่อเวลาผ่านไปหนึ่งปี ดอกเบี้ยที่จะได้ก็คือ $\frac{100}{(1+p)^2}p฿$ ดังนั้น เงินรวมสุทธิหลัง 1 ปีคือ $\frac{100}{(1+p)^2}(1+p)฿=\frac{100}{1+p}฿$ \enskip และเมื่อเวลาผ่านไปอีกหนึ่งปีรวมเป็นสองปี ดอกเบี้ยที่จะได้เพิ่มเติมก็คือ $\frac{100}{1+p}p฿$ ดังนั้น เงินรวมสุทธิหลัง 2 ปีคือ $\frac{100}{1+p}(1+p)฿=100฿$
\item ในทำนองเดียวกัน หากเริ่มด้วย $\frac{100}{(1+p)^3}฿$ วันนี้ เมื่อเวลาผ่านไปสามปี เงินรวมสุทธิจะเป็น $100฿$
\end{itemize}
การคำนวณในลักษณะนี้สามารถนำไปใช้หา break-even value สำหรับวิธีการจ่ายเงินรางวัลข้างต้นได้ดังนี้ \enskip สมมุติว่าทางรายการจ่ายเงินรางวัลปีละ $m฿$ เป็นเวลา $n$ ปี (จากตัวอย่างข้างต้น $m=500{,}000$ และ $n=20$)
\begin{itemize}
\item ในงวดแรก ทางรายการจ่ายเงิน $m฿$ ณ วันนี้ \enskip มูลค่าเงินดังกล่าว ณ เวลานี้คือ $m฿$
\item ในงวดที่สอง ทางรายการจ่ายเงิน $m฿$ อีกหนึ่งปีถัดไป \enskip มูลค่าเงินดังกล่าว ณ เวลานี้คือ $\frac{m}{1+p}฿$ \enskip กล่าวคือ หากทางรายการจ่ายเรา $\frac{m}{1+p}฿$ ณ วันนี้ อีกหนึ่งปีก็จะมีค่าเท่ากับที่ทางรายการจะรอจ่าย $m฿$ ในปีหน้า
\item ในงวดที่สาม ทางรายการจ่ายเงิน $m฿$ อีกสองปีถัดไป \enskip มูลค่าเงินดังกล่าว ณ เวลานี้คือ $\frac{m}{(1+p)^2}฿$
\item ในงวดสุดท้าย (ที่ $n$) ทางรายการจ่ายเงิน $m฿$ อีก $n-1$ ปีถัดไป \enskip มูลค่าเงินดังกล่าว ณ เวลานี้คือ $\frac{m}{(1+p)^{n-1}}฿$
\end{itemize}
ดังนั้น หากทางรายการจะจ่ายเป็นงวดๆ ทุกๆ ปี มูลค่าเงินรวมที่เทียบเท่า ณ เวลานี้คือ \[V\triangleq m+\frac{m}{1+p}+\frac{m}{(1+p)^2}+\cdots+\frac{m}{(1+p)^{n-1}}=\sum_{i=0}^{n-1}\frac{m}{(1+p)^i} \quad ฿\]

นั่นคือ หากทางรายการจะจ่ายเราเป็นเงินรางวัลก้อนใหญ่ที่มีค่าเป็นผลรวมดังข้างต้น ณ เวลานี้ ก็จะมีมูลค่ารวมเท่ากับการที่เรารับเงินเป็นงวดๆ ทุกๆ ปี
