\section{Sums}

ผลรวมของมูลค่าเงินรางวัลจากก่อนหน้านี้ \[V=\sum_{i=0}^{n-1}\frac{m}{(1+p)^i}\] อาจจะคำนวณได้ไม่สะดวกเท่าใดนัก เนื่องจากเราต้องนำแต่ละพจน์ในผลรวมมาบวกกันทีละตัวจนได้คำตอบ \enskip หากต้องการคำนวณผลรวมให้รวดเร็ว เราต้องหา\emph{สูตรในรูปแบบปิด} (closed-form formula) ของผลรวมดังกล่าว ให้เป็นฟังก์ชันที่ไม่ขึ้นกับตัวแปร $i$ แต่สามารถขึ้นกับค่าของ $n$ ได้

หากจะจัดรูปผลรวมข้างต้น โดยให้ $x\triangleq \frac{1}{1+p}$ จะได้ว่า
\begin{align*}
\sum_{i=0}^{n-1}\frac{m}{(1+p)^i}
&=m\sum_{i=0}^{n-1}\frac{1}{(1+p)^i} \\
&=m\sum_{i=0}^{n-1}\left(\frac{1}{1+p}\right)^i \\
&=m\sum_{i=0}^{n-1}x^i
\end{align*}
สังเกตว่า $\sum_{i=0}^{n-1}{x^i}$ เป็นอนุกรมเรขาคณิต (geometric series) \enskip หากเราสามารถหาสูตรในรูปแบบปิดของอนุกรมดังกล่าวได้ เราก็จะสามารถหามูลค่าเงินรางวัลได้ง่ายขึ้น
\begin{theorem}\label{thm:geom-sum}
ให้ $x\neq 1$ \enskip จะได้ว่า $\forall n\geq 1: \sum_{i=0}^{n-1}x^i=\frac{1-x^n}{1-x}$
\begin{pf}
By induction.  ให้ \[P(n)\triangleq \sum_{i=0}^{n-1}x^i=\frac{1-x^n}{1-x}\] โดยที่ $n\geq 1$
\begin{itemize}
\item {\bf Base case}: ($n=1$) \quad $\sum_{i=0}^{1-1}x^i=x^0=1$ และ $\frac{1-x^1}{1-x}=1$ \quad\yea
\item {\bf Inductive step}: สมมุติว่า $\sum_{i=0}^{n-1}x^i=\frac{1-x^n}{1-x}$ \enskip ต้องพิสูจน์ว่า $P(n+1)$ เป็นจริง ดังนี้
\begin{align*}
\sum_{i=0}^n x^i
&=\sum_{i=0}^{n-1}x^i+x^n \\
&=\frac{1-x^n}{1-x}+x^n \qquad\textup{(by I.H.)} \\
&=\frac{1-x^n+x^n-x\cdot x^n}{1-x} \\
&=\frac{1-x^{n+1}}{1-x} \qquad\textup{\yea}
\end{align*}
\end{itemize}
\end{pf}
อย่างไรก็ดี จะเห็นว่าการใช้ induction ไม่ช่วยให้เราค้นหาสูตรในรูปแบบปิดได้เอง \enskip ดังนั้น หากต้องการสร้าง closed form ขึ้นมาเอง ต้องใช้วิธีการอื่น ดังนี้
\begin{pf}
ใช้ \emph{perturbation method} โดยการปรับเปลี่ยนรูปร่างลักษณะของผลรวม ดังที่จะได้แสดงต่อไปนี้

ให้ $S=\sum_{i=0}^{n-1}x^i$ \enskip หากเขียนกระจายผลรวมทีละพจน์ จะได้ว่า \[S=1+x+x^2+\cdots+x^{n-1}\] ต่อไป จะทำการคูณทั้งสมการด้วย $x$ ได้ผลลัพธ์เป็น \[xS=x+x^2+x^3+\cdots+x^n\] หากนำสมการแรก ลบด้วยสมการที่สองข้างต้น จะได้ว่า \[(1-x)S=1-x^n\] ดังนั้น \[S=\frac{1-x^n}{1-x}\] ตามที่ต้องการ (เราสามารถนำ $1-x$ หารทั้งสมการได้ เนื่องจากเราทราบว่า $x\neq 1$)
\end{pf}
\end{theorem}

ดังนั้น เราสามารถหา closed form ของมูลค่าเงินรางวัลได้ดังนี้
\begin{align*}
V
&=m\sum_{i=0}^{n-1}x^i \\
&=m\left(\frac{1-x^n}{1-x}\right) \\
&=m\left(\frac{1-\left(\frac{1}{1+p}\right)^n}{1-\frac{1}{1+p}}\right) \\
&=m\left(\frac{1-\left(\frac{1}{1+p}\right)^n}{\frac{p}{1+p}}\right) \\
&=m(1+p)\left(\frac{1-\frac{1}{(1+p)^n}}{p}\right) \\
&=m\left(\frac{1+p-\frac{1}{(1+p)^{n-1}}}{p}\right)
\end{align*}
%
\begin{example}
หาก $m=500{,}000฿$, $n=20$ ปี, และ $p=0.06$ จะได้ว่า $V=6{,}079{,}058$ \enskip ดังนั้น หากทางรายการเสนอจ่ายเกินก้อนโตครั้งเดียวที่มีมูลค่ามากกว่า $V$ เราควรจะเลือกรับเงินก้อนนี้ไว้ มิฉะนั้น ให้เลือกรับเป็นงวดๆ จะคุ้มค่ากว่า
\end{example}

หลังจากที่เราหาวิธีคำนวณ break-even value ดังกล่าวได้ ทางรายการได้ทราบถึงวิธีการดังกล่าว จึงได้คิดค้นวิธีการจ่ายเงินขึ้นมาใหม่ โดยแทนที่จะจ่ายเป็นงวดๆ ไปสักระยะหนึ่ง จะจ่ายเป็นงวดๆ ตลอดไปไม่มีที่สิ้นสุด คำถามเดิมจึงย้อนกลับมา ว่าเราควรจะรับเงินก้อนครั้งแรกเพียงครั้งเดียว หรือจะรับแบบเป็นงวดๆ ตลอดไปจึงจะคุ้มค่ากว่า

ในกรณีนี้ จำนวนปี $n=\infty$ ดังนั้น หากจะคำนวณ break-even value สามารถหา limit ของสูตรของ $V$ ดังกล่าวได้ดังนี้ \[\lim_{n\to\infty}{m\left(\frac{1+p-\frac{1}{(1+p)^{n-1}}}{p}\right)}=m\frac{1+p}{p}\] กล่าวคือ มูลค่าเงินรางวัลรวม ณ ปัจจุบันนั้น แท้ที่จริงแล้วก็ยังเป็นจำนวนจำกัด
%
\begin{example}
หาก $m=500{,}000฿$ และ $p=0.06$ จะได้ว่า break-even value คือ $V=8{,}833{,}333$
\end{example}

\begin{theorem}
ถ้า $|x|<1$ แล้ว $\sum_{i=0}^{\infty}{x^i}=\frac{1}{1-x}$
\begin{pf}
\[\lim_{n\to\infty}{\sum_{i=0}^{n}{x^i}}=\lim_{n\to\infty}{\frac{1-x^{n+1}}{1-x}}=\frac{1}{1-x}\]
\end{pf}
\end{theorem}
%
\begin{example}
\[1+\frac{1}{2}+\frac{1}{4}+\frac{1}{8}+\cdots=\frac{1}{1-\frac{1}{2}}=2\]
\end{example}
\begin{example}
\[1+\frac{1}{3}+\frac{1}{9}+\frac{1}{27}+\cdots=\frac{1}{1-\frac{1}{3}}=\frac{3}{2}\]
\end{example}

ตัวอย่างต่อไป จะใช้ perturbation method ในการหา closed form ของผลรวมที่ซับซ้อนขึ้น
\begin{example}
\[\sum_{i=1}^n{ix^i}=x+2x^2+3x^3+\cdots+nx^n\]
หากใช้ perturbation method จะได้ว่า
\begin{align*}
S &= x+2x^2+3x^3+\cdots+nx^n \\
xS &= x^2+2x^3+3x^4+\cdots+(n-1)x^n+nx^{n+1} \\
(1-x)S &= x+x^2+x^3+\cdots+x^n-nx^{n+1}
\end{align*}
จะเห็นว่า ผลลัพธ์ทางด้านขวาไม่ได้หักล้างกันหมด \enskip อย่างไรก็ดี หากละพจน์สุดท้ายไว้ พจน์ที่เหลือก็คืออนุกรมเรขาคณิตนั่นเอง
\begin{align*}
x+x^2+x^3+\cdots+x^n &= x(1+x+x^2+\cdots+x^{n-1}) \\
&= x\left(\frac{1-x^n}{1-x}\right) \qquad\textup{(จาก Theorem~\ref{thm:geom-sum})}
\end{align*}
ดังนั้น
\begin{align*}
(1-x)S
&= x\left(\frac{1-x^n}{1-x}\right)-nx^{n+1} \\
&= \frac{x-x^{n+1}-nx^{n+1}+nx^{n+2}}{1-x} \\
&= \frac{x-(n+1)x^{n+1}+nx^{n+2}}{1-x} \\
S &= \frac{x-(n+1)x^{n+1}+nx^{n+2}}{(1-x)^2}
\end{align*}

สำหรับอนุกรมนี้ นอกจากเราจะใช้ perturbation method แล้ว เรายังสามารถใช้\emph{วิธีเชิงอนุพันธ์} (derivative method) ได้ ดังนี้ โดยเริ่มจากอนุกรมเรขาคณิต (Theorem~\ref{thm:geom-sum}) \[\sum_{i=0}^n{x^i}=\frac{1-x^{n+1}}{1-x}\] จากนั้น หาอนุพันธ์เทียบกับ $x$ (derivative with respect to $x$) ของทั้งสองข้าง จะได้
\begin{align*}
\frac{d}{dx}\sum_{i=0}^n{x^i} &= \frac{d}{dx}\frac{1-x^{n+1}}{1-x} \\
\sum_{i=0}^{n}{ix^{i-1}} &= \frac{-(n+1)x^n}{1-x}+\frac{(1-x^{n+1})(-1)(-1)}{(1-x)^2} \\
&= \frac{-(n+1)x^n(1-x)}{(1-x)^2}+\frac{(1-x^{n+1})}{(1-x)^2} \\
&= \frac{-(n+1)x^n(1-x)+(1-x^{n+1})}{(1-x)^2} \\
&= \frac{-(n+1)x^n + (n+1)x^{n+1}+1-x^{n+1}}{(1-x)^2} \\
\sum_{i=0}^{n}{ix^{i-1}} &= \frac{1-(n+1)x^n +nx^{n+1}}{(1-x)^2}
\end{align*}
คูณ $x$ เข้าไปทั้งสองข้าง จะได้ \[\sum_{i=0}^{n}{ix^i} = \frac{x-(n+1)x^{n+1} +nx^{n+2}}{(1-x)^2}\]
จะเห็นว่า ผลลัพธ์ที่ได้นั้นตรงกับผลลัพธ์จาก perturbation method
\end{example}

\begin{example}
สมมุติว่าเราอยากออมเงินให้ได้ $250{,}000฿$ ใน 15 ปี โดยฝากเงินเดือนละเท่าๆ กัน เดือนละ $m฿$ \enskip หากอัตราดอกเบี้ยต่อเดือนคือ $0.15\%$ เราต้องฝากเงินเดือนละเท่าไรจึงจะออมได้ตามเป้าหมาย

สมมุติว่าอัตราดอกเบี้ยคือ $p$ (จากตัวอย่างนี้ $p=0.0015$) สามารถคำนวณยอดเงินสะสมในแต่ละต้นเดือนได้ดังนี้
\begin{itemize}
\item ณ วันนี้ เราฝากเงิน $m฿$ จึงมีเงินสะสม $m฿$
\item เมื่อเวลาผ่านไปหนึ่งเดือน เราจะมีเงินต้นคือ $m฿$ บวกกับดอกเบี้ยที่ได้ คือ $mp฿$ และฝากเพิ่มเข้าไปอีก $m฿$ \enskip ดังนั้น เงินรวมสุทธิหลัง 1 เดือนคือ $[m(1+p)+m]฿$
\item เมื่อเวลาผ่านไปสองเดือน เงินที่เรามีหลังจากหนึ่งเดือนก็จะได้ดอกเบี้ยเข้ามาสมทบ กล่าวคือ จากเงินต้น $[m(1+p)+m]฿$ ซึ่งก็คือเงินรวมสุทธิจากเดือนที่แล้ว เราจะได้ดอกเบี้ย $[m(1+p)+m]p฿$ แต่เรายังคงฝากเงินเพิ่มไปอีก $m฿$ \enskip ดังนั้น เงินรวมสุทธิหลัง 2 เดือนคือ $[[m(1+p)+m](1+p)+m]฿=[m(1+p)^2+m(1+p)+m]฿$
\item เมื่อเวลาผ่านไปสามเดือน เงินที่เรามีหลังจากสองเดือนก็จะได้ดอกเบี้ยเข้ามาสมทบ กล่าวคือ จากเงินต้น $[m(1+p)^2+m(1+p)+m]฿$ ซึ่งก็คือเงินรวมสุทธิจากเดือนที่แล้ว เราจะได้ดอกเบี้ย $[m+(1+p)^2+m(1+p)+m]p฿$ แต่เรายังคงฝากเงินเพิ่มไปอีก $m฿$ \enskip ดังนั้น เงินรวมสุทธิหลัง 3 เดือนคือ $[[m(1+p)^2+m(1+p)+m](1+p)+m]฿=[m(1+p)^3+m(1+p)^2+m(1+p)+m]฿$
\item จากแบบอย่างข้างต้น เราสามารถสรุปได้ว่า เมื่อเวลาผ่านไป 180 เดือน (15 ปี) เราจะมีเงินออมสะสม
\begin{align*}
S
&=[m(1+p)^{180}+\cdots+m(1+p)^2+m(1+p)+m]฿ \\
&=m\left[(1+p)^{180}+\cdots+(1+p)^2+(1+p)+1\right] \quad ฿ \\
&=m\sum_{i=0}^{180}{(1+p)^i} \quad ฿ \\
S &=m\frac{1-(1+p)^{181}}{1-(1+p)}฿
\end{align*}
ดังนั้น หาก $p=0.0015$ และเราต้องการให้ $S=250{,}000฿$ จะได้ว่า $m=1{,}203.22฿$ ซึ่งเป็นจำนวนเงินที่เราควรเก็บสะสมต่อเดือนเพื่อทำให้มีเงินออมตามเป้าหมาย
\end{itemize}
\end{example}
