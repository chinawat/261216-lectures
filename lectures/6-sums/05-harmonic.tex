\section{Harmonic numbers}

ในเกมซ้อนกองไม้ มีจุดมุ่งหมายคือ ต้องการซ้อนแท่งไม้ขึ้นมาเป็นกองสูง โดยพยายามทำให้ไม้ชิ้นบนยื่นออกมาจากขอบโต๊ะให้ได้มากที่สุดโดยที่กองไม้ทั้งกองยังไม่ล้ม ดังรูปที่~\ref{fig:book-stacking}
%
\begin{figure}
\begin{center}
\begin{tikzpicture}[node distance=0pt]
\node[block] (b1) at (0,0) {$b_1$};
\node[block] (b2) [below left=of b1.south] {$b_2$};
\node[block] (b3) [below left=of b2.south] {$b_3$};
\node [below left=1cm of b3.south] {\reflectbox{$\ddots$}};
\node[block] (bn-1) [below left=2cm of b3.south] {$b_{n-1}$};
\node[block] (bn) [below left=of bn-1.south] {$b_n$};
\node[draw,minimum width=3cm,minimum height=2cm] (table) [below left=of bn.south] {โต๊ะ};

\draw[dashed] (table.south east) -- (table.south east |- 0,-8);
\draw[dashed] (bn.south east) -- (bn.south east |- 0,-7);
\draw[<->] (table.south east |- 0,-7) -- node[above] {$r_n$} (bn.south east |- 0,-7);
\draw[dashed] (b1.south east) -- (b1.south east |- 0,-8);
\draw[<->] (table.south east |- 0,-8) -- node[above] {$r_1$} (b1.south east |- 0,-8);
\end{tikzpicture}
\end{center}
\caption{Block-stacking game}
\label{fig:book-stacking}
\end{figure}
%
\enskip หากพิจารณาไม้ชิ้นบนสุด ($b_1$) จะได้ว่าไม้ชิ้นนี้สามารถยื่นมาจากไม้ชิ้นรองลงมา ($b_2$) ได้ครึ่งท่อนก่อนที่ $b_1$ จะร่วงหล่นลงมา \enskip หากน้อยกว่าครึ่งท่อน เรายังสามารถดัน $b_1$ ออกมาได้อีกเพื่อเพิ่มระยะห่างจากโต๊ะ แต่ถ้าเกินครึ่งท่อน $b_1$ ก็จะร่วงลงมา

หากพิจารณาไม้สองชิ้นบนสุด ซึ่งต้องวางอยู่บนชิ้นที่สาม ($b_3$) ได้โดยไม่ร่วงหล่นลงมา จะได้ว่า จุดศูนย์กลางมวลของทั้งสองชิ้นจะต้องไม่เกินขอบด้านขวาของชิ้นที่สาม แต่ถ้าจุดศูนย์กลางมวลดังกล่าวไม่ได้อยู่ที่ขอบพอดี เราก็ยังสามารถดันไม้ทั้งสองท่อนออกมาได้อีกเพื่อเพิ่มระยะทาง

วิธีวางซ้อนท่อนไม้ดังที่ได้อธิบายไปข้างต้นเรียกว่า\emph{กลยุทธ์ละโมบ} (greedy strategy)

ต่อไป จะทำการวิเคราะห์ว่า กลยุทธ์ละโมบนี้สามารถทำให้เราซ้อนกองไม้ที่ยื่นออกมาจากขอบโต๊ะได้มากที่สุดเท่าใด \enskip ทั้งนี้ สมมุติว่าเรามีไม้ $n$ ชิ้น ยาวชิ้นละ 1 หน่วย

ให้ $r_k$ เป็นระยะที่ไม้ชิ้นที่ $k$ ยื่นออกมาจากขอบโต๊ะ ดังรูปข้างต้น \enskip ทั้งนี้ เพื่อความสะดวกในการวิเคราะห์ ให้ $r_{n+1}=0$ กล่าวคือ ให้นับโต๊ะเป็นไม้ชิ้นที่ $n+1$ โดยระยะที่โต๊ะยื่นออกมาจากขอบโต๊ะก็คือ 0 \enskip ถัดไป ให้ $C_k$ เป็นตำแหน่งของจุดศูนย์กลางมวลของกองไม้ $k$ ชิ้นบนสุด โดยวัดจากขอบโต๊ะ \enskip \emph{เงื่อนไขเสถียรภาพ} (stability constraint) ที่เราต้องปฏิบัติตามกล่าวว่า จุดศูนย์กลางมวล $C_k$ ของกองไม้ $k$ ชิ้นบนสุดต้องอยู่ในขอบเขตของไม้ชิ้นที่ $k+1$ (มิฉะนั้นไม้ทั้งกองเหนือชิ้นที่ $k+1$ จะล้มลงมา) \enskip ใน greedy strategy ข้างต้นนั้น จะได้ว่า $C_k=r_{k+1}$

เนื่องจากไม้ชิ้นที่ $k$ ยื่นออกมาจากขอบโต๊ะเป็นระยะ $r_k$ และไม้แต่ละชิ้นมีความยาว 1 หน่วย จะได้ว่า จุดศูนย์กลางมวลของไม้ชิ้นที่ $k$ เพียงชิ้นเดียวอยู่ที่ตำแหน่ง $r_k-\frac{1}{2}$ จากขอบโต๊ะ \enskip จากข้อมูลดังกล่าว เราสามารถคำนวณหาตำแหน่งของ $C_k$ ได้โดยใช้ค่าเฉลี่ยถ่วงน้ำหนัก กล่าวคือ ไม้ $k-1$ ชิ้นบนสุดมีจุดศูนย์กลางมวลอยู่ที่ $C_{k-1}$ และน้ำหนักรวม $k-1$ ชิ้น ส่วนไม้ชิ้นที่ $k$ มีจุดศูนย์กลางมวลอยู่ที่ $r_k-\frac{1}{2}$ และน้ำหนักรวม 1 ชิ้น \enskip ดังนั้น
\begin{align*}
C_k &= \frac{(k-1)C_{k-1}+1\cdot(r_k-\frac{1}{2})}{(k-1)+1} \\
r_{k+1} &= \frac{(k-1)r_k+(r_k-\frac{1}{2})}{k} \qquad\textup{(เนื่องจาก $C_k=r_{k+1}$)} \\
r_{k+1} &= \frac{(k-1+1)r_k-\frac{1}{2}}{k} \\
r_{k+1} &= r_k-\frac{1}{2k} \\
r_k-r_{k+1} &= \frac{1}{2k}
\end{align*}
นั่นคือ ไม้ชิ้นที่ $k$ จะยื่นออกมาจากไม้ชิ้นที่ $k+1$ ได้เป็นระยะทาง $1/2k$ หน่วย ดังรูปต่อไปนี้
%
\begin{center}
\begin{tikzpicture}[node distance=0pt]
\node[block] (b1) at (0,0) {$b_k$};
\node[block] (b2) [below left=of b1.south] {$b_{k+1}$};

\draw[dashed] (b2.south east) -- (b2.south east |- 0,-2);
\draw[dashed] (b1.south east) -- (b1.south east |- 0,-2);
\draw[<->] (b2.south east |- 0,-2) -- node[above] {$\frac{1}{2k}$} (b1.south east |- 0,-2);
\draw[->] (b2.south east |- 0,-2 -| -4,0) -- node[above,near start] {$r_{k+1}$} (b2.south east |- 0,-2);
\draw[->] (b2.south east |- 0,-2.5 -| -4,0) -- node[above,near start] {$r_k$} (b1.south east |- 0,-2.5);
\end{tikzpicture}
\end{center}
%
หากเขียนแจกแจงความสัมพันธ์ดังกล่าวออกมาสำหรับไม้แต่ละคู่ จะได้
\begin{align*}
r_1-r_2 &= \frac{1}{2} \\
r_2-r_3 &= \frac{1}{4} \\
r_3-r_4 &= \frac{1}{6} \\
 &\vdots \\
r_n-r_{n+1} &= \frac{1}{2n}
\end{align*}
เมื่อบวกสมการทั้งหมดเข้าด้วยกัน จะได้
\begin{align*}
r_1-r_{n+1} &= \frac{1}{2}+\frac{1}{4}+\frac{1}{6}+\cdots+\frac{1}{2n} \\
r_1 &= \sum_{i=1}^n{\frac{1}{2i}} \qquad\textup{(เนื่องจาก $r_{n+1}=0$)} \\
r_1 &= \frac{1}{2}\sum_{i=1}^n{\frac{1}{i}}
\end{align*}
ผลรวม $\sum_{i=1}^n{\frac{1}{i}}$ นี้เกิดขึ้นในการแก้ปัญหาอื่นๆ อีกหลายๆ ปัญหา จึงมีชื่อเรียกเฉพาะว่า \emph{$n$\oth{} harmonic number} โดยเขียนเป็นสัญลักษณ์ว่า $H_n$: \[H_n\triangleq\sum_{i=1}^n{\frac{1}{i}}\]

หากคำนวณค่าของ $H_n$ ตัวแรกๆ จะได้ว่า
\begin{align*}
H_1 &= 1 \\
H_2 &= 1+\frac{1}{2}=\frac{3}{2} \\
H_3 &= \frac{3}{2}+\frac{1}{3}=\frac{11}{6} \\
H_4 &= \frac{11}{6}+\frac{1}{4}=\frac{25}{12}>2
\end{align*}
ดังนั้น เมื่อใช้ไม้ได้ทั้งหมด 4 ชิ้น จะได้ว่า ระยะที่ไม้ชิ้นบนสุดสามารถยื่นออกมาจากขอบโต๊ะได้คือ $\frac{1}{2}H_4=\frac{25}{24}$ หน่วย ซึ่งมากกว่าความยาวไม้หนึ่งชิ้น กล่าวคือ หากใช้ไม้ 4 ชิ้น ไม้ชิ้นบนสุดจะสามารถยื่นพ้นขอบโต๊ะมาทั้งชิ้นได้

เราสามารถใช้ integration bounds ในการประมาณค่าของ $H_n$ ได้ \enskip ทั้งนี้ สังเกตว่า $f(i)=1/i$ เป็นฟังก์ชันบวกและลด \enskip ก่อนอื่น \[\int_1^n\frac{1}{x}dx=\left.\ln x\right|_1^n=\ln n\] ดังนั้น \[\frac{1}{n}+\ln n \leq H_n \leq 1+\ln n\] กล่าวคือ $H_n\sim\ln n$ (โดยค่าความคลาดเคลื่อนมีไม่เกิน $1-1/n\leq 1$)

เนื่องจาก $\lim_{n\to\infty}{\ln n}=\infty$ จะได้ว่า $\lim_{n\to\infty}{H_n}=\infty$ ด้วยเช่นกัน กล่าวคือ harmonic numbers มีค่าไม่จำกัด ดังนั้น เราสามารถซ้อนกองไม้ให้ชิ้นบนสุดยื่นออกมาจากขอบโต๊ะเป็นระยะทางเท่าใดก็ได้ \enskip หากต้องการทราบว่าถ้าใช้ไม้หนึ่งล้านชิ้น จะสามารถวางให้ยื่นออกมาจากขอบโต๊ะได้เป็นระยะทางเท่าใด ก็ให้คำนวณค่าของ $H_{1{,}000{,}000}$ จาก integration bounds ข้างต้น ซึ่งจะได้ประมาณ $\ln{1{,}000{,}000}=14.3927\!\ldots$ \enskip นั่นคือ หากใช้ไม้ซ้อนกันหนึ่งล้านชิ้น ชิ้นบนสุดจะสามารถยื่นออกไปจากขอบโต๊ะได้เพียง 7 ช่วงไม้เท่านั้น
