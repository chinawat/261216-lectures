\section{Counting rules}
\emph{คณิตศาสตร์เชิงการจัด} (combinatorics) เป็นแขนงหนึ่งในวิชา discrete mathematics ที่เป็นเรื่องเกี่ยวกับการนับ โดยมีกฎต่างๆ ที่เป็นพื้นฐานในการนับดังนี้

\subsection{Product rule}
\begin{example}\label{ex:meal-combos}
สมมุติว่าในแต่ละวัน เรารับประทานอาหารทั้งหมดสามมื่อ ได้แก่ มื้อเช้า มื้อกลางวัน และมื้อเย็น \enskip ในแต่ละมื้อ เรามีตัวเลือกอาหารต่างๆ ดังนี้
\begin{itemize}[]
\item เช้า เลือกจาก แพนเค้ก ไข่ดาว ข้าวต้ม และ Oreo
\item กลางวัน เลือกจาก ก๋วยเตี๋ยว ไข่เจียว และ Oreo
\item เย็น เลือกจาก พิซซ่า สเต๊ก ผัดผัก ผลไม้ และ Oreo
\end{itemize}
จากตัวเลือกดังกล่าว เราสามารถเลือกรับประทานอาหารในแต่ละวันได้แตกต่างกันทั้งหมดกี่แบบ

จะเห็นว่า ในมื้อเช้า เรามีอาหาร 4 ชนิดให้เลือก \enskip ในแต่ละตัวเลือกสำหรับอาหารเช้า เราสามารถเลือกอาหารกลางวันได้ 3 ชนิด \enskip ในท้ายที่สุด ในแต่ละตัวเลือกสำหรับอาหารเช้าและอาหารกลางวัน เราสามารถเลือกอาหารเย็นได้อีก 5 ชนิด \enskip รายการต่อไปนี้เป็นตัวอย่างวิธีการเลือกที่แตกต่างกัน 3 แบบ
\begin{itemize}[]
\item (แพนเค้ก, ไข่เจียว, พิซซ่า)
\item (ข้าวต้ม, ก๋วยเตี๋ยว, ผัดผัก)
\item (Oreo, Oreo, ผลไม้)
\end{itemize}

ในกรณีทั่วไป วิธีการเลือกรับประทานอาหารในแต่ละวันนั้น สามารถเขียนเป็นลำดับความยาว 3 ดังนี้
\[(\uline{\hspace*{1cm}},\uline{\hspace*{1cm}},\uline{\hspace*{1cm}})\]
โดยตำแหน่งแรกเป็นตัวเลือกของอาหารเช้า (4 แบบ) ตำแหน่งที่สองเป็นตัวเลือกของอาหารกลางวัน (3 แบบ) และตำแหน่งสุดท้ายเป็นตัวเลือกของอาหารเย็น (5 แบบ) \enskip ดังนั้น จำนวนลำดับทั้งหมดที่เป็นไปได้ ซึ่งก็คือจำนวนวิธีเลือกรับประทานอาหารในแต่ละวันที่แตกต่างกันนั้น คือ $4\cdot 3\cdot 5=60$
\end{example}

\begin{theorem}[\emph{กฎการคูณ} (product rule)]
\newcommand\card[1]{\ensuremath\left|#1\right|}
ถ้า $A_1,A_2,\ldots,A_n$ เป็นเซตจำกัด แล้ว
\[\card{A_1\times A_2\times \cdots\times A_n}=\card{A_1}\cdot\card{A_2}\cdot \cdots \cdot \card{A_n}\]
\end{theorem}
ในตัวอย่างข้างต้น $A_1$ คือเซตของตัวเลือกอาหารเช้า $A_2$ คือเซตของตัวเลือกอาหารกลางวัน และ $A_3$ คือเซตของตัวเลือกอาหารเย็น

\subsection{Sum rule}
\begin{example}
สมมุติว่ามีนักศึกษาวิศวกรรมคอมพิวเตอร์ชั้นปีที่ 2 จำนวน 85 คน นักศึกษาวิศวกรรมคอมพิวเตอร์ชั้นปีที่ 3 จำนวน 90 คน และนักศึกษาวิศวกรรมสารสนเทศและเครือช่ายจำนวน 70 คน \enskip มีนักศึกษารวมกี่คนที่เป็นนักศึกษาวิศวกรรมคอมพิวเตอร์ชั้นปีที่ 2 หรือ 3 หรือนักศึกษาวิศวกรรมสารสนเทศและเครือข่าย

จะเห็นว่า นักศึกษาคนใดๆ สามารถเป็นได้ (อย่างมาก) หนึ่งในสามหมวดข้างต้น \enskip กล่าวอีกนัยหนึ่ง ไม่มีนักศึกษาใดๆ ที่เข้าข่ายมากกว่าหนึ่งหมวดข้างต้นเลย \enskip ดังนั้น จำนวนนักศึกษาทั้งหมด ก็คือผลรวมของจำนวนนักศึกษาในแต่ละหมวด นั่นคือ $85+90+70=245$
\end{example}

\begin{theorem}[\emph{กฎการบวก} (sum rule)]
\newcommand\card[1]{\ensuremath\left|#1\right|}
ถ้า $A_1,A_2,\ldots,A_n$ เป็นเซตจำกัดที่\emph{ไม่มีส่วนร่วมทุกคู่} (pairwise disjoint) กล่าวคือ $\forall i,j: i\neq j\implies A_i\cap A_j=\emptyset$ แล้ว
\[\card{A_1\cup A_2\cup \cdots\cup A_n}=\card{A_1}+\card{A_2}+ \cdots + \card{A_n}\]
\end{theorem}
ในตัวอย่างข้างต้น $A_1$ คือเซตของนักศึกษาวิศวกรรมคอมพิวเตอร์ชั้นปีที่ 2 $A_2$ คือเซตของนักศึกษาวิศวกรรมคอมพิวเตอร์ชั้นปีที่ 3 และ $A_3$ คือเซตของนักศึกษาวิศวกรรมสารสนเทศและเครือข่าย \enskip จะเห็นว่า ในกลุ่มของเซตข้างต้นนั้น ไม่มีเซตคู่ใดๆ ที่มีสมาชิกร่วมกันเลย กล่าวคือ เซตทั้งหมดนี้มีคุณสมบัติ pairwise disjoint

แต่ถ้าเราเปลี่ยนเซตข้างต้นดังนี้ $A_1$ คือเซตของนักศึกษาวิศวกรรมคอมพิวเตอร์ชั้นปีที่ 2 $A_2$ คือเซตของนักศึกษาวิศวกรรมคอมพิวเตอร์ที่เป็นชาย และ $A_3$ คือเซตของนักศึกษาวิศวกรรมสารสนเทศและเครือข่าย จะเห็นว่า จำนวนนักศึกษารวมทั้งหมด ที่เป็นนักศึกษาวิศวกรรมคอมพิวเตอร์ชั้นปีที่ 2 หรือเป็นชาย หรือเป็นนักศึกษาวิศวกรรมสารสนเทศและเครือข่ายนั้น ไม่สามารถคำนวณได้จากการนำจำนวนสมาชิกในแต่ละเซตมาบวกกัน เนื่องจากอาจจะมีนักศึกษาที่เป็นชาย และเป็นนักศึกษาวิศวกรรมคอมพิวเตอร์ชั้นปีที่ 2 ซึ่งอยู่ในทั้ง $A_1$ และ $A_2$ \enskip ในกรณีนี้ เราไม่สามารถใช้กฎการบวกได้

\begin{example}
จาก Example~\ref{ex:meal-combos} เราสามารถเลือกรับประทานอาหารได้กี่แบบ หากในแต่ละวัน ต้องรับประทานอาหารที่ไม่ซ้ำอย่างกัน

จะเห็นว่า Oreo เป็นอาหารอย่างเดียวที่สามารถรับประทานซ้ำหลายมื้อได้ ดังนั้น หากจะหาจำนวนวิธีที่รับประทานอาหารไม่ซ้ำอย่างกัน จะต้องรับประกันให้ได้ว่า เราจะรับประทาน Oreo ได้อย่างมากเพียงมื้อเดียวเท่านั้น ซึ่งสามารถแยกกรณีได้ดังนี้
\begin{itemize}[]
\item ไม่ทาน Oreo เลย: จะได้ว่า มื้อเช้าเลือกทานได้ 3 อย่าง มื้อกลางวันเลือกทานได้ 2 อย่าง และมื้อเย็นเลือกทานได้ 4 อย่าง \enskip ดังนั้น จำนวนวิธีเลือกทานอาหารที่เป็นไปได้คือ $3\cdot 2\cdot 4=24$ (โดยใช้กฎการคูณ)
\item ทาน Oreo มื้อเช้าเท่านั้น: จะได้ว่า มื้อเช้าเลือกทานได้ 1 อย่าง (คือ Oreo) มื้อกลางวันเลือกทานได้ 2 อย่าง และมื้อเย็นเลือกทานได้ 4 อย่าง \enskip ดังนั้น จำนวนวิธีเลือกทานอาหารที่เป็นไปได้คือ $1\cdot 2\cdot 4=8$ (โดยใช้กฎการคูณเช่นกัน)
\item ทาน Oreo มื้อกลางวันเท่านั้น: จะได้ว่า มื้อเช้าเลือกทานได้ 3 อย่าง มื้อกลางวันเลือกทานได้ 1 อย่าง (คือ Oreo) และมื้อเย็นเลือกทานได้ 4 อย่าง \enskip ดังนั้น จำนวนวิธีเลือกทานอาหารที่เป็นไปได้คือ $3\cdot 1\cdot 4=12$ (โดยใช้กฎการคูณเช่นกัน)
\item ทาน Oreo มื้อเย็นเท่านั้น: จะได้ว่า มื้อเช้าเลือกทานได้ 3 อย่าง มื้อกลางวันเลือกทานได้ 2 อย่าง และมื้อเย็นเลือกทานได้ 1 อย่าง (คือ Oreo) \enskip ดังนั้น จำนวนวิธีเลือกทานอาหารที่เป็นไปได้คือ $3\cdot 2\cdot 1=6$ (โดยใช้กฎการคูณอีกเช่นกัน)
\end{itemize}
เนื่องจากทั้งสี่กรณีข้างต้นนั้นไม่ซ้อนทับกันเลย เราจึงสามารถใช้กฎการบวกคำนวณจำนวนวิธีทั้งหมดในการรับประทานอาหารในแต่ละวันให้ไม่ซ้ำอย่างกัน ได้ $24+8+12+6=50$ วิธี
\end{example}

ในครั้งที่แล้ว เราได้เรียนรู้กฎการนับบางส่วน \enskip ในครั้งนี้ จะกล่าวถึงกฎการนับอื่นๆ ที่สำคัญ \enskip ก่อนที่จะเข้าสู่กฎการนับกฎต่อไป จะทบทวนการใช้กฎการนับที่เราได้เรียนรู้จากครั้งที่แล้วก่อน
%
\begin{example}
เลขสามหลักที่แต่ละหลักเป็นเลขใดก็ได้ตั้งแต่ 1 ถึง 9 มีทั้งหมดกี่ตัว

จะเห็นว่า ในแต่ละหลัก เรามีตัวเลือกสำหรับตัวเลขในหลักนั้นๆ ทั้งหมด 9 ตัวเลือก \enskip เนื่องจากเรามีสามหลักที่ต้องเลือก จำนวนตัวเลขทั้งหมดที่เป็นไปได้คือ $9\cdot 9\cdot 9=9^3$

ในตัวอย่างข้างต้นนี้ เราใช้กฎการคูณในการนับ เนื่องจากกระบวนการเลือกตัวเลขในแต่ละหลักนั้นเป็นขั้นตอนที่เป็นลำดับต่อเนื่องกัน
\end{example}

\subsection{Generalized product rule}

\begin{example}\label{ex:digit-number-count-norep}
จากตัวอย่างที่แล้ว หากเราเพิ่มเงื่อนไขด้วยว่า ต้องไม่มีเลขหลักใดที่ซ้ำกันเลย จะมีเลขที่เป็นไปได้ทั้งหมดกี่ตัว

จะเห็นว่า การเลือกเลขในหลักแรกจะมีผลต่อจำนวนตัวเลือกที่เรามีในการเลือกเลขในหลักที่สอง กล่าวคือ ในหลักแรก เรามีตัวเลือกทั้งหมด 9 ตัวเลือก \enskip จากนั้น ในหลักที่สอง เราไม่สามารถเลือกตัวเลขเดียวกับที่เราเลือกไปแล้ว จึงมีตัวเลือกเพียง 8 ตัวเลือก \enskip ในท้ายที่สุด การเลือกตัวเลขสำหรับหลักสุดท้าย จะไม่สามารถเลือกตัวเลขที่เราเลือกใช้ในสองหลักแรกได้อีก จึงเหลือเพียง 7 ตัวเลือกเท่านั้น

ดังนั้น จำนวนตัวเลขทั้งหมดที่ไม่มีเลขหลักใดซ้ำกันเลยคือ $9\cdot 8\cdot 7=9\cdot 8\cdot 7\cdot\frac{6\cdot 5\cdot \cdots \cdot 1}{6\cdot 5\cdot \cdots \cdot 1}=\frac{9!}{6!}$
\end{example}

\begin{theorem}[\emph{กฎการคูณแบบทั่วไป} (generalized product rule)]
ให้ $S$ เป็นเซตของลำดับที่แต่ละลำดับมีความยาว $k$ (กล่าวคือ $S$ เป็นเซตของ $k$-tuples) \enskip ถ้า
\begin{itemize}[]
\item ตำแหน่งแรกของลำดับ เป็นไปได้ $m_1$ แบบ
\item ตำแหน่งที่สองของลำดับ เป็นไปได้ $m_2$ แบบ
\item[$\vdots$]
\item ตำแหน่งที่ $k$ ของลำดับ เป็นไปได้ $m_k$ แบบ
\end{itemize}
แล้ว $|S|=m_1\cdot m_2\cdot \cdots \cdot m_k$
\end{theorem}

ใน Example~\ref{ex:digit-number-count-norep} $S$ คือเซตของลำดับซึ่งแต่ละลำดับมีความยาว 3 โดยตำแหน่งแรกของลำดับคือเลขหลักแรก ตำแหน่งที่สองของลำดับคือเลขหลักที่สอง และตำแหน่งสุดท้ายของลำดับคือเลขหลักที่สาม \enskip ตัวอย่างเช่น สมาชิก $(2, 1, 6)$ ใน $S$ จะสามารถตีความได้เป็นเลข 216 \enskip จะเห็นว่า ตำแหน่งแรกเป็นไปได้ 9 แบบ ตำแหน่งที่สองเป็นไปได้ 8 แบบ และตำแหน่งสุดท้ายเป็นได้ไป 7 แบบ

\begin{example}
พิจารณาตารางหมากรุกขนาด $8\times 8$ \enskip หากต้องการวางเบี้ย ม้า และเรือ โดยที่ไม่มีสองชิ้นใดๆ อยู่ในแถวหรือคอลัมน์เดียวกัน จะสามารถทำได้ทั้งหมดกี่วิธี

เราสามารถวางหมากครั้งละตัวได้ดังนี้ \enskip เริ่มแรก วางเบี้ยในตารางที่ยังว่างอยู่ ซึ่งมีทั้งหมด 64 ช่องที่สามารถเลือกได้ \enskip เมื่อวางเบี้ยแล้ว จะเห็นว่าเราจะไม่สามารถวางหมากที่เหลือในแถวหรือคอลัมน์เดียวกันกับเบี้ยได้ (ช่องที่แรเงาสีแดงในรูปกลาง) จึงเหลือเพียง 49 ช่องที่สามารถเลือกได้ \enskip สมมุติว่าในลำดับถัดไป เราเลือกที่จะวางม้า เมื่อวางแล้ว จะเห็นว่าเราจะไม่สามารถวางเรือในแถวหรือคอลัมน์เดียวกันกับเบี้ยหรือม้าได้ (ช่องที่แรเงาสีแดงหรือสีฟ้าในรูปขวา) จึงเหลือเพียง 36 ช่องที่เราสามารถวางเรือได้ \enskip ดังนั้น จำนวนวิธีวางหมากทั้งหมดที่เป็นไปได้ คือ $64\cdot 49 \cdot 36$
\begin{center}
\begin{tikzpicture}[scale=0.6]
\draw[shift={(0.5,0.5)}] (0,0) grid +(8,8);
\end{tikzpicture}
%
\quad
%
\begin{tikzpicture}[scale=0.6]
\draw[shift={(0.5,0.5)}] (0,0) grid +(8,8);
\draw[shift={(-0.5,0.5)},fill=red,opacity=0.2] (2,0) rectangle (3,8);
\draw[shift={(0.5,-0.5)},fill=red,opacity=0.2] (0,3) rectangle (8,4);
\node at (2,3) {บ};
\end{tikzpicture}
%
\quad
%
\begin{tikzpicture}[scale=0.6]
\draw[shift={(0.5,0.5)}] (0,0) grid +(8,8);
\draw[shift={(-0.5,0.5)},fill=red,opacity=0.2] (2,0) rectangle (3,8);
\draw[shift={(0.5,-0.5)},fill=red,opacity=0.2] (0,3) rectangle (8,4);
\draw[shift={(-0.5,0.5)},fill=blue,opacity=0.2] (3,0) rectangle (4,8);
\draw[shift={(0.5,-0.5)},fill=blue,opacity=0.2] (0,4) rectangle (8,5);
\node at (2,3) {บ};
\node at (3,4) {ม};
\end{tikzpicture}
\end{center}

อย่างไรก็ดี เราสามารถแก้ปัญหาข้างต้นได้อีกวิธีหนึ่ง โดยพิจารณาจำนวนแถวและคอลัมน์ที่เราสามารถวางหมากแต่ละตัว กล่าวคือ วิธีการวางหมากใดๆ สามารถเขียนเป็นลำดับที่มีความยาว 6 ซึ่งแต่ละตำแหน่งมีความหมายดังนี้
\begin{enumerate}[]
\item จำนวนแถวที่สามารถวางเบี้ยได้
\item จำนวนคอลัมน์ที่สามารถวางเบี้ยได้
\item จำนวนแถวที่สามารถวางม้าได้
\item จำนวนคอลัมน์ที่สามารถวางม้าได้
\item จำนวนแถวที่สามารถวางเรือได้
\item จำนวนคอลัมน์ที่สามารถวางเรือได้
\end{enumerate}
จะเห็นว่า จำนวนแถวที่สามารถวางเบี้ยได้ คือ 8 และจำนวนคอลัมน์ที่สามารถวางเบี้ยได้ คือ 8 เช่นกัน \enskip จากนั้น เมื่อวางเบี้ยไปแล้ว จำนวนแถวและคอลัมน์ที่สามารถวางม้าได้ เป็น 7 ทั้งคู่ \enskip ในท้ายที่สุด จะเหลือจำนวนแถวและคอลัมน์ให้วางเรือได้ทั้งหมด 6 แถว และ 6 คอลัมน์ \enskip ดังนั้น จำนวนวิธีวางหมากทั้งหมดคือ $8\cdot 8\cdot 7\cdot 7\cdot 6\cdot 6=8^2\cdot 7^2\cdot 6^2$ ซึ่งเท่ากับค่าที่ได้ก่อนหน้านี้
\end{example}

\subsection{Division rule}

\begin{example}
พิจารณาตารางหมากรุกขนาด $8\times 8$ \enskip หากต้องการวางเรือ 2 ตัว ให้ไม่อยู่ในแถวหรือคอลัมน์เดียวกัน จะสามารถทำได้ทั้งหมดกี่วิธี

ในข้อนี้ เราสามารถใช้เหตุผลในลักษณะเดียวกันกับตัวอย่างที่แล้ว โดยนับจำนวนลำดับที่มีความยาว 4 ซึ่งแต่ละตำแหน่งมีความหมายดังนี้
\begin{enumerate}[]
\item จำนวนแถวที่สามารถวางเรือตัวแรกได้
\item จำนวนคอลัมน์ที่สามารถวางเรือตัวแรกได้
\item จำนวนแถวที่สามารถวางเรือตัวที่สองได้
\item จำนวนคอลัมน์ที่สามารถวางเรือตัวที่สองได้
\end{enumerate}
นั่นคือ จำนวนแถวที่สามารถวางเรือตัวแรกได้ คือ 8 และจำนวนคอลัมน์ที่สามารถวางเรือตัวแรกได้คือ 8 เช่นกัน \enskip จากนั้น เมื่อวางเรือตัวแรกไปแล้ว จำนวนแถวและคอลัมน์ที่สามารถวางเรือตัวที่สองได้ เป็น 7 ทั้งคู่ \enskip ดังนั้น จำนวนวิธีวางเรือทั้งหมดควรจะเป็น $8^2\cdot 7^2$

อย่างไรก็ดี หากพิจารณาให้ถี่ถ้วน จะเห็นว่า หากเราวางเรือสองตัว ณ ตำแหน่ง $(8,a)$ และ $(5,d)$ ดังรูป จะมีลำดับที่สอดคล้องกับวิธีนี้ทั้งหมดสองลำดับ ได้แก่ $(8,a,5,d)$ และ $(5,d,8,a)$ \enskip โดยในลำดับแรกนั้น เราเลือกที่จะวางเรือตัวแรก ณ ตำแหน่ง $(8,a)$ และตัวที่สอง ณ ตำแหน่ง $(5,d)$ (ดังรูปด้านซ้าย) \enskip ส่วนในลำดับที่สองนั้น เราเลือกที่จะวางเรือตัวแรก ณ ตำแหน่ง $(5,d)$ และตัวที่สอง ณ ตำแหน่ง $(8,a)$ (ดังรูปด้านขวา)
%
\begin{center}
\begin{tikzpicture}[scale=0.6]
\draw[shift={(0.5,0.5)}] (0,0) grid +(8,8);
\draw[shift={(-0.5,0.5)},fill=red,opacity=0.2] (1,0) rectangle (2,8);
\draw[shift={(0.5,-0.5)},fill=red,opacity=0.2] (0,1) rectangle (8,2);
\node at (1,1) {ร$_1$};
\node at (4,4) {ร$_2$};

\foreach \j [count=\i from 1] in {a,...,h} {
  \node at (0,9-\i) {\i};
  \node[anchor=base] at (\i,9) {\j};
}
\end{tikzpicture}
%
\quad
%
\begin{tikzpicture}[scale=0.6]
\draw[shift={(0.5,0.5)}] (0,0) grid +(8,8);
\draw[shift={(-0.5,0.5)},fill=red,opacity=0.2] (4,0) rectangle (5,8);
\draw[shift={(0.5,-0.5)},fill=red,opacity=0.2] (0,4) rectangle (8,5);
\node at (1,1) {ร$_2$};
\node at (4,4) {ร$_1$};

\foreach \j [count=\i from 1] in {a,...,h} {
  \node at (0,9-\i) {\i};
  \node[anchor=base] at (\i,9) {\j};
}
\end{tikzpicture}
\end{center}
%
แต่จะเห็นว่า แม้ว่าเราจะเลือกวางเรือตัวแรก ณ ตำแหน่งที่ต่างกันในสองวิธีข้างต้น แต่ผลลัพธ์สุดท้ายที่ได้ ก็เป็นวิธีการวางเรือสองตัวบนกระดานที่เหมือนกันอยู่ดี กล่าวคือ ลำดับสองลำดับข้างต้น แสดงวิธีวางหมากที่เหมือนกัน \enskip ดังนั้น จำนวนวิธีวางหมากทั้งหมดที่เป็นไปได้ ควรจะเป็นครึ่งหนึ่งของจำนวนลำดับที่มีอยู่ \enskip เนื่องจากจำนวนลำดับที่เป็นไปได้คือ $8^2\cdot 7^2$ จำนวนวิธีวางหมากที่ถูกต้องจึงเป็น $\frac{8^2\cdot 7^2}{2}$
\end{example}

ในตัวอย่างข้างต้น เราสามารถสร้างฟังก์ชันที่รับค่าเป็นลำดับ และให้ผลลัพธ์เป็นวิธีการวางเรือสองตัวบนกระดานหมากรุก ซึ่งจากข้อสังเกตข้างต้น จะมีลำดับสองลำดับ ที่เมื่อส่งให้ฟังก์ชันดังกล่าวนี้แล้ว จะให้ผลลัพธ์เป็นวิธีเดียวกัน

\begin{definition}
ฟังก์ชัน $f:A\to B$ ซึ่งมีคุณสมบัติ total และ surjective เป็น \emph{$k$-to-1 mapping} หากสมาชิก $k$ ตัวใน domain ชี้ไปหาสมาชิก 1 ตัวใน codomain
\end{definition}

\noindent{\bf Remark}: ฟังก์ชัน $f:A\to B$ มีคุณสมบัติ total หาก $\forall a\in A\exists b\in B: f(a)=b$

\noindent{\bf Remark}: ฟังก์ชัน $f:A\to B$ มีคุณสมบัติ surjective หาก $\forall b\in B\exists a\in A: f(a)=b$

จากตัวอย่างข้างต้น domain ของฟังก์ชันคือเซตของลำดับความยาว 4 ที่แต่ละตำแหน่งเป็นจำนวนแถวหรือคอลัมน์ที่สามารถวางเรือแต่ละตัวได้ และ codomain คือเซตของวิธีการวางเรือสองตัวให้ไม่ซ้ำแถวหรือคอลัมน์ \enskip จากการให้เหตุผลก่อนหน้านี้ ฟังก์ชันดังกล่าวเป็น 2-to-1 mapping

\begin{example}
หากเรามีฟังก์ชันดังแผนภาพต่อไปนี้ ฟังก์ชันดังกล่าวจะเป็น 3-to-1 mapping
\begin{center}
\begin{tikzpicture}[xscale=3,yscale=0.5]
\foreach \i in {1,...,9} {
  \node (a\i) at (0,\i) {\textbullet};
}
\foreach \i in {2,5,8} {
  \node (b\i) at (1,\i) {\textbullet};
}
\draw[arrow] (a1) -- (b2);
\draw[arrow] (a2) -- (b2);
\draw[arrow] (a3) -- (b2);
\draw[arrow] (a4) -- (b5);
\draw[arrow] (a5) -- (b5);
\draw[arrow] (a6) -- (b5);
\draw[arrow] (a7) -- (b8);
\draw[arrow] (a8) -- (b8);
\draw[arrow] (a9) -- (b8);
\end{tikzpicture}
\end{center}
\end{example}

\begin{example}
พิจารณาฟังก์ชันที่รับค่าเป็นแขนของมนุษย์ และให้ผลลัพธ์เป็นเจ้าของของแขนที่ให้มา \enskip หากเราตั้งสมมุติฐานว่ามนุษย์ทุกคนมีแขนสองข้าง จะได้ว่า ฟังก์ชันดังกล่าวเป็น 2-to-1 mapping
\end{example}

\begin{theorem}[\emph{กฎการหาร} (division rule)]
ถ้า $f:A\to B$ เป็น $k$-to-1 mapping แล้ว $|A|=k|B|$
\end{theorem}

จากตัวอย่างจำนวนวิธีวางเรือสองตัวในตารางหมากรุกก่อนหน้านี้ จะเห็นว่า $|A|=2|B|$ กล่าวคือ จำนวนวิธีวางเรือสองตัวไม่ให้ซ้ำแถวหรือคอลัมน์กัน จะเป็นครึ่งหนึ่งของจำนวนลำดับความยาว 4 ที่ระบุตำแหน่งของเรือตัวที่หนึ่งและตัวที่สอง

\begin{example}
มีคนอยู่ $n$ คน ต้องการจัดคนกลุ่มนี้ให้นั่งล้อมรอบโต๊ะกลม จะสามารถทำได้กี่วิธี

วิธีการจัดคนนั่งโต๊ะกลมแบบหนึ่งที่ทำได้ คือการเลือกคนไปนั่งเก้าอี้ที่อยู่รอบๆ โต๊ะ โดยให้คนที่ถูกเลือกถัดจากคนปัจจุบันนั่งทางด้านขวาของคนปัจจุบัน \enskip ตัวอย่างเช่น เราอาจจะเลือกคนที่ 1 ไปนั่งเป็นคนแรก ตามด้วยคนที่สอง ซึ่งจะนั่งทางด้านขวาของคนแรก ไปเรื่อยๆ ตามลำดับ จนเหลือคนที่ $n$ เป็นคนไปนั่งเก้าอี้ที่เหลือตัวสุดท้าย \enskip เราสามารถเขียนหมายเลขของคนที่ถูกเลือกไปนั่งเก้าอี้ตามลำดับได้เป็น $(1,2,\ldots,n)$ \enskip จะเห็นว่า ตัวอย่างของลำดับในการเลือกคนไปนั่งเก้าอี้ เป็นไปได้ดังนี้
\begin{itemize}[]
\item $(1,2,\ldots,n-1,n)$
\item $(2,3,\ldots,n,1)$
\item[$\vdots$]
\item $(n-1,n,\ldots,n-3,n-2)$
\item $(n,1,\ldots,n-2,n-1)$
\end{itemize}
ลำดับต่างๆ ในรายการข้างต้นมีรูปแบบเดียวกัน กล่าวคือ เป็นการเลือกคนที่หมายเลขสูงขึ้นเรื่อยๆ ไปนั่ง และหลังจากเลือกคนที่ $n$ ไปนั่งแล้ว ให้กลับไปเลือกคนที่ 1 เป็นคนถัดไป จนกว่าจะครบ \enskip จะเห็นว่า จำนวนลำดับดังกล่าวที่เป็นไปได้ มีทั้งหมด $n$ ลำดับ \enskip อย่างไรก็ดี ลำดับทั้งหมดในรายการข้างต้นนั้นแสดงวิธีการนั่งแบบเดียวกัน กล่าวคือ ทางด้านขวาของคนหมายเลข $i$ จะเป็นคนหมายเลข $i+1$ และด้านขวาของคนหมายเลข $n$ จะเป็นคนหมายเลข 1 เสมอ \enskip หากจะเขียนเป็นแผนภาพ จะได้ว่า ไม่ว่าจะเลือกคนไปนั่งอย่างไรก็ตามด้วยลำดับข้างต้น จะได้ลำดับของคนรอบโต๊ะตามรูป
\begin{center}
\begin{tikzpicture}
\draw (0,0) circle (1);
\foreach \i/\j/\k in {0/1/0,1/2/0,2/3/0,4/{\cdots}/-67.5,6/{n-1}/22.5,7/n/0} {
  \node[rotate=\k] at ($(-\i/8*360+22.5:1.25)$) {$\j$};
}
% TODO
\end{tikzpicture}
\end{center}
เนื่องจาก $n$ ลำดับข้างต้นนั้นแสดงวิธีการนั่งแบบเดียวกัน ฟังก์ชันจากลำดับการเลือกคนไปนั่งเก้าอี้รอบโต๊ะกลม ไปยังวิธีการนั่งรอบโต๊ะกลม จึงเป็น $n$-to-1 mapping \enskip ดังนั้น เราสามารถหาจำนวนวิธีการนั่งที่แตกต่างกันได้ หากทราบจำนวนลำดับที่มีความยาว $n$ ซึ่งไม่มีสมาชิกในลำดับที่ซ้ำกันเลย

จำนวนลำดับที่มีความยาว $n$ โดยที่ไม่มีสมาชิกใดๆ ในลำดับซ้ำกันเลยนั้น สามารถคำนวณได้จากกฎการคูณแบบทั่วไป โดยตำแหน่งแรกนั้น เรามีตัวเลือกทั้งหมด $n$ คนให้ไปนั่งเก้าอี้ตัวแรก \enskip จากนั้น จะเหลือ $n-1$ คนให้เลือกไปนั่งเก้าอี้ตัวที่สอง \enskip ทำอย่างนี้ไปเรื่อยๆ \enskip จนกระทั่งเหลือคนสุดท้ายเพียงคนเดียว ที่ต้องไปนั่งเก้าอี้ตัวสุดท้าย \enskip จะเห็นว่า จำนวนลำดับทั้งหมดที่เป็นไปได้คือ
\[n\cdot(n-1)\cdot(n-2)\cdot\cdots\cdot 1=n!\]
เนื่องจากฟังก์ชันข้างต้นเป็น $n$-to-1 mapping จะได้ว่า จำนวนวิธีจัดคน $n$ คนให้นั่งรอบโต๊ะกลมคือ $\frac{n!}{n}=(n-1)!$
\end{example}

\subsection{Subset rule}
พิจารณาเซตที่มีสมาชิก $n$ ตัว \enskip จำนวน subsets ที่มีสมาชิก $k$ ตัวของเซตข้างต้นเป็นเท่าใด

คำตอบของคำถามข้างต้น สามารถเขียนได้ในรูป $\binom{n}{k}$

\begin{definition}
$\binom{n}{k}$ คือจำนวน subsets ที่มีขนาด $k$ ของเซตที่มีขนาด $n$
\end{definition}

หากจะคำนวณค่าของ $\binom{n}{k}$ ออกมาเป็นตัวเลข เราสามารถทำได้ดังนี้ \enskip ก่อนอื่น เราจะทำการสร้าง subsets ที่เป็นไปได้จากสมาชิกในเซตตั้งต้น \enskip เริ่มแรก นำสมาชิกทั้ง $n$ ตัวของเซตมาเรียงแถว จากนั้น ให้เลือกสมาชิก $k$ ตัวแรกของแถวนี้มาอยู่ใน subset ที่เราต้องการ ส่วน $n-k$ ตัวหลังที่เหลือของแถวดังกล่าว จะไม่นำมาเป็นสมาชิกของ subset ที่เรากำลังสร้างขึ้น ดังรูป
\begin{center}
\begin{tikzpicture}[scale=0.75]
% TODO
\foreach \i in {1,2,5} {
  \node[circle,draw] at (\i,0) {};
}
\foreach \i in {6,7,10} {
  \node[circle,draw,red] at (\i,0) {};
}
\node at (3.5,0) {$\cdots$};
\node at (8.5,0) {$\cdots$};

\draw (0.75,0.375) -- (0.75,0.75) -- node[above=0.5ex] {$n$ ตัว} (10.25,0.75) -- (10.25,0.375);
\draw (0.75,-0.375) -- (0.75,-0.75) -- node[below=0.5ex] {$k$ ตัว} node[anchor=base,below=3ex] {เอาไว้ใน subset} (5.25,-0.75) -- (5.25,-0.375);
\draw (5.75,-0.375) -- (5.75,-0.75) -- node[below=0.5ex] {$n-k$ ตัว} node[anchor=base,below=3ex] {ไม่เอาไว้ใน subset}  (10.25,-0.75) -- (10.25,-0.375);
\end{tikzpicture}
\end{center}
จะเห็นว่า จำนวนลำดับของสมาชิก $n$ ตัวจากเซต มีทั้งสิ้น $n!$
%
\begin{example}
หาก $n=10$ และ $k=4$ จะได้ว่า ตัวอย่างของลำดับที่เป็นไปได้ คือ
\begin{itemize}[]
\item \str{1 3 5 7 \textcolor{red}{2 4 6 8 9 10}}
\item \str{1 3 5 7 \textcolor{red}{6 4 2 8 9 10}}
\item \str{7 3 5 1 \textcolor{red}{2 4 6 8 9 10}}
\end{itemize}
โดยสมาชิกสีแดงในแต่ละลำดับนั้นไม่อยู่ใน subset ที่เรากำลังสร้างขึ้น จะเห็นว่า subset ที่ได้นั้นคือ $\set{1,3,5,7}$ จากทุกลำดับข้างต้น
\end{example}

จากตัวอย่างข้างต้น จะเห็นว่า ไม่ว่าเราจะสลับที่สมาชิก $k$ ตัวแรกอย่างไรก็ตาม ผลลัพธ์ที่ได้ก็ยังคงเป็น subset เดิมอยู่ดี (สามารถทำได้ $k!$ วิธี) \enskip ในทำนองเดียวกัน ไม่ว่าเราจะสลับที่สมาชิก $n-k$ ตัวหลังอย่างไรก็ตาม ผลลัพธ์ที่ได้ก็ยังคงเป็น subset เดิมอยู่ดี (สามารถทำได้ $(n-k)!$ วิธี) \enskip ดังนั้น ในแต่ละลำดับที่ให้มานั้น เราสามารถสลับตำแหน่งสมาชิกในลำดับได้ $k!(n-k)!$ วิธี ก็ยังได้ลำดับที่ตีความได้เป็น subset เดิม \enskip กล่าวอีกนัยหนึ่ง ฟังก์ชันที่รับค่าเป็นลำดับที่มีความยาว $n$ และให้ผลลัพธ์เป็น subset ที่มีขนาด $k$ นั้นเป็น $(k!(n-k)!)$-to-1 mapping นั่นคือ $n!=k!(n-k)!\binom{n}{k}$ \enskip ในท้ายที่สุด เราสามารถใช้กฎการหารหาค่าของ $\binom{n}{k}$ ได้

\begin{theorem}[\emph{กฎเซตย่อย} (subset rule)]
\[\binom{n}{k}=\frac{n!}{k!(n-k)!}\]
\end{theorem}

\begin{example}
Binary strings ความยาว 20 ที่มีเลข \str{1} 5 ตัว มีทั้งหมดกี่ strings

เนื่องจากมี 20 ตำแหน่งที่ต้องเป็น \str{0} หรือ \str{1} โดยที่ต้องมี 5 ตำแหน่งที่ต้องเป็น \str{1} เราสามารถเลือก 5 ตำแหน่งที่ต้องเป็น \str{1} จากทั้งหมด 20 ตำแหน่งดังกล่าวได้ ซึ่งทำได้ $\binom{20}{5}$ วิธี \enskip กล่าวอีกนัยหนึ่ง จำนวน strings ทั้งหมดดังกล่าว ก็คือจำนวน subsets ของ $\set{1,2,\ldots,20}$ ที่มีขนาด 5 ซึ่งสามารถตีความได้เป็นตำแหน่งใน string ที่จะเป็น \str{1}
\end{example}
