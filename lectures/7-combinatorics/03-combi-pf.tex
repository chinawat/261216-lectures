\section{Combinatorial proofs}
\emph{การพิสูจน์เชิงการจัด} (combinatorial proof) เป็นวิธีการพิสูจน์สมการพีชคณิต (algebraic equation) โดยใช้กฎการนับเป็นวิธีการหลักในการให้เหตุผล \enskip การพิสูจน์เชิงการจัดประกอบไปด้วย 4 ขั้นตอน ได้แก่
\begin{enumerate}[]
\item นิยามเซต $S$ ที่จะนับจำนวนสมาชิก
\item ให้เหตุผลว่า $|S|=n$ ด้วยวิธีการนับแบบที่ 1
\item ให้เหตุผลว่า $|S|=m$ ด้วยวิธีการนับแบบที่ 2
\item เนื่องจาก $n$ และ $m$ ที่นับได้ข้างต้นเป็นจำนวนสมาชิกของ $S$ เหมือนกัน จึงสรุปได้ว่า $n=m$
\end{enumerate}

\begin{example}
ใน Example~\ref{ex:loop3-combi-proof} เราได้ใช้ combinatorial proof ในแต่ละขั้นตอนข้างต้นดังนี้
\begin{enumerate}[]
\item กำหนด $S=\set{(i,j,k)\mid 1\leq k\leq j\leq i\leq n}$
\item $|S|$ เป็นจำนวนวิธีซื้อของ 4 ชนิด รวม $n$ ชิ้น โดยชนิดแรกซื้ออย่างน้อย 1 ชิ้น
\item $|S|$ เป็นจำนวนวิธีซื้อของ 3 ชนิด รวม $i$ ชิ้น โดยชนิดแรกซื้อย่างน้อย 1 ชิ้น และ $1\leq i\leq n$
\item จึงสรุปได้ว่า $\binom{n+2}{3}=\sum_{i=1}^{n}{\binom{i+1}{2}}$
\end{enumerate}
\end{example}

\begin{example}
ถ้า $0\leq k\leq n$ แล้ว $\binom{n}{k}=\binom{n}{n-k}$
\begin{pf}
ให้ $S$ เป็นเซตของวิธีการเลือกสิ่งของ $k$ ชิ้น จากสิ่งของ $n$ ชิ้นที่แตกต่างกัน \enskip หากจะนับจำนวนสมาชิกของ $S$ เราต้องพิจารณาว่า จะสามารถเลือกสิ่งของ $k$ ชิ้นมาได้อย่างไรบ้าง ซึ่งสามารถทำได้สองวิธีการ ดังนี้
\begin{enumerate}[]
\item เลือกสิ่งของ $k$ ชิ้นที่จะใช้ จากทั้งหมด $n$ ชิ้น ซึ่งสามารถทำได้ $\binom{n}{k}$ วิธี
\item เลือกสิ่งของ $n-k$ ชิ้นที่จะ\emph{ไม่ใช้} จากทั้งหมด $n$ ชิ้น (กล่าวคือ ชิ้นที่เราไม่ได้เลือกจะเป็นชิ้นที่เราใช้) ซึ่งสามารถทำได้ $\binom{n}{n-k}$ วิธี
\end{enumerate}
จะเห็นว่า ทั้งสองวิธีข้างต้นนั้นให้ผลลัพธ์ที่เหมือนกัน กล่าวคือ ได้ของ $k$ ชิ้นที่เราต้องการเช่นกัน จึงสรุปได้ว่า จำนวนวิธีจากทั้งสองวิธีการข้างต้นจำเป็นต้องเท่ากัน นั่นคือ $\binom{n}{k}=\binom{n}{n-k}$
\end{pf}
\end{example}

\begin{example}[Pascal's triangle]
สามเหลี่ยมปาสคาล (Pascal's triangle) เป็นโครงสร้างทางคณิตศาสตร์ที่ใช้แสดงความสัมพันธ์ระหว่าง binomial coefficients ต่างๆ ดังรูป
\begin{center}
\begin{tikzpicture}
\foreach \n in {0,...,4} {
  \node[color=red] (n-\n) at (-4,-\n) {$n=\n$};
  \draw[arrow,color=red!50] (-3,-\n-1/4) -- (\n-\n/2+1,-\n-1/4);
  
  \node[color=blue] (k-\n) at (\n-\n/2+1,-\n+1/2) {$k=\n$};
  \draw[arrow,color=blue!50] (\n-\n/2+1/2,-\n+1/4) -- (\n-2,-4-3/4);
}
\foreach \n/\k/\v in {0/0/1,1/0/1,1/1/1,2/0/1,2/1/2,2/2/2,3/0/1,3/1/3,3/2/3,3/3/1,4/0/1,4/1/4,4/2/6,4/3/4,4/4/1} {
  \node (b-\n-\k) at (\k-\n/2,-\n) {$\v$};
}
\draw[arrow] (b-2-0) -- (b-3-1);
\draw[arrow] (b-2-1) -- (b-3-1);
\draw[arrow] (b-3-0) -- (b-4-1);
\draw[arrow] (b-3-1) -- (b-4-1);
\draw[arrow] (b-3-1) -- (b-4-2);
\draw[arrow] (b-3-2) -- (b-4-2);
\end{tikzpicture}
\end{center}
แต่ละตำแหน่งของ Pascal's triangle นั้นเป็นค่าของ binomial coefficient $\binom{n}{k}$ ตามตำแหน่งในรูป \enskip โดยแถวหนึ่งๆ จะมีค่า $n$ เป็นค่าเดียวกัน และแนวทแยงลงจากขวาไปซ้าย จะมีค่า $k$ เป็นค่าเดียวกัน ตัวอย่างเช่น $\binom{4}{2}=6$ จากรูปข้างต้น

นอกจากนี้ ค่าในแต่ละตำแหน่งของ Pascal's triangle สามารถคำนวณได้เป็นผลรวมของค่าที่อยู่ติดกันจากแถวบน \enskip ตัวอย่งเช่น ค่าของ $\binom{4}{1}$ สามารถคำนวณได้จากผลรวมของ $\binom{3}{0}$ และ $\binom{3}{1}$ \enskip ในกรณีทั่วไป สามารถเขียนเป็นบทตั้งได้ดังนี้

\begin{lemma}[Pascals' triangle identity]
ถ้า $1\leq k\leq n$ แล้ว
\[\binom{n}{k}=\binom{n-1}{k-1}+\binom{n-1}{k}\]
\begin{pf}
By combinatorial argument. \enskip ให้ $S$ เป็นเซตของวิธีการเลือกทีมที่มีสมาชิก $k$ คน จากคนทั้งหมด $n$ คน \enskip ในการเลือกสมาชิกของทีมนั้น สามารถทำได้สองวิธี ดังนี้
\begin{enumerate}[]
\item เลือก $k$ คนที่ต้องการอยู่ในทีมจากคนทั้งหมด $n$ คน ซึ่งทำได้ $\binom{n}{k}$ วิธี
\item พิจารณาคนสุดท้าย (คนที่ $n$) ว่าจะเลือกมาอยู่ในทีมหรือไม่
\begin{itemize}[]
\item คนสุดท้ายอยู่ในทีม จะได้ว่า ยังเหลือ $k-1$ คนที่ต้องเลือกมาอยู่ในทีมจาก $n-1$ คนที่เหลือ ซึ่งจำนวนวิธีการเลือกคนที่เหลือนี้ทำได้ $\binom{n-1}{k-1}$ วิธี
\item คนสุดท้ายไม่อยู่ในทีม จะได้ว่า ยังเหลือ $k$ คนที่ต้องเลือกมาอยู่ในทีมจาก $n-1$ คนที่เหลือ ซึ่งจำนวนวิธีการเลือกคนที่เหลือนี้ทำได้ $\binom{n-1}{k}$ วิธี
\end{itemize}
เนื่องกจากวิธีการเลือกทีมใดๆ ต้องเข้าข่ายสองกรณีข้างต้น อย่างใดอย่างหนึ่งเท่านั้น จะได้ว่า จำนวนวิธีการเลือกทีมทั้งหมดที่เป็นไปได้คือ $\binom{n-1}{k-1}+\binom{n-1}{k}$
\end{enumerate}
ดังนั้น $\binom{n}{k}=\binom{n-1}{k-1}+\binom{n-1}{k}$
\end{pf}
\end{lemma}
\end{example}

\begin{example}
\[\binom{3n}{n}=\sum_{r=0}^{n}{\binom{n}{r}\binom{2n}{n-r}}\]
\begin{pf}
By combinatorial argument. \enskip ให้ $S$ เป็นเซตของวิธีการเลือกไพ่ $n$ ใบ จากสำรับไพ่ที่มีไพ่สีดำ $n$ ใบ และไพ่สีแดง $2n$ ใบ \enskip ในการเลือกไพ่ตามเงื่อนไขดังกล่าวนั้น สามารถทำได้สองวิธี ดังนี้
\begin{enumerate}[]
\item จะเห็นว่า มีไพ่ทั้งหมดรวม $3n$ ใบ โดยที่เราต้องเลือกมา $n$ ใบ ซึ่งสามารถทำได้ $\binom{3n}{n}$ วิธี \enskip ดังนั้น $|S|=\binom{3n}{n}$
\item อย่างไรก็ดี ไพ่ที่เราจะเลือกมานั้นสามารถแบ่งได้เป็นสองส่วน คือไพ่สีดำและไพ่สีแดง \enskip ในการเลือกไพ่นั้น เราสามารถเลือกไพ่สีดำมาก่อนจำนวน $r$ ใบ โดยที่ $0\leq r\leq n$ (กล่าวคือ จำนวนไพ่สีดำที่เลือกมานั้นต้องไม่เกินจำนวนไพ่สีดำที่มีอยู่) \enskip จากนั้น เลือกไพ่ $n-r$ ใบที่เหลือจากกองไพ่สีแดง ก็จะได้ไพ่มาทั้งหมด $n$ ใบตามที่เราต้องการ \enskip ในแต่ละขั้นตอนข้างต้น สามารถนับจำนวนวิธีได้ดังนี้
\begin{itemize}[]
\item เลือกไพ่สีดำ $r$ ใบ จากทั้งหมด $n$ ใบ สามารถทำได้ $\binom{n}{r}$ วิธี
\item เลือกไพ่สีแดง $n-r$ ใบ จากทั้งหมด $2n$ ใบ สามารถทำได้ $\binom{2n}{n-r}$ วิธี
\end{itemize}
จะได้ว่า จำนวนวิธีเลือกไพ่ $n$ ใบ โดยในกองที่เลือกมานั้นมีไพ่สีดำอยู่ $r$ ใบ คือ $\binom{n}{r}\binom{2n}{n-r}$ \enskip แต่เนื่องจาก $r$ นั้นเป็นค่าใดก็ได้ที่สอดคล้องกับเงื่อนไข $0\leq r\leq n$ จะได้ว่า จำนวนวิธีเลือกไพ่ $n$ ใบทั้งหมดที่เป็นไปได้ คือผลรวมของผลคูณข้างต้นจากทุกค่าของ $r$ ที่เป็นไปได้ กล่าวคือ \[|S|=\sum_{r=0}^{n}{\binom{n}{r}\binom{2n}{n-r}}\]
\end{enumerate}
เนื่องจากทั้งสองวิธีข้างต้นนั้นเป็นการนับจำนวนสมาชิกของ $S$ เหมือนกัน จึงสรุปได้ว่า
\[\binom{3n}{n}=\sum_{r=0}^{n}{\binom{n}{r}\binom{2n}{n-r}}\]
\end{pf}
\end{example}
