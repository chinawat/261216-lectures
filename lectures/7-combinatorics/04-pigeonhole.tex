\section{Pigeonhole Principle}
\begin{axiom}[\emph{หลักการช่องนกพิราบ} (Pigeonhole Principle)]
ถ้ามีนกพิราบมากกว่าจำนวนช่องให้อยู่ แล้วจะต้องมีช่องใดช่องหนึ่งที่มีนกพิราบอย่างน้อยสองตัว
\end{axiom}
หากจะเขียน axiom ข้างต้นเป็นเชิงรูปนัย จะได้ดังนี้
\begin{axiom}
ให้ $f:A\to B$ เป็น total function กล่าวคือ $\forall x\in A\exists y\in B: f(x)=y$ \enskip ถ้า $|A|>|B|$ แล้ว $\exists x_1,x_2\in A: x_1\neq x_2 \wedge f(x_1)=f(x_2)$
\end{axiom}

\begin{example}
ในห้องมืดมีถุงเท้าสีแดง, เขียว, และฟ้า จำนวนมาก \enskip เราจะต้องหยิบถุงเท้าอย่างน้อยกี่ข้างก่อนออกมาจากห้องมืด จึงจะมั่นใจได้ว่ามีคู่ที่มีสีเดียวกัน

หากเราเปรียบถุงเท้าแต่ละข้างเป็นนกพิราบแต่ละตัว และช่องนกพิราบเป็นสีของถุงเท้า จะได้ว่า มีช่องทั้งหมด 3 ช่อง \enskip จาก Pigeonhole Principle ข้างต้น จะได้ว่า หากเรามีถุงเท้า 4 ข้าง (กล่าวคือ นกพิราบ 4 ตัว) จะต้องมีถุงเท้าสองข้างที่มีสีเดียวกัน ซึ่งนำมาจับคู่กันได้ \enskip ทั้งนี้ ในกรณีที่เลวร้ายที่สุดนั้น เมื่อเราอยู่ในห้องมืด เราอาจจะหยิบถุงเท้าข้างแรกเป็นสีแดง ข้างที่สองเป็นสีเขียว และข้างที่สามเป็นสีฟ้า \enskip แต่หลังจากนั้น ไม่ว่าเราจะหยิบถุงเท้าสีอะไรมาก็ตาม จะต้องซ้ำกับข้างใดข้างหนึ่งที่เราเคยหยิบไปแล้ว

อย่างไรก็ดี หากเราเปลี่ยนเงื่อนไขว่า จำเป็นจะต้องได้ถุงเท้าสีแดง 1 คู่ จะเห็นว่า เราไม่สามารถรับประกันจำนวนถุงเท้าที่น้อยที่สุดที่ต้องหยิบออกมาได้เลย เนื่องจากเราอาจจะโชคร้ายหยิบถุงเท้าสีเขียวทุกครั้งเลยก็ได้
\end{example}

\begin{example}
มี DFA ที่มี $n$ states \enskip ถ้า DFA นี้ประมวลผล string ที่มีความยาว $n$ แล้วจะต้องมี state ที่ถูกใช้งานอย่างน้อย 2 ครั้ง

จากหลักการทำงานของ DFA หากประมวลผล string ที่มีความยาว $n$ จะได้ลำดับของ states ที่ DFA ใช้งานจากต้นจนจบ โดยที่ลำดับนี้มีความยาว $n+1$ \enskip ตัวอย่างเช่น หากประมวลผล string \str{a} จะได้ว่า states ที่ถูกใช้งานคือ start state $q_0$ และ state ถัดไปที่กำหนดโดย transition function $\delta(q_0,\str{a})$

ดังนั้น ในสถานการณ์นี้ แต่ละ state ที่ปรากฏในลำดับก็คือนกพิราบแต่ละตัว และ state ที่มีอยู่ใน DFA ก็คือช่องนกพิราบ \enskip เนื่องจากจำนวน states ในลำดับนั้นมีมากกว่าจำนวน states ที่มีอยู่ใน DFA จะได้ว่า ต้องมี state ที่ถูกใช้งานอย่างน้อยสองครั้ง

ตัวอย่างนี้เป็นขั้นตอนหนึ่งในการให้เหตุผลของบทพิสูจน์ที่แสดงคุณสมบัติของ regular languages ที่มีชื่อว่า \emph{pumping lemma}
\end{example}

\begin{example}
หากเราเลือกจุดมา 5 จุดภายในสี่เหลี่ยมจัตุรัสขนาด $1\times 1$ ตารางหน่วย จะต้องมีอย่างน้อย 2 จุดที่มีระยะห่างกันไม่เกิน $1/\sqrt{2}$ หน่วย

ในการวิเคราะห์ตัวอย่างนี้ เราจะทำการแบ่งสี่เหลี่ยมออกเป็น 4 ส่วนย่อย โดยมีพื้นที่ส่วนละ $\frac{1}{2}\times\frac{1}{2}$ ตารางหน่วย ดังรูป
\begin{center}
\begin{tikzpicture}
\draw (-2,-2) rectangle (2,2);
\draw[dotted] (-2,0) -- (2,0);
\draw[dotted] (0,-2) -- (0,2);
\node at (-2.5,-0) {$1$};
\node at (0,-2.5) {$1$};
\node at (-1,2.5) {$1/2$};
\node at (1,2.5) {$1/2$};
\node at (2.5,-1) {$1/2$};
\node at (2.5,1) {$1/2$};

\draw[fill,red] (-1,1) circle (0.05);
\draw[fill,red] (0.1,0.1) circle (0.05);
\draw[fill,red] (2,2) circle (0.05);
\draw[fill,red] (-1.8,-1.8) circle (0.05);
\draw[fill,red] (1.8,-0.8) circle (0.05);
\end{tikzpicture}
\end{center}
หากเราเลือกจุดมา 5 จุดในสี่เหลี่ยมใหญ่นี้ จาก Pigeonhole Principle จะได้ว่า ต้องมีสี่เหลี่ยมย่อยที่มีจุดอย่างน้อยสองจุด เนื่องจากแต่ละจุดจำเป็นต้องอยู่ในสี่เหลี่ยมย่อยอย่างน้อยหนึ่งรูป (อาจจะมากกว่าหนึ่งรูปได้หากจุดนั้นๆ อยู่บนขอบเขตระหว่างสี่เหลี่ยมย่อยพอดี) \enskip จากรูปข้างต้น สี่เหลี่ยมย่อยบนขวา มีจุดอยู่สองจุด

พิจารณาสี่เหลี่ยมย่อยที่มีจุดอย่างน้อยสองจุดดังกล่าว \enskip เนื่องจากสี่เหลี่ยมย่อยนี้มีขนาด $\frac{1}{2}\times\frac{1}{2}$ จะได้ว่า ระยะทางที่ไกลที่สุดระหว่างสองจุดใดๆ ในสี่เหลี่ยมย่อยนี้ ต้องไม่เกินความยาวของเส้นทแยงมุม ซึ่งมีความยาว
\[\sqrt{\left(\frac{1}{2}\right)^2+\left(\frac{1}{2}\right)^2}=\sqrt{\frac{1}{4}+\frac{1}{4}}=\frac{1}{\sqrt{2}}\]
ดังนั้น สองจุดที่อยู่ในสี่เหลี่ยมย่อยนี้ จึงมีระยะห่างกันไม่เกิน $1/\sqrt{2}$
\end{example}

\begin{example}
ในห้องมืดมีถุงเท้าสีแดง, เขียว, และฟ้า จำนวนมาก \enskip เราจะต้องหยิบถุงเท้าอย่างน้อยกี่ข้างก่อนออกมาจากห้องมืด จึงจะมั่นใจได้ว่ามี 4 ข้างที่มีสีเดียวกันทั้งหมด

ในกรณีที่เลวร้ายที่สุด แต่ละข้างที่เราหยิบได้นั้นจะผลัดสีกันไปเรื่อยๆ กล่าวคือ เมื่อหยิบไป 3 ข้างแรก จะมีถุงเท้าครบทุกสามสี สีละหนึ่งข้าง \enskip เมื่อหยิบไป 6 ข้างแล้ว จะมีถุงเท้าสีละสองข้าง \enskip และเมื่อหยิบไป 9 ข้างแล้ว จะมีถุงเท้าสีละสามข้างพอดี \enskip อย่างไรก็ได้ เมื่อหยิบข้างที่ 10 แล้ว สีของข้างสุดท้ายนี้จะต้องซ้ำกับสีใดสีหนึ่งซึ่งมีอยู่แล้วสามข้าง จึงทำให้มีถุงเท้าสี่ข้างที่มีสีเดียวกัน \enskip ดังนั้น เราจำเป็นต้องหยิบถุงเท้าอย่างน้อย 10 ข้าง
\end{example}

ตัวอย่างข้างต้นนี้เป็นการประยุกต์ใช้งานของ \emph{generalized Pigeonhole Principle}
\begin{axiom}[generalized Pigeonhole Principle]
ถ้ามีนกพิราบ $a$ ตัว และมีช่องให้อยู่ $b$ ช่อง โดยที่ $a>kb$ และ $k\in\mathbb{Z}^+$ แล้วจะต้องมีช่องใดช่องหนึ่งที่มีนกพิราบอย่างน้อย $k+1$ ตัว
\end{axiom}

จากตัวอย่างข้างต้น นกพิราบคือถุงเท้าแต่ละข้าง ส่วนช่องนกพิราบคือสีของถุงเท้า โดยที่เราต้องการให้มีสีที่มีถุงเท้าอย่างน้อย $4=3+1$ ข้าง กล่าวคือ $k=3$ \enskip ดังนั้น หากจำนวนข้างของถุงเท้ามากกว่า $3\cdot 3=9$ ก็จะได้ผลลัพธ์ที่ต้องการ นั่นคือ ต้องหยิบถุงเท้าออกมา 10 ข้าง ก็สามารถรับประกันได้ว่ามีสีใดสีหนึ่งที่มี 4 ข้าง

หากจะเขียน axiom ข้างต้นเป็นเชิงรูปนัย จะได้ดังนี้
\begin{axiom}
ให้ $f:A\to B$ เป็น total function กล่าวคือ $\forall x\in A\exists y\in B: f(x)=y$ \enskip ถ้า $|A|>k|B|$ โดยที่ $k\in\mathbb{Z}^+$ แล้ว
\[\exists x_1,\ldots,x_{k+1}\in A: \left[\forall i,j: 1\leq i,j\leq k+1\wedge i\neq j\implies x_i\neq x_j\right] \wedge f(x_1)=\cdots=f(x_{k+1})\]
\end{axiom}
