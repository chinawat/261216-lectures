\section{Binomial theorem}
\begin{theorem}[\emph{ทฤษฎีบททวินาม} (binomial theorem)]
\[(a+b)^n=\sum_{k=0}^{n}{\binom{n}{k}a^{n-k}b^k}\]
\begin{pf}
ก่อนอื่น $(a+b)^n$ สามารถเขียนกระจายได้เป็น \[\underbrace{(a+b)(a+b)\cdots(a+b)}_{\text{$n$ ตัว}}\]
ตัวอย่างเช่น 
\begin{align*}
(a+b)^3
&=(a+b)(a+b)(a+b) \\
&=aaa+aab+aba+abb+baa+bab+bba+bbb \\
&=a^3+3a^2b+3ab^2+b^3
\end{align*}
จะเห็นว่า สัมประสิทธิ์ของ $a^2b$ นั้นเป็น 3 เนื่องจากมีพจน์ $aab$, $aba$, และ $baa$ ที่คูณกันแล้วได้ $a^2b$ \enskip ดังนั้น จำนวนพจน์ที่คูณกันแล้วได้ $a^2b$ ก็คือจำนวนลำดับซึ่งแต่ละลำดับมีความยาว 3 โดยมี $a$ สองตัวและ $b$ หนึ่งตัว

ในกรณีทั่วไป สัมประสิทธิ์ของพจน์ $a^{n-k}b^k$ คือจำนวนลำดับซึ่งแต่ละลำดับมีความยาว $n$ โดยมี $a$ อยู่ $n-k$ ตัว และ $b$ อยู่ $k$ ตัว \enskip ลำดับต่างๆ เหล่านี้สามารถมองได้เป็น strings ที่มีความยาว $n$ โดยที่แต่ละตัวเป็น $a$ หรือ $b$

จำนวน strings ดังกล่าวที่มี $a$ อยู่ $n-k$ ตัว และ $b$ อยู่ $k$ ตัว สามารถคำนวณได้ดังนี้ \enskip ใน string แต่ละตัวที่มีคุณสมบัติดังกล่าว มีทั้งหมด $n$ ตำแหน่ง ที่เราต้องเลือกว่าในท้ายที่สุดจะเป็น $a$ หรือ $b$ \enskip เนื่องจากเราต้องการให้มี $b$ จำนวน $k$ ตัว จำนวน strings ทั้งหมดก็คือจำนวนวิธีการเลือก $k$ ตำแหน่งมาจาก $n$ ตำแหน่งให้เป็น $b$ (ส่วนตำแหน่งที่เหลือให้เป็น $a$) ซึ่งสามารถทำได้ $\binom{n}{k}$ วิธี

ดังนั้น สัมประสิทธิ์ของ $a^{n-k}b^k$ จึงเป็น $\binom{n}{k}$ แต่เนื่องจาก $k$ นั้นเป็นเท่าใดก็ได้ในช่วง 0--$n$ กล่าวคือ $0\leq k\leq n$ ผลลัพธ์จากการกระจายจึงเป็นผลรวมของทุกพจน์ กล่าวคือ
\[(a+b)^n=\sum_{k=0}^{n}{\binom{n}{k}a^{n-k}b^k}\]
\end{pf}
\end{theorem}

\begin{corollary}
\[2^n=\sum_{k=0}^{n}{\binom{n}{k}}\]
\begin{pf}
ในที่นี้ จะนำเสนอวิธีพิสูจน์สองวิธี

ในวิธีแรก เราสามารถแทนค่า $a=b=1$ ไปใน binomial theorem จะได้ $2^n=\sum_{k=0}^{n}{\binom{n}{k}}$

ในวิธีที่สอง จะใช้ combinatorial argument โดยจะนับจำนวน subsets ของเซตที่มีสมาชิก $n$ ตัว ซึ่งสามารถทำได้สองวิธีการ ดังนี้
\begin{enumerate}[]
\item เมื่อพิจารณาสมาชิกแต่ละตัวในเซต จะเห็นว่า เราสามารถเลือกได้ว่าสมาชิกตัวนั้นๆ จะอยู่ใน subset ที่เรากำลังสร้างขึ้นหรือไม่ \enskip กล่าวอีกนัยหนึ่ง เรามีสองทางเลือกสำหรับสมาชิกแต่ละตัวในเซต \enskip เนื่องจากเซตนี้มีสมาชิก $n$ ตัว แต่ละตัวมี 2 ทางเลือก จะได้ว่า จำนวนวิธีสร้าง subsets ทั้งหมดที่เป็นไปได้คือ $\underbrace{2\cdot 2\cdot\cdots\cdot 2}_{\text{$n$ ตัว}}=2^n$
\item วิธีการสร้าง subset อีกแบบหนึ่ง คือเลือกสมาชิกที่ต้องการไว้ใน subset จากสมาชิกทั้งหมดที่เป็นไปได้ของเซต \enskip ให้ $k$ เป็นจำนวนสมาชิกที่เราต้องการใน subset ที่เราสร้างขึ้น \enskip จะเห็นว่า $k$ นั้นเป็นได้ตั้งแต่ 0 ถึง $n$ กล่าวคือ $0\leq k\leq n$ (ในกรณีที่ $k=0$ แสดงว่าเราไม่เลือกสมาชิกใดๆ ในเซตมาเลย ทำให้ subset ที่ได้เป็นเซตว่าง) \enskip การเลือกสมาชิก $k$ ตัว จากทั้งหมด $n$ ตัวมาไว้ใน subset สามารถทำได้ $\binom{n}{k}$ วิธี \enskip แต่เนื่องจาก $0\leq k\leq n$ จะได้ว่า จำนวนวิธีการเลือก subsets ทั้งหมดที่เป็นไปได้ ก็คือผลรวมของจำนวน subsets ที่เป็นไปได้สำหรับแต่ละค่าของ $k$ กล่าวคือ จำนวน subsets ทั้งหมดที่เป็นไปได้เท่ากับ $\sum_{k=0}^{n}{\binom{n}{k}}$
\end{enumerate}
เนื่องจากในสองวิธีข้างต้นนั้นเป็นการนับสิ่งเดียวกัน จึงสรุปได้ว่า $2^n=\sum_{k=0}^{n}{\binom{n}{k}}$
\end{pf}
\end{corollary}
