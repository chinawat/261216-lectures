\chapter{Graph theory}

เราอาจจะเคยได้ยินคำว่ากราฟมาก่อน จากบทเรียนวิชาคณิตศาสตร์ที่ผ่านๆ มา ซึ่งความหมายของกราฟในบทเรียนดังกล่าวนั้น คือการวาดแผนภาพความสัมพันธ์ของฟังก์ชันในระนาบสองมิติ \enskip แต่ในวิชาวิยุตคณิต กราฟจะเป็นโครงสร้างเชิงคณิตศาสตร์แบบหนึ่งที่สามารถใช้จำลองปัญหาต่างๆ ได้หลากหลาย และสามารถนำไปเขียนเป็นโครงสร้างข้อมูล (data structure) ที่ใช้ในโปรแกรมคอมพิวเตอร์เพื่อแก้ปัญหาได้เช่นกัน

ก่อนที่จะเข้าสู่เนื้อหาหลักของกราฟ จะขอยกตัวอย่างปัญหาที่สามารถแก้ไขได้ด้วยกราฟ \enskip สมมุติว่ามีเมืองอยู่เมืองหนึ่ง ซึ่งมีแม่น้ำสองสายไหลมาบรรจบกันที่ใจกลางเมือง ดังรูปที่~\ref{fig:island-map} \enskip เมืองนี้มีเกาะอยู่หนึ่งเกาะกลางเมือง ทำให้มีแผ่นดินทั้งสิ้น 4 ส่วน ดังหมายเลขที่ระบุในรูป \enskip เนื่องจากสภาพภูมิประเทศของเมืองนี้แตกต่างจากเมืองทั่วๆ ไป ประชาชนของเมืองนี้จึงต้องการผลักดันให้เมืองนี้เป็นแหล่งท่องเที่ยว ทำให้มีการสร้างสะพานเชื่อมแผ่นดินแต่ละส่วนเพื่ออำนวยความสะดวกให้่แก่นักท่องเที่ยว \enskip เมื่อการก่อสร้างเสร็จสิ้นแล้ว มีสะพานทั้งสิ้น 5 สะพาน เชื่อมส่วนต่างๆ ของเมืองเช้าด้วยกัน ดังรูป
%
\begin{figure}
    \centering
    \begin{tikzpicture}[scale=0.6]
        \pgfdeclarelayer{bg}
        \pgfsetlayers{background,bg,main}
        
        \draw[very thick] (0,0) rectangle (15,9);
        
        \scoped[on background layer] {
          \draw[very thick,fill=blue!20]
            plot[smooth] coordinates {(12.5,9) (10,7.2) (5,5.8) (2.5,4.8) (0,4.4)} --
            plot[smooth] coordinates {(0,3) (2,3.1) (5,2.9) (8,2.7) (10,2.3) (11,2) (12,1.4) (13.8,0)} --
            (15,0) --
            plot[smooth] coordinates {(15,1.5) (14,2) (13,3) (12,3.3) (10,3.5) (9.1,4) (8.8,5) (11,6) (13,6.5) (15,7.8)} --
            (15,9) --
            cycle;
        
          \draw[very thick,fill=white] plot[smooth cycle] coordinates {(4.3,3.9) (5,3.5) (6.5,3.5) (7.8,3.9) (7.3,4.7) (5,4.5)};
        }
        
        \draw[ultra thick,name path=b11] plot[smooth] coordinates {(5.4,6.2) (5.7,5.3) (5.8,4.3)};
        \draw[ultra thick,name path=b12] plot[smooth] coordinates {(5.8,6.3) (6.1,5.2) (6.2,4.4)};
        
        \draw[ultra thick,name path=b21] plot[smooth] coordinates {(6,3.8) (6.1,3) (5.9,2.3)};
        \draw[ultra thick,name path=b22] plot[smooth] coordinates {(6.4,3.9) (6.5,3) (6.3,2.2)};
        
        \draw[ultra thick,name path=b31] plot[smooth] coordinates {(7.2,4.3) (8.6,4.7) (9.4,4.8)};
        \draw[ultra thick,name path=b32] plot[smooth] coordinates {(7.3,4) (8.9,4.4) (9.5,4.5)};
        
        \draw[ultra thick,name path=b41] plot[smooth] coordinates {(10.3,1.7) (11.1,2.8) (12,3.7)};
        \draw[ultra thick,name path=b42] plot[smooth] coordinates {(10.7,1.6) (11.5,2.6) (12.5,3.6)};
        
        \draw[ultra thick,name path=b51] plot[smooth] coordinates {(10,7.6) (10.7,6.2) (11.3,5.5)};
        \draw[ultra thick,name path=b52] plot[smooth] coordinates {(10.4,7.8) (11.1,6.4) (11.8,5.7)};
        
        \tikzfillbetween[of=b11 and b12,on layer=bg] {gray!30};
        \tikzfillbetween[of=b21 and b22,on layer=bg] {gray!30};
        \tikzfillbetween[of=b31 and b32,on layer=bg] {gray!30};
        \tikzfillbetween[of=b41 and b42,on layer=bg] {gray!30};
        \tikzfillbetween[of=b51 and b52,on layer=bg] {gray!30};
        
        \node[circle,draw,thick,inner sep=1pt] at (2,7) {1};
        \node[circle,draw,thick,inner sep=1pt] at (5.25,4) {2};
        \node[circle,draw,thick,inner sep=1pt] at (2,2) {3};
        \node[circle,draw,thick,inner sep=1pt] at (13,4.75) {4};
    \end{tikzpicture}
    \caption{แผนที่เมืองซึ่งประกอบไปด้วยแผ่นดิน 4 ส่วน และสะพาน 5 สะพาน}
    \label{fig:island-map}
\end{figure}

นักท่องเที่ยวที่มาเยี่ยมชมเมืองแห่งนี้สามารถมาถึงได้โดยรถยนต์ส่วนตัว ซึ่งสามารถจอดได้ที่แผ่นดินหมายเลข 1 ด้านเหนือของเมืองเท่านั้น \enskip หลังจากลงรถแล้ว นักท่องเที่ยวต้องการแวะไปยังแผ่นดินทุกส่วนของเมืองโดยเดินข้ามสะพานต่างๆ พร้อมกับเก็บภาพบรรยากาศไปด้วย \enskip อย่างไรก็ดี หากจะใช้สะพานไม่ครบ ก็จะทำให้ภาพบรรยากาศนั้นไม่ครบถ้วน และหากจะใช้สะพานเดิมซ้ำ ก็จะได้ภาพบรรยากาศเดิมๆ เป็นการเสียเวลาโดยใช่เหตุ \enskip นักท่องเที่ยวจึงสงสัยว่า มีวิธีการเดินชมเมืองโดยเริ่มจากที่จอดรถ เดินข้ามสะพานให้ครบ แต่ละสะพานใช้เพียงครั้งเดียว และจบสิ้นการเดินที่ที่จอดรถหรือไม่ \enskip หากมี ควรเดินอย่างไร \enskip หากไม่มี จะอธิบายเหตุผลว่าไม่มีวิธีดังกล่าวอย่างไร

ปัญหาการใช้สะพานครั้งเดียวไม่ขาดไม่เกินข้างต้นนั้น ดัดแปลงมาจากปัญหาเก่าแก่ที่ชื่อว่า \emph{Seven Bridges of Königsberg} ซึ่งเป็นสะพานที่อยู่ในเมือง Königsberg, Prussia (ปัจจุบันเป็นเมือง Kaliningrad ประเทศรัสเซีย) \enskip ในปัญหาดังกล่าว ได้ระบุเพิ่มเติมว่า กติกาในการเดินชมเมืองนั้น หากจะเยี่ยมชมส่วนของแผ่นดินที่แตกต่างจากส่วนปัจจุบัน จำเป็นจะต้องใช้สะพานเท่านั้น (ไม่สามารถว่ายน้ำ กระโดดร่ม หรือโดยวิธีการอื่นๆ ได้) และในการใช้สะพานแต่ละครั้ง จะต้องข้ามจากฝั่งหนึ่งไปยังอีกฝั่งหนึ่งเท่านั้น ไม่สามารถเดินครึ่งๆ กลาง แล้วกลับมาฝั่งเดิมได้

แม้ว่าแผนที่เมืองข้างต้นจะช่วยให้เราเห็นภาพรวมของเมืองมากขึ้น แต่ในการแก้ปัญหานี้นั้น เราไม่จำเป็นจะต้องใช้ทุกรายละเอียดของแผนที่ เนื่องจากเราไม่สนใจว่าแต่ละสะพานมีความยาวเท่าใด หรือควรใช้เวลาในแผ่นดินแต่ละส่วนมากน้อยเพียงใด \enskip ส่วนที่จำเป็นต่อการแก้ปัญหามีเพียงข้อมูลที่ว่า มีแผ่นดินทั้งสิ้น 4 ส่วน และสะพานแต่ละสะพานนั้นเชื่อมแผ่นดินสองส่วนให้ไปหามากันได้ \enskip ดังนั้น เราจึงสามารถตัดรายละเอียดที่ไม่จำเป็นต่อการแก้ปัญหาออกไป โดยการแปลงแผนที่ข้างต้นเป็นแผนภาพได้ดังรูปที่~\ref{fig:map-graph}
%
\begin{figure}
\centering
\begin{tikzpicture}
    \node[circle,draw] (1) at (0,0) {1};
    \node[circle,draw] (2) [below=of 1] {2};
    \node[circle,draw] (3) [below=of 2] {3};
    \node[circle,draw] (4) [right=of 2] {4};

    \draw[thick] (1) -- (2) -- (3) -- (4) -- (1);
    \draw[thick] (2) -- (4);
\end{tikzpicture}
\caption{แผนภาพแสดงข้อมูลของเมืองเท่าที่จำเป็นต่อการแก้ปัญหา}
\label{fig:map-graph}
\end{figure}
%
โดยแผ่นดินแต่ละส่วนจะกลายเป็นจุดแต่ละจุดในแผนภาพ และสะพานแต่ละสะพานจะกลายเป็นเส้นที่เชื่อมสองจุดเข้าด้วยกัน \enskip เราสามารถแปลงปัญหาตั้งต้นที่ต้องการใช้สะพานเพียงครั้งเดียว เป็นปัญหาเกี่ยวกับแผนภาพใหม่นี้ ที่ต้องการใช้เส้นแต่ละเส้นหนึ่งครั้งพอดี โดยเริ่มต้นที่จุดที่ 1 และสิ้นสุดที่จุดที่ 1 เช่นกัน

การตัดรายละเอียดที่ไม่จำเป็นออกข้างต้นนั้น เป็นส่วนหนึ่งของกระบวนการคิด\emph{เชิงนามธรรม} (abstraction) ที่จะช่วยให้เราใส่ใจกับส่วนที่สำคัญของปัญหาได้อย่างเต็มที่ และเป็นพื้นฐานในการแก้ปัญหาที่ซับซ้อนขึ้น \enskip ในปัญหานี้ เราได้ทำการแปลงแผนที่เมืองซึ่งมีรายละเอียดปลีกย่อยที่ไม่จำเป็น ให้กลายเป็นแผนภาพ ซึ่งในที่นี้ก็คือกราฟนั่นเอง

เมื่อพิจารณาอย่างถี่ถ้วน จะเห็นว่า ไม่มีวิธีการเดินโดยเริ่มต้นและสิ้นสุดที่จุดที่ 1 โดยใช้แต่ละเส้นเพียงครั้งเดียว ไม่ขาดไม่เกิน \enskip สำหรับเหตุผลในการอธิบายว่าไม่มีวิธีที่เป็นไปได้นั้น จะได้กล่าวถึงในภายหลัง

\section{Basics of graph theory}

ในส่วนนี้ จะกล่าวถึงนิยามต่างๆ ที่เป็นพื้นฐานของทฤษฎีกราฟ

\begin{definition}
    \emph{กราฟ} (graph) คือคู่อันดับ $G=(V,E)$ โดยที่ $V$ เป็นเซตของ\emph{จุดยอด} (vertices\footnote{หากเป็นเอกพจน์ จะเขียนว่า vertex}) และ $E$ เป็นเซตของ\emph{เส้นเชื่อม} (edges) โดยแต่ละเส้นนั้นเป็น subset ของ $V$ ที่มีสมาชิก 2 ตัว
\end{definition}
ทั้งนี้ อาจจะเรียก vertices ว่า nodes ก็ได้

\begin{example}
กราฟในรูป~\ref{fig:map-graph} ประกอบไปด้วย vertices และ edges ดังนี้
\begin{align*}
    V &= \set{1,2,3,4} \\
    E &= \set{\set{1,2}, \set{1,4}, \set{2,3}, \set{2,4}, \set{3,4}}
\end{align*}
\end{example}
%
สังเกตว่า เส้นในกราฟนั้นสามารถเขียนได้เป็น subsets ของ $V$ ซึ่งแต่ละ subset นั้นมีสมาชิก 2 ตัว เช่น เส้นที่เชื่อมจุด 1 และ 2 จะเป็น subset $\set{1,2}\subseteq V$ \enskip เหตุผลที่เราเขียนแต่ละเส้นเป็นเซตนั้น เนื่องจากลำดับของจุดไม่สำคัญ กล่าวคือ หากจะเขียนแทนเส้นด้วยเซต $\set{2,1}$ ก็ยังคงหมายถึงเส้นที่เชื่อมจุด 1 และ 2 เช่นเดิม

\begin{definition}
ให้ $u,v\in V$ \enskip เรียก $u$ ว่า\emph{ประชิด} (adjacent) กับ $v$ [denoted $u\sim v$] หาก $\set{u,v}\in E$
\end{definition}
%
\begin{example}
จากกราฟในรูปที่~\ref{fig:map-graph} จะได้ว่า $4\sim 2$ เนื่องจาก $\set{4,2}\in E$ แต่ $1\not\sim 3$
\end{example}

\begin{definition}
ให้ $e=\set{u,v}$ เป็น edge ในกราฟ \enskip เรียก $e$ ว่า \emph{incident with} $u$ (และ $v$)
\end{definition}
นั่นคือ แต่ละเส้นในกราฟจะ incident with จุดปลายแต่ละข้างของเส้นนั้นๆ

\begin{example}
จากกราฟในรูปที่~\ref{fig:map-graph} จะได้ว่าเส้น $\set{1,4}$ is incident with vertex 1.
\end{example}

\begin{definition}
ให้ $v\in V$ \enskip \emph{ดีกรี} (degree) ของ $v$ [denoted $\deg(v)$] คือจำนวนเส้นที่ incident with $v$
\end{definition}
นั่นคือ degree ของแต่ละจุด คือจำนวน edges ที่ต่ออยู่กับจุดนั้นๆ

\begin{example}
จากกราฟในรูปที่~\ref{fig:map-graph} จะได้ว่า $\deg(1)=2$ และ $\deg(2)=3$
\end{example}

จากนิยามต่างๆ ที่กล่าวมาข้างต้น เราสามารถพิสูจน์คุณสมบัติแรกของกราฟได้ดังนี้
\begin{lemma}
ให้ $G=(V,E)$ \enskip $\sum_{v\in V}{\deg(v)}=2|E|$
\begin{pf}
ในที่นี้ จะใช้ combinatorial proof โดยกำหนด $S\triangleq\set{(u,e)\mid\text{$e$ is incident with $u$}}$ เป็นเซตที่เราจะนับจำนวนสมาชิก \enskip เซตนี้เป็นเซตของคู่อันดับ ที่ตัวหน้าของคู่อันดับเป็น vertex ในกราฟ และตัวหลังของคู่อันดับเป็นเส้นที่ต่ออยู่กับ vertex นั้นๆ

ถัดไป จะทำการคำนวณ $|S|$ โดยสองวิธี
\begin{enumerate}
\item จะเห็นว่า แต่ละจุด $v$ ในกราฟ จะก่อให้เกิดคู่อันดับใน $S$ ทั้งสิ้น $\deg(v)$ คู่อันดับ โดยแต่ละคู่อันดับจะเป็นแต่ละเส้นที่ incident with $v$ \enskip กล่าวอีกนัยหนึ่ง $\deg(v)$ คือจำนวนสมาชิกของ $S$ โดยที่ตัวหน้าของคู่อันดับเป็น $v$ \enskip เมื่อรวม degrees ของทุกๆ จุดในกราฟ จะได้จำนวนสมาชิกทั้งหมดของ $S$ กล่าวคือ $|S|=\sum_{v\in V}{\deg(v)}$
\item จะเห็นว่า แต่ละเส้น $e=\set{u,v}$ ในกราฟ จะก่อให้เกิดคู่อันดับใน $S$ ทั้งสิ้น 2 คู่อันดับ ได้แก่ $(u,e)$ และ $(v,e)$ \enskip ดังนั้น จำนวนสมาชิกทั้งหมดของ $S$ ย่อมเป็นสองเท่าของจำนวนเส้นในกราฟ กล่าวคือ $|S|=2|E|$
\end{enumerate}
เนื่องจากเราสามารถนับจำนวนสมาชิกของ $S$ ได้สองวิธีที่แตกต่างกันข้างต้น จึงสรุปได้ว่า จำนวนสมาชิกที่นับได้ในแต่ละวิธีนั้นย่อมเท่ากัน นั่นคือ $\sum_{v\in V}{\deg(v)}=2|E|$
\end{pf}
\end{lemma}
%
\begin{example}
จากกราฟในรูปที่~\ref{fig:map-graph} จะได้ว่า เซต $S$ ใน lemma ข้างต้นเป็นเซต
\begin{align*}
\{
& (1,\set{1,2}),(2,\set{1,2}), (1,\set{1,4}),(4,\set{1,4}), \\
& (2,\set{2,3}),(3,\set{2,3}), (2,\set{2,4}),(4,\set{2,4}), \\
& (3,\set{3,4}),(4,\set{3,4})
\}
\end{align*}
\end{example}

\begin{definition}
ให้ $G=(V,E)$ \enskip \emph{แนวเดิน} (walk) คือลำดับของ vertices โดยที่แต่ละ vertex ในลำดับ ต้อง adjacent กับ vertex ตัวถัดไปในลำดับ \enskip ความยาวของ walk คือจำนวนครั้งที่มีการเปลี่ยนผ่านจุดในลำดับ ซึ่งจะน้อยกว่าความยาวของลำดับนั้นๆ อยู่ 1
\end{definition}
นั่นคือ walk เป็นการเดินไปมาในกราฟ โดยเริ่มจากจุดใดจุดหนึ่ง และเปลี่ยนจุดไปเรื่อยๆ โดยเดินตามเส้นที่ต่ออยู่กับจุดปัจจุบัน

\begin{example}
จากกราฟในรูปที่~\ref{fig:map-graph} ลำดับของ vertices ต่อไปนี้เป็น walks:
\begin{itemize}
\item $1,2,4,3$ เป็น walk ที่มีความยาว 3
\item $1,4,2,3,4$ เป็น walk ที่มีความยาว 4
\item $1,2,1,2,3,4,2,1,4$ เป็น walk ที่มีความยาว 8
\item $3$ เป็น walk ที่มีความยาว 0
\end{itemize}
\end{example}

\begin{definition}
\emph{วิถี} (path) คือ walk ที่ไม่มี vertex ซ้ำในลำดับ
\end{definition}
%
\begin{example}
จากกราฟในรูปที่~\ref{fig:map-graph} ลำดับของ vertices ต่อไปนี้เป็น paths:
\begin{itemize}
\item $1,2,4,3$ เป็น path ที่มีความยาว 3
\item $3$ เป็น path ที่มีความยาว 0
\end{itemize}
\end{example}

\begin{exercise}
กราฟในรูปที่~\ref{fig:map-graph} มี walks จากจุด 1 ไปยังจุด 3 ทั้งสิ้นเป็นจำนวนเท่าใด
\end{exercise}

\begin{exercise}
กราฟในรูปที่~\ref{fig:map-graph} มี paths จากจุด 1 ไปยังจุด 3 ทั้งสิ้นเป็นจำนวนเท่าใด
\end{exercise}

\begin{definition}
ให้ $u,v\in V$ \enskip เรียก $u$ ว่า\emph{เชื่อมโยงกับ} (connected to) $v$ หากมี path จาก $u$ ไป $v$
\end{definition}
%
\begin{example}
จากกราฟในรูปที่~\ref{fig:map-graph}, vertex 1 is connected to vertex 3, and vertex 4 is connected to itself.
\end{example}

\begin{definition}
เรียกกราฟว่า\emph{เชื่อมต่อ} (connected) หาก $\forall u,v\in V \exists\,\text{path จาก $u$ ไป $v$}$
\end{definition}
%
\begin{example}
กราฟในรูปที่~\ref{fig:map-graph} เป็น connected graph
\end{example}
%
\begin{example}
กราฟในรูปที่~\ref{fig:disconnected-graph} เป็น disconnected graph เนื่องจากไม่มี path จากจุด 2 ไปยังจุด 5
\begin{figure}
\centering
\begin{tikzpicture}
\node[circle,draw] (1) at (0,0) {1};
\node[circle,draw] (2) at (1,1) {2};
\node[circle,draw] (3) at (2,0) {3};
\node[circle,draw] (4) at (3,1) {4};
\node[circle,draw] (5) at (4,0) {5};
\node[circle,draw] (6) at (5,1) {6};
\node[circle,draw] (7) at (6,0) {7};
\draw (1) -- (2) -- (4) -- (3) -- (1);
\draw (5) -- (6);
\end{tikzpicture}
\caption{A disconnected graph}
\label{fig:disconnected-graph}
\end{figure}
\end{example}

\subsection{Common graphs}

กราฟที่พบเห็นบ่อย มีดังนี้

\begin{definition}
ให้ $n$ เป็นจำนวนเต็มบวก \enskip \emph{กราฟแบบบริบูรณ์} (complete graph) $K_n$ คือกราฟที่มี $n$ vertices และมี edge เชื่อมทุกคู่ของ vertices ที่เป็นไปได้
\end{definition}
%
\begin{example}
กราฟต่อไปนี้เป็น $K_5$ ซึ่งมีทั้งสิ้น 5 vertices และ 10 edges
\begin{center}
\begin{tikzpicture}
\foreach \i in {0,...,4} {
    \node[circle,draw] (\i) at (90-72*\i:1) {};
    \foreach \j in {0,...,\i} {
        \ifnum\i=\j\relax\else
        \draw (\i) -- (\j);
        \fi
    }
}
\end{tikzpicture}
\end{center}
\end{example}
เนื่องจาก $K_n$ มี $n$ vertices และมีเส้นเชื่อมทุกคู่จุดที่เป็นไปได้ จะได้ว่า จำนวน edges ใน $K_n$ มีค่าเป็น $\binom{n}{2}=\frac{n(n-1)}{2}$

\begin{definition}
\emph{กราฟว่าง} (empty graph) คือกราฟที่ไม่มี edge เลย
\end{definition}
%
\begin{example}
กราฟต่อไปนี้เป็น empty graph ที่มี 5 vertices
\begin{center}
\begin{tikzpicture}
\foreach \i in {0,...,4} {
    \node[circle,draw] (\i) at (90-72*\i:1) {};
}
\end{tikzpicture}
\end{center}
\end{example}

\begin{definition}
ให้ $n$ เป็นจำนวนเต็มบวก \enskip \emph{กราฟเส้น} (line graph) $L_n$ คือกราฟที่มี $n$ vertices และมี $n-1$ edges ต่อกันเป็นเส้นตรง
\end{definition}
%
\begin{example}
สองกราฟต่อไปนี้เป็น $L_5$ \enskip จะเห็นว่า แม้ว่าทั้งสองกราฟนี้จะวาดไม่เหมือนกัน แต่เราสามารถเขียนแจกแจงกราฟในรูปของคู่อันดับ $G=(V,E)$ ได้เหมือนกัน ดังนั้น สองกราฟนี้จึงเป็นกราฟเดียวกัน
\begin{center}
\begin{multicols}{2}
\begin{tikzpicture}
\foreach \i[count=\j from -1] in {0,...,4} {
    \node[circle,draw] (\i) at (\i,0) {};
    \ifnum\i=0\relax\else
        \draw (\j) -- (\i);
    \fi
}
\end{tikzpicture}

\begin{tikzpicture}
\node[circle,draw] (0) at (0,0) {};
\node[circle,draw] (1) at (0,-1) {};
\node[circle,draw] (2) at (1,-1) {};
\node[circle,draw] (3) at (1,0) {};
\node[circle,draw] (4) at (2,0) {};
\foreach \i[count=\j from 0] in {1,...,4} {
    \ifnum\i=0\relax\else
        \draw (\j) -- (\i);
    \fi
}
\end{tikzpicture}
\end{multicols}
\end{center}
\end{example}

\begin{definition}
ให้ $n$ เป็นจำนวนเต็มบวกที่มีค่าอย่างน้อย 3 \enskip \emph{กราฟวง} (cycle) $C_n$ คือกราฟที่มี $n$ vertices และมี $n$ edges ต่อกันเป็นวงรอบ
\end{definition}
%
\begin{example}
\label{ex:cycle-graph}
สองกราฟต่อไปนี้เป็น $C_5$ \enskip จะเห็นว่า แม้ว่าทั้งสองกราฟนี้จะวาดไม่เหมือนกัน แต่เราสามารถเขียนแจกแจงกราฟในรูปของคู่อันดับ $G=(V,E)$ ได้เหมือนกัน ดังนั้น สองกราฟนี้จึงเป็นกราฟเดียวกัน
\begin{center}
\begin{multicols}{2}
\begin{tikzpicture}
\foreach \i[count=\j from -1] in {0,...,4} {
    \node[circle,draw] (\i) at (90-72*\i:1) {};
    \ifnum\i=0\relax\else
        \draw (\j) -- (\i);
    \fi
}
\draw (4) -- (0);
\end{tikzpicture}

\begin{tikzpicture}
\foreach \i[count=\j from -1] in {0,...,4} {
    \node[circle,draw] (\i) at (90-144*\i:1) {};
    \ifnum\i=0\relax\else
        \draw (\j) -- (\i);
    \fi
}
\draw (4) -- (0);
\end{tikzpicture}
\end{multicols}
\end{center}
\end{example}

\subsection{Subgraphs}

\begin{definition}
ให้ $G=(V,E)$ เป็นกราฟ \enskip เรียก $G'=(V',E')$ ว่า\emph{กราฟย่อย} (subgraph) ของ $G$ หาก $V'\subseteq V$, $E'\subseteq E$, และ $\forall\set{u,v}\in E': u\in V'\wedge v\in V'$
\end{definition}
%
กล่าวอีกนัยหนึ่ง จากกราฟใดๆ เราสามารถสร้างกราฟย่อยได้ โดยการลบจุดหรือเส้นออก \enskip ทั้งนี้ หากลบจุดใดจุดหนึ่งออกจากกราฟตั้งต้น จำเป็นจะต้องลบเส้นที่ incident with จุดนั้นๆ ด้วย มิฉะนั้นเส้นดังกล่าวจะมีจุดปลายไม่ครบทั้งสองด้าน

\begin{example}
$K_5$ มี $C_5$, $K_4$, และ $L_3$ เป็น subgraphs ดังรูปที่~\ref{fig:subgraphs}

\begin{figure}
\centering
\begin{multicols}{2}
\begin{tikzpicture}
\foreach \i in {0,...,4} {
    \node[circle,draw] (\i) at (90-72*\i:1) {};
    \foreach \j in {0,...,\i} {
        \ifnum\i=\j\relax\else
        \draw (\i) -- (\j);
        \fi
    }
}
\end{tikzpicture}
\\
$K_5$ (กราฟตั้งต้น)

\begin{tikzpicture}
\foreach \i in {0,...,4} {
    \ifnum\i=0
        \node[circle] (\i) at (90-72*\i:1) {};
    \else
        \node[circle,draw] (\i) at (90-72*\i:1) {};
        \foreach \j in {1,...,\i} {
            \ifnum\i=\j\relax\else
            \draw (\i) -- (\j);
            \fi
        }
    \fi
}
\end{tikzpicture}
\\
$K_4$ ได้จากการลบจุดออกจาก $K_5$\\
(ซึ่งทำให้ต้องลบเส้นทิ้งไปด้วย)
\columnbreak

\begin{tikzpicture}
\foreach \i[count=\j from -1] in {0,...,4} {
    \node[circle,draw] (\i) at (90-72*\i:1) {};
    \ifnum\i=0\relax\else
        \draw (\j) -- (\i);
    \fi
}
\draw (4) -- (0);
\end{tikzpicture}
\\
$C_5$ ได้จากการลบเส้นออกจาก $K_5$

\begin{tikzpicture}
\foreach \i in {0,2,3} {
    \node[circle,draw] (\i) at (90-72*\i:1) {};
}
\draw (3) -- (0) -- (2);
\end{tikzpicture}
\\
$L_3$ ได้จากการลบทั้งจุดและเส้นออกจาก $K_5$
\end{multicols}
\caption{Subgraphs}
\label{fig:subgraphs}
\end{figure}
\end{example}

\subsection{Graph isomorphism}

จากตัวอย่างที่ผ่านๆ มา จะเห็นว่า กราฟแต่ละกราฟ อาจจะมีวิธีวาดได้หลายแบบ \enskip ในบางครั้ง อาจจะไม่ชัดเจนว่ากราฟสองกราฟที่เห็นนั้นเป็นกราฟเดียวกันหรือไม่ แต่ในเชิงทฤษฎีกราฟนั้น เราสามารถนิยามว่าเมื่อใดกราฟสองกราฟนั้น แท้จริงแล้วเป็นกราฟเดียวกัน

\begin{definition}
กราฟ $G$ และ $H$ มี\emph{สัณฐานเหมือนกัน} (isomorphic) หากสามารถตั้งชื่อ vertices ในแต่ละกราฟ แล้วทำให้เขียนกราฟทั้งสองในรูป $(V,E)$ ให้ตรงกันได้
\end{definition}
%
\begin{example}
กราฟใน Example~\ref{ex:cycle-graph} นั้น isomorphic กัน เนื่องจากเราสามารถตั้งชื่อแต่ละจุดในแต่ละกราฟดังรูปด้านล่าง แล้วทำให้เขียนแจกแจงกราฟในรูปของคู่อันดับ $G=(V,E)$ ได้เหมือนกัน กล่าวคือ
\begin{align*}
    V &= \set{1,2,3,4,5} \\
    E &= \set{\set{1,2},\set{2,3},\set{3,4},\set{4,5},\set{5,1}}
\end{align*}
\begin{center}
\begin{multicols}{2}
\begin{tikzpicture}
\foreach \i[count=\j from -1] in {0,...,4} {
    \node[circle,draw] (\i) at (90-72*\i:1) {\number\numexpr\i+1\relax};
    \ifnum\i=0\relax\else
        \draw (\j) -- (\i);
    \fi
}
\draw (4) -- (0);
\end{tikzpicture}

\begin{tikzpicture}
\foreach \i[count=\j from -1] in {0,...,4} {
    \node[circle,draw] (\i) at (90-144*\i:1) {\number\numexpr\i+1\relax};
    \ifnum\i=0\relax\else
        \draw (\j) -- (\i);
    \fi
}
\draw (4) -- (0);
\end{tikzpicture}
\end{multicols}
\end{center}
\end{example}

\begin{exercise}
สองกราฟต่อไปนี้ isomorphic กัน \enskip จากกราฟที่กำหนดให้ ให้กำหนดชื่อจุดแต่ละจุดในแต่ละกราฟ ที่ทำให้เขียนแจกแจงกราฟในรูปของคู่อันดับแล้ว ได้ผลลัพธ์ที่ตรงกัน
\begin{center}
\begin{multicols}{2}
\begin{tikzpicture}
\foreach \i[count=\j from -1] in {0,...,4} {
    \node[circle,draw] (v1\i) at (90-72*\i:1) {};
    \node[circle,draw] (v2\i) at (90-72*\i:2) {};
    \draw (v1\i) -- (v2\i);
    \ifnum\i=0\relax\else
        \draw (v2\j) -- (v2\i);
    \fi
}
\draw (v24) -- (v20);
\draw (v10) -- (v12) -- (v14) -- (v11) -- (v13) -- (v10);
\end{tikzpicture}

\begin{tikzpicture}
\node[circle,draw] (v0) at (0,0) {};
\foreach \j in {1,...,3} {
    \node[circle,draw] (v\j) at (150-120*\j:1.1) {};
}
\foreach \i[count=\p from 0] in {1,2} {
    \foreach \j in {0,...,2} {
        \node[circle,draw] (v\i\j) at (90-120*\j:1.1*\i) {};
        \ifnum\i=1
            \draw (v\p) -- (v\i\j);
        \else
            \draw (v\p\j) -- (v\i\j);
        \fi
    }
}
\draw (v20) -- (v1) -- (v21) -- (v2) -- (v22) -- (v3) -- (v20);
\path
    (v1) edge[bend left] (v12)
    (v2) edge[bend left] (v10)
    (v3) edge[bend left] (v11);
\end{tikzpicture}
\end{multicols}
\end{center}
\end{exercise}

\section{Euler tours and trails}

หลังจากที่เราได้นิยามกราฟและคุณสมบัติต่างๆ ของกราฟแล้ว ณ จุดนี้ เราจะสามารถตอบคำถามเกี่ยวกับกราฟในรูปที่~\ref{fig:map-graph} ได้

\begin{definition}
\emph{Euler trail} คือ walk ที่แวะผ่าน (traverses) ทุกเส้นในกราฟหนึ่งครั้งพอดี \enskip \emph{Euler tour} คือ walk ที่ traverses ทุกเส้นในกราฟหนึ่งครั้งพอดี และเริ่มต้นและสิ้นสุดที่ vertex เดียวกัน
\end{definition}
%
จากกราฟในรูปที่~\ref{fig:map-graph} จะได้ว่า ปัญหาดังกล่าวสามารถเขียนใหม่ได้เป็น กราฟนี้มี Euler tour ที่เริ่มต้นและสิ้นสุดที่จุดที่ 1 หรือไม่ \enskip คำตอบของปัญหาดังกล่าว หาได้จากทฤษฎีบทต่อไปนี้

\begin{theorem}
\label{thm:euler-tour-degree}
ให้ $G$ เป็น connected graph
\begin{itemize}
    \item $G$ มี Euler tour ก็ต่อเมื่อ ทุก vertices ใน $G$ มี degree เป็นเลขคู่
    \item $G$ มี Euler trail ก็ต่อเมื่อ มี 2 vertices ใน $G$ (ไม่ขาดไม่เกิน) ที่มี degree เป็นเลขคี่ และ Euler trail ต้องเริ่มต้นและสิ้นสุดที่ 2 vertices นี้
\end{itemize}
\end{theorem}
%
\begin{example}
เนื่องจากจุด 2 ในกราฟในรูปที่~\ref{fig:map-graph} มี degree เป็น 3 จะได้ว่า กราฟดังกล่าวไม่มี Euler tour \enskip อย่างไรก็ดี มีเพียง 2 จุดเท่านั้นในกราฟที่มี degree เป็นเลขคี่ ได้แก่ จุด 2 และ 4 ซึ่งมี degree เป็น 3 ทั้งคู่ \enskip ดังนั้น กราฟดังกล่าวจึงมี Euler trail ที่เริ่มต้นที่จุด 2 และสิ้นสุดที่จุด 4 (หรือกลับกัน) โดยมีลำดับการเดินแบบหนึ่งที่เป็นไปได้ดังนี้
\[2 → 3 → 4 → 2 → 1 → 4\]
\end{example}

ถัดไป จะทำการพิสูจน์ทฤษฎีบทดังกล่าว ซึ่งจะกล่าวถึงวิธีการหา Euler tour หรือ Euler trail ด้วย
%
\begin{pf}[Proof of Theorem~\ref{thm:euler-tour-degree}]
\end{pf}

นอกจาก Euler tours และ Euler trails จะสามารถให้เหตุผลเกี่ยวกับปัญหาการเดินชมเมือง ดังที่ได้กล่าวไปแล้ว ยังสามารถประยุกต์ใช้งานกับปัญหาการวาดภาพโดยไม่ยกดินสอได้ด้วย ดังตัวอย่างต่อไปนี้

\begin{example}
ในแต่ละภาพวาดต่อไปนี้ เราสามารถแปลงแต่ละตำแหน่งที่มีเส้นเชื่อมกัน หรือตำแหน่งที่เส้นมีการหักมุม เป็นแต่ละจุดในกราฟ \enskip ทั้งนี้ สำหรับตำแหน่งที่เกิดจากการตัดกันของเส้นสองเส้น เราอาจจะพิจารณาว่าไม่เป็นจุดในกราฟก็ได้ แต่หากจะนับเป็นจุดในกราฟ สังเกตว่า จุดเหล่านี้จะมี degree เป็น 4 ซึ่งไม่ส่งผลลบต่อความเป็นไปได้ในการหา Euler trail หรือ Euler tour
\begin{tabularx}{\linewidth}{@{}cX@{}}
& กราฟนี้ไม่สามารถวาดโดยไม่ยกดินสอได้ เนื่องจากทุกจุดมี degree เป็น 3 (ยกเว้นจุดตรงกลาง ซึ่งมี degree เป็น 4 แต่เราอาจจะไม่นับตำแหน่งดังกล่าวเป็นจุดในกราฟ) กล่าวคือ กราฟนี้ไม่มีทั้ง Euler trail และ Euler tour \\
& กราฟนี้มี Euler tour เนื่องจากทุกจุดมี degree เป็น 2 โดยสามารถวาดรูปนี้ได้โดยลากเส้นตามเข็มหรือทวนเข็มนาฬิกาจนครบทุกเส้น \\
& กราฟนี้ไม่มี Euler tour แต่มี Euler trail เนื่องจากสองจุดล่างนั้นมี degree เป็น 3 ส่วนจุดอื่นๆ มี degree เป็นเลขคู่ทั้งหมด \enskip ดังนั้น หากจะวาดรูปนี้โดยไม่ยกดินสอ จะต้องเริ่มจากจุดล่างจุดใดจุดหนึ่ง แล้วสิ้นสุดที่จุดล่างอีกจุดหนึ่ง
\end{tabularx}
\end{example}

\section{Graph coloring}

\subsection{Bipartite graphs}

\section{Planar graphs}

