\section{Euler tours and trails}

หลังจากที่เราได้นิยามกราฟและคุณสมบัติต่างๆ ของกราฟแล้ว ณ จุดนี้ เราจะสามารถตอบคำถามเกี่ยวกับกราฟในรูปที่~\ref{fig:map-graph} ได้

\begin{definition}
\emph{Euler trail} คือ walk ที่แวะผ่าน (traverses) ทุกเส้นในกราฟหนึ่งครั้งพอดี \enskip \emph{Euler tour} คือ walk ที่ traverses ทุกเส้นในกราฟหนึ่งครั้งพอดี และเริ่มต้นและสิ้นสุดที่ vertex เดียวกัน
\end{definition}
%
จากกราฟในรูปที่~\ref{fig:map-graph} จะได้ว่า ปัญหาดังกล่าวสามารถเขียนใหม่ได้เป็น กราฟนี้มี Euler tour ที่เริ่มต้นและสิ้นสุดที่จุดที่ 1 หรือไม่ \enskip คำตอบของปัญหาดังกล่าว หาได้จากทฤษฎีบทต่อไปนี้

\begin{theorem}
\label{thm:euler-tour-degree}
ให้ $G$ เป็น connected graph
\begin{itemize}
    \item $G$ มี Euler tour ก็ต่อเมื่อ ทุก vertices ใน $G$ มี degree เป็นเลขคู่
    \item $G$ มี Euler trail ก็ต่อเมื่อ มี 2 vertices ใน $G$ (ไม่ขาดไม่เกิน) ที่มี degree เป็นเลขคี่ และ Euler trail ต้องเริ่มต้นและสิ้นสุดที่ 2 vertices นี้
\end{itemize}
\end{theorem}
%
\begin{example}
เนื่องจากจุด 2 ในกราฟในรูปที่~\ref{fig:map-graph} มี degree เป็น 3 จะได้ว่า กราฟดังกล่าวไม่มี Euler tour \enskip อย่างไรก็ดี มีเพียง 2 จุดเท่านั้นในกราฟที่มี degree เป็นเลขคี่ ได้แก่ จุด 2 และ 4 ซึ่งมี degree เป็น 3 ทั้งคู่ \enskip ดังนั้น กราฟดังกล่าวจึงมี Euler trail ที่เริ่มต้นที่จุด 2 และสิ้นสุดที่จุด 4 (หรือกลับกัน) โดยมีลำดับการเดินแบบหนึ่งที่เป็นไปได้ดังนี้
\[2 → 3 → 4 → 2 → 1 → 4\]
\end{example}

ถัดไป จะทำการพิสูจน์ทฤษฎีบทดังกล่าว ซึ่งจะกล่าวถึงวิธีการหา Euler tour หรือ Euler trail ด้วย
%
\begin{pf}[Proof of Theorem~\ref{thm:euler-tour-degree}]
\end{pf}

นอกจาก Euler tours และ Euler trails จะสามารถให้เหตุผลเกี่ยวกับปัญหาการเดินชมเมือง ดังที่ได้กล่าวไปแล้ว ยังสามารถประยุกต์ใช้งานกับปัญหาการวาดภาพโดยไม่ยกดินสอได้ด้วย ดังตัวอย่างต่อไปนี้

\begin{example}
ในแต่ละภาพวาดต่อไปนี้ เราสามารถแปลงแต่ละตำแหน่งที่มีเส้นเชื่อมกัน หรือตำแหน่งที่เส้นมีการหักมุม เป็นแต่ละจุดในกราฟ \enskip ทั้งนี้ สำหรับตำแหน่งที่เกิดจากการตัดกันของเส้นสองเส้น เราอาจจะพิจารณาว่าไม่เป็นจุดในกราฟก็ได้ แต่หากจะนับเป็นจุดในกราฟ สังเกตว่า จุดเหล่านี้จะมี degree เป็น 4 ซึ่งไม่ส่งผลลบต่อความเป็นไปได้ในการหา Euler trail หรือ Euler tour
\begin{tabularx}{\linewidth}{@{}cX@{}}
& กราฟนี้ไม่สามารถวาดโดยไม่ยกดินสอได้ เนื่องจากทุกจุดมี degree เป็น 3 (ยกเว้นจุดตรงกลาง ซึ่งมี degree เป็น 4 แต่เราอาจจะไม่นับตำแหน่งดังกล่าวเป็นจุดในกราฟ) กล่าวคือ กราฟนี้ไม่มีทั้ง Euler trail และ Euler tour \\
& กราฟนี้มี Euler tour เนื่องจากทุกจุดมี degree เป็น 2 โดยสามารถวาดรูปนี้ได้โดยลากเส้นตามเข็มหรือทวนเข็มนาฬิกาจนครบทุกเส้น \\
& กราฟนี้ไม่มี Euler tour แต่มี Euler trail เนื่องจากสองจุดล่างนั้นมี degree เป็น 3 ส่วนจุดอื่นๆ มี degree เป็นเลขคู่ทั้งหมด \enskip ดังนั้น หากจะวาดรูปนี้โดยไม่ยกดินสอ จะต้องเริ่มจากจุดล่างจุดใดจุดหนึ่ง แล้วสิ้นสุดที่จุดล่างอีกจุดหนึ่ง
\end{tabularx}
\end{example}
